%2multibyte Version: 5.50.0.2960 CodePage: 65001
% define the absolute path for table fragments

\documentclass[notitlepage,11pt]{article}%
\usepackage{amsmath}
\usepackage{natbib}
\usepackage{color}
\usepackage{geometry}
\usepackage[onehalfspacing]{setspace}
\usepackage{hyperref}
\usepackage{amsfonts}
\usepackage{amssymb}
\usepackage{caption}
\usepackage{tabularx}
\usepackage{graphicx}%
\setcounter{MaxMatrixCols}{30}
%TCIDATA{OutputFilter=latex2.dll}
%TCIDATA{Version=5.50.0.2960}
%TCIDATA{Codepage=65001}
%TCIDATA{CSTFile=aer.cst}
%TCIDATA{Created=Tuesday, July 23, 2013 22:04:26}
%TCIDATA{LastRevised=Friday, August 05, 2016 22:39:04}
%TCIDATA{<META NAME="GraphicsSave" CONTENT="32">}
%TCIDATA{<META NAME="SaveForMode" CONTENT="1">}
%TCIDATA{BibliographyScheme=BibTeX}
%TCIDATA{<META NAME="DocumentShell" CONTENT="Standard LaTeX\Blank - Standard LaTeX Article">}
%TCIDATA{Language=American English}
%TCIDATA{ComputeDefs=
%$\lambda$
%}
%BeginMSIPreambleData
\providecommand{\U}[1]{\protect\rule{.1in}{.1in}}
%EndMSIPreambleData
\newtheorem{theorem}{Theorem}
\newtheorem{acknowledgement}[theorem]{Acknowledgement}
\newtheorem{algorithm}[theorem]{Algorithm}
\newtheorem{assumption}{Assumption}
\newtheorem{axiom}[theorem]{Axiom}
\newtheorem{case}[theorem]{Case}
\newtheorem{claim}[theorem]{Claim}
\newtheorem{conclusion}[theorem]{Conclusion}
\newtheorem{condition}[theorem]{Condition}
\newtheorem{conjecture}[theorem]{Conjecture}
\newtheorem{corollary}{Corollary}
\newtheorem{criterion}[theorem]{Criterion}
\newtheorem{definition}{Definition}
\newtheorem{example}[theorem]{Example}
\newtheorem{exercise}[theorem]{Exercise}
\newtheorem{lemma}{Lemma}
\newtheorem{notation}[theorem]{Notation}
\newtheorem{problem}[theorem]{Problem}
\newtheorem{proposition}{Proposition}
\newtheorem{remark}[theorem]{Remark}
\newtheorem{solution}[theorem]{Solution}
\newtheorem{summary}[theorem]{Summary}
\newenvironment{proof}[1][Proof]{\noindent \textbf{#1.} }{\  \rule{0.5em}{0.5em}}
\renewcommand{\bibAnnoteFile}[1]{\IfFileExists{#1}{\begin{quotation}\noindent \texts c{Key:} #1\\
\textsc{Annotation:}\  \input{#1}\end{quotation}}{}}
\renewcommand{\bibAnnote}[2]{\begin{quotation}\noindent \textsc{Key:} #1\\
\textsc{Annotation:}\ #2\end{quotation}}
\newcommand \fnote[1]{\captionsetup{font=small}\caption*{#1}}
\geometry{left=1.5in,right=1.5in,top=1.5in,bottom=1.5in}
\RequirePackage{threeparttable}
\RequirePackage{booktabs}
\makeatletter
\def\input@path{{C:/Users/robsunch/Dropbox/Projects/FDI_panel/writing/paper/v1/}}
\makeatother
\graphicspath{{C:/Users/robsunch/Dropbox/Projects/FDI_panel/writing/paper/v1/figures/}}
\begin{document}

\title{A note on constructing panel data set on multinational production}
\author{Chang Sun%
%TCIMACRO{\TeXButton{thanks}{\thanks
%{I thank Natalia Ramondo for generously providing her data on multinational production.}%
%}}%
%BeginExpansion
\thanks
{I thank Natalia Ramondo for generously providing her data on multinational production.}%
%EndExpansion
\\\emph{Preliminary and Incomplete}}
\maketitle

\begin{abstract}
This note discusses the methods I use when constructing the panel data set on
multinational production in my paper.

\end{abstract}

\section{Introduction}

There is no ready-to-use panel data set on multinational production. I combine
OECD and Eurostat FATS data set to construct one. The primary goal of the data
set is to provide bilateral statistics on total non-financial MP sales,
employment and number of firms used in my job market project : the factor bias
in multinational production and the labor share. This note details the data
sources, choice of variables and extrapolation.%

%TCIMACRO{\TeXButton{tech_origin}{\input{tables/tech_origin.tex}}}%
%BeginExpansion
\input{tables/tech_origin.tex}%
%EndExpansion


\bibliographystyle{econometrica}
\bibliography{myLib160429}


\appendix{}

\section{Data sources}

\cite{ramondo_multinational_2015} use UNCTAD as a basic source of FATS
(foreign affiliate statistics) data. However, according to email exchange with
Masataka Fujita, the head of investment trends and issues branch at UNCTAD,
the organization has stopped producing the FATS series. In contrast, Eurostat
and OECD have expanded their effort providing FATS data. I use them as the
major data source for the construction of my panel data.

\subsection{Eurostat}

The following tables are downloaded from Eurostat and used in the construction
of my data.%

\begin{tabular}
[c]{lllll}%
Table & TableDes & Years & Variables & Industry\\
fats\_96/fats\_sum &  & 1999-2002 & V12110 & NACE Rev. 1.1 C-K\_X\_J\\
fats\_g1b\_03 & total inward MP activities & 2003-2007 & V12110 & NACE Rev.
1.1 C-K\_X\_J\\
fats\_g1b\_08 & total inward MP activities & 2008-2012 & V12110 & NACE Rev2
B-N\_S95\_X\_K\\
fats\_out1 & outward MP activities by broad industries & 2004-2006 & TUR &
NACE Rev1 A-O\_X\_L\\
fats\_out2 & outward MP activities by broad industries & 2007-2009 & TUR &
Nace Rev1 A-O\_X\_L\\
fats\_out2\_r2 &  & 2010-2012 & TUR & Nace Rev2 B-S\_X\_O
\end{tabular}


Variable definitions: 

\begin{itemize}
\item V12110: Turnover or gross premiums written; 

\item TUR: Turnover - Million ECU/EUR.
\end{itemize}

Industry definitions:

\begin{itemize}
\item NACE Rev. 1.1 C-K\_X\_J: Business economy - Industry and services
(except financial intermediation)

\item Nace Rev2 B-N\_S95\_X\_K: Foreign control of enterprises by controlling
countries (from 2008 onwards) NACE Rev 2

\item NACE Rev1 A-O\_X\_L: All NACE activities (except public administration;
activities of households and extra-territorial organizations)

\item Nace Rev2 B-S\_X\_O: Industry, construction and services (except public
administration, defense, compulsory social security)
\end{itemize}

Therefore, the inward FATS aggregate excludes the financial sector, while the
outward statistics include the financial sector, which needs adjustment.

For years before 2002, only 11 countries voluntarily report bilateral FATS
tables. Among them, Ireland and Germany do not report total values (sector
C-K\_X\_J). I aggregate industry level values to get the total. For Ireland, I
aggregate NACE Rev1.1 sectors \textquotedblleft C D E G H I
K\textquotedblright, requiring no missing values in any of the sectors. 

\subsection{OECD}

OECD also compiles data on FDI and multinational activities, based on
statistics reported by OECD countries. The multinational firm activity data
can be downloaded from OECD database -- Globalization -- Activity of
Multinationals. There are 12 tables in this section, and four of which focus
on bilateral aggregate statistics, which are used here.%

\begin{tabular}
[c]{lll}%
Table & Years & Industry\\
Inward, ISIC3 (service) & 1995-2008 & 01-93 excl. 75\\
Inward, ISIC4 & 2008-2013 & sec B to N excl. K\\
Outward, ISIC3 (service) & 1995-2009 & 01-93 minus 75\\
Outward, ISIC4 & 2007-2013 & sec B to S excl. O
\end{tabular}


Note that only inward statistics ISIC4 exclude the financial sector. All the
other statistics need adjustment.

For years before 2007 (including 2007), ISIC Rev 3 data provide most of the
information, while for years after 2007, ISIC Rev 4 data provide most of the
information. Thus I ignore ISIC Rev 3 data for 2008 and 2009, and ignore ISIC
Rev 4 data for 2007.

The monetary variables are in millions of local currency. According to the
meta data in the online database, 

\begin{quotation}
For Euro area countries, national currency data is expressed in euro beginning
with the year of entry into the Economic and Monetary Union (EMU). For years
prior to the year of entry into EMU, data have been converted from the former
national currency using the appropriate irrevocable conversion rate. This
presentation facilitates comparisons within a country over time and ensures
that the historical evolution is preserved. Please note, however, that pre-EMU
euro are a notional unit and should not be used to form area aggregates or to
carry out cross-country comparisons.
\end{quotation}

\subsubsection{Anomalies}

German data on outward MNE sales are outliers in 2007. It seems the
undocumented table \textquotedblleft Outward Activity of Multinationals in
ISIC Rev 3 (services)\textquotedblright\ present values that are 1000 times
larger. For example, in the OECD database, table \textquotedblleft Outward
Activity of multinationals by industrial sector (manufacturing) -- ISIC Rev
3\textquotedblright\ shows that Germany's total outward sales (Total Business
Enterprise) is 1.497 million Euros in 2007. This number, however, is 1.486
billion Euros in the first table. Thus, I rescale all values for Germany in
2007 by 1000 times. 

Slovenia presents another anomaly in the data. The grand total outward sales
to the WORLD are 3394715 million in local currency in 2007, and there seems to
be no break after that (Slovenia adopted Euro in 2007, and the table only
contains data for 2007-2009). I suspect that they are still denominated in
Slovenian Tolar rather than Euros. Thus for these three years, I transformed
the Euro exchange rate to notional Tolar exchange rate using the fixed
exchange rate for Solvenia.

\subsection{Measure of employment}

In OECD and Eurostat inward tables, there are two variables related to total
employment of multinational firms: (1) number of employees and (2) number of
persons employed. A detailed definition of these two concepts can be found on
Page 56-57 in the Eurostat Foreign AffiliaTes Statistics (FATS)
Recommendations Manual (2012 edition). The manual does not provide a direct
comparison between the two concepts but the key difference between the two
concepts seems to be that one is counted as an employee only when a contract
of employment is provided.

For country-pair-years which both variables are nonmissing, I can calculate
the difference between the two variables. Most of them are exactly the same,
and (2) is in general larger than (1). When possible, I use (2) as a measure
of employment and supplement (1) if (2) is missing or zero but (1) is positive.

\section{Construction}

Given MNE activity data from OECD and Eurostat, and FDI stocks and flows data
from OECD, Eurostat and UNCTAD, I try to impute some of the missing bilateral
MNE activity variables (mainly revenue, but employment is also considered). I
use the following steps.

\subsection{Consolidate three sources of FDI statistics}

\begin{enumerate}
\item Combine information in number of employees (n\_emp) and number of
persons employed (n\_psn\_emp). Use number of persons employed as the primary
source and supplement with number of employees. See the do file
\texttt{code/check\_data/emp\_vs\_psn\_emp.do} and the companion output for
the comparison between the two variables. Most of the numbers are the same,
but on average, number of employees are smaller than number of persons
employed since the former only include workers with employment contract.

\item Drop outliers in bilateral relationships defined using year-to-year
growth rates. I look for outliers within a home-host country pair. I compute
both the deviation from the log mean and the log change in a certain variable,
and define an observation to be an outlier if it satisfies two conditions (1)
the log change from last period
%TCIMACRO{\TEXTsymbol{>} }%
%BeginExpansion
$>$
%EndExpansion
5 or
%TCIMACRO{\TEXTsymbol{<} }%
%BeginExpansion
$<$
%EndExpansion
-5, or the log change into next period
%TCIMACRO{\TEXTsymbol{>}}%
%BeginExpansion
$>$%
%EndExpansion
5 or
%TCIMACRO{\TEXTsymbol{<} }%
%BeginExpansion
$<$
%EndExpansion
-5; (2) the deviation from the log mean
%TCIMACRO{\TEXTsymbol{>} }%
%BeginExpansion
$>$
%EndExpansion
5 or
%TCIMACRO{\TEXTsymbol{<} }%
%BeginExpansion
$<$
%EndExpansion
-5. Note that in this way, we first take care of the zeros since they won't
enter the candidates for outliers. Second, the observation adjacent to the
outlier is not likely to be misidentified as an outlier since it will be close
to the mean. There are a few scenarios in which I might fail to identify an
outlier: (1) if an outlier has no adjacent observations (2) if the value of
the outlier is large enough to make the average close to itself, so it does
not satisfy condition (2) and will not be identified as outlier. However, the
adjacent values might be identified as an outlier since they are likely to be
away from the mean. In the tables in \textquotedblleft%
\texttt{output/data\_management/tables/potential\_outliers.xlsx}%
\textquotedblright, I list all the country pairs which have at least one
observation satisfying condition (2).

\item Supplement missing data in Eurostat with OECD -- document them well.
First, identify countries that report Eurostat or OECD. If a country reports
in Eurostat, use Eurostat as the primary source. If a country reports in OECD,
use OECD as the primary source.

\item Impute additional zeros using FDI stocks.

If one of the Eurostat or OECD stock is non-positive, and the other is missing
or non-positive, impute the missing MNE activities (employment or revenue) to
be zero. If employment is zero, impute revenue to be zero too (very few observations).

\item Extrapolate still missing sales data with inward FDI stocks (and outward
revenue) using both cross-sectional and over time variation. For the period
1995 -- 2012, I run the following regression for host and home countries with
at least three observations%
\[
\log X_{ilt}=\beta\log Q_{ilt}+\delta_{i}+\delta_{l}+\delta_{i}\times
t+\delta_{l}\times t+\varepsilon_{ilt},
\]
where the dependent variable is the inward sales while the key independent
variable is either inward stock or outward sales. I estimate this regression
using different time periods and the coefficient $\beta$ seems very stable.
(see \texttt{output/data\_management/extrap\_activities.csv})

\item Next I consider using the time-series property of the data only and do
not bring in any new variable besides inward sales. I extrapolate over time
using a constant growth model within each pair. For country pairs with at
least 4 observations between 2001 and 2012, I estimate the following equation%
\[
logX_{ilt}=δ _{il}+δ _{t}+δ _{i}\times t+δ _{l}\times
t+ε _{ilt}%
\]


Note that I do not impose a pair-specific trend since the trend can be
imprecisely estimated with only a few observations within a pair. Instead, I
impose the growth rates to have host and home specific components, and also a
global trend, which is not restricted to be linear.

Besides the linear growth model extrapolation for observations with positive
sales, I also impute additional zeros using the time series data. I first
identify consecutive runs of missing values or zeros. I require the missing
values can potentially be replaced as zeros, i.e., it cannot have positive
stock and other MNE activities (inward employment, number of enterprises and
outward employment, sales and number of enterprises). \qquad After such runs
are identified, I identify missing values that are \textquotedblleft
squeezed\textquotedblright\ between two zeros. These values are replaced with zeros.
\end{enumerate}

\subsection{Extrapolation for total inward activities}

The extrapolation for total inward activities is a bit easier. The procedures
are similar to the extrapolation of bilateral activities. I describe the
procedures as follows.

\begin{enumerate}
\item Combine information in number of employees (n\_emp) and number of
persons employed (n\_psn\_emp). Use number of persons employed as the primary
source and supplement with number of employees.

\item Drop outliers in total inward/outward variables defined using
year-to-year growth rates

I compute both the deviation from the log mean and the log change in a certain
variable, and define an observation to be an outlier if it satisfies two
conditions (1) the log change from last period
%TCIMACRO{\TEXTsymbol{>} }%
%BeginExpansion
$>$
%EndExpansion
5 or
%TCIMACRO{\TEXTsymbol{<} }%
%BeginExpansion
$<$
%EndExpansion
-5, or the log change into next period
%TCIMACRO{\TEXTsymbol{>}}%
%BeginExpansion
$>$%
%EndExpansion
5 or
%TCIMACRO{\TEXTsymbol{<} }%
%BeginExpansion
$<$
%EndExpansion
-5; (2) the deviation from the log mean
%TCIMACRO{\TEXTsymbol{>} }%
%BeginExpansion
$>$
%EndExpansion
5 or
%TCIMACRO{\TEXTsymbol{<} }%
%BeginExpansion
$<$
%EndExpansion
-5. 

\item Supplement missing data in Eurostat with OECD -- document them well

First, identify countries that report Eurostat or OECD. If a country reports
in Eurostat, use Eurostat as the primary source. If a country reports in OECD,
use OECD as the primary source.

\item Impute additional zeros using FDI stocks

If one of the Eurostat or OECD stock is non-positive, and the other is missing
or non-positive, impute the missing MNE activities (employment or revenue) to
be zero. If employment is zero, impute revenue to be zero too (very few observations).

\item Extrapolate still missing sales data with inward FDI stocks and
employment using both cross-sectional and over time variation. For the period
1995 -- 2012, I run the following regression for host and home countries with
at least three observations%
\[
\log X_{lt}=\beta\log Q_{lt}+\delta_{l}+\delta_{l}\times t+\delta
_{t}+\varepsilon_{lt},
\]
where the dependent variable is the inward sales while the key independent
variable is either inward stock or employment. I estimate this regression
using different time periods and the coefficient $\beta$ seems very stable.
(see \texttt{output/data\_management/extrap\_tot\_in\_activities.csv})

\item Next I consider using the time-series property of the data only and do
not bring in any new variable besides inward sales. I extrapolate over time
using a constant growth model within each pair. For country pairs with at
least 6 observations between 2001 and 2012, I estimate the following equation%
\[
\log X_{lt}=\delta_{l}+\delta_{t}+\delta_{l}\times t+\varepsilon_{lt}.
\]
Besides the linear growth model extrapolation for observations with positive
sales, I also impute additional zeros using the time series data. I first
identify consecutive runs of missing values or zeros. I require the missing
values can potentially be replaced as zeros, i.e., it cannot have positive
stock and other MNE activities (inward employment, number of enterprises and
outward employment, sales and number of enterprises). \qquad After such runs
are identified, I identify missing values that are \textquotedblleft
squeezed\textquotedblright\ between two zeros. These values are replaced with zeros.
\end{enumerate}

\subsection{Inward v.s. Outward}

Ramondo, Rodriguez-Clare and Tintelnot (2015) give two reasons for using
outward sales as the primary source. First, they argue statistics reported by
th host country is more likely on \textquotedblleft immediate
owners\textquotedblright\ rather than \textquotedblleft ultimate beneficiary
owners (UBO)\textquotedblright. Second, sales reported by the host country may
be only for local sales and miss sales from all other countries. 

This may be true for their UNCTAD data (though I highly doubt since some of
the UNCTAD data should come from Eurostat and OECD). However, I cannot find
support for their arguments in FATS statistical manuals. For the first
argument, I found the following related points in the relevant manuals

\begin{itemize}
\item UNCTAD Manual

\begin{itemize}
\item Inward FATS: As far as possible, it is recommended that countries use
the UBO unit when compiling operational statistics for inward investment
(Volume 2, II.39)\footnote{The footnote in that paragraph reads: Out of 15
countries providing operational data in the OECD's Manual (OECD, 2001) eight
(Belgium, France, Germany, Japan, Luxembourg, Norway, Poland and Portugal) use
immediate foreign owner, and seven (Finland, Ireland, Italy, the Netherlands,
Sweden, the United Kingdom and the United States) use ultimate owner.}

\item Outward FATS: The second issue deals with the treatment of foreign
investments of those domestic enterprises, which are themselves foreign-owned.
This volume recommends that the compiling country should collect data for all
resident enterprises direct investor, regardless of where they are owned.
However, in its published statistics it should provide separate breakdowns for
the foreign affiliates of domestically and foreign-owned enterprises. (Volume
2, II.41)
\end{itemize}

\item FATS Manual (Eurostat)

\begin{itemize}
\item Inward FATS: Ultimate Control Institutional Unit (UCI) is recommended
(see I.1.1)

\item Outward FATS: Ultimate Control Institutional Unit (UCI) is recommended
(see I.1.1)
\end{itemize}
\end{itemize}

UNCTAD recommends that countries use the UBO unit when compiling operational
statistics for inward investment (activities), but IMF does require that BOP
statistics record transactions based on the immediate foreign owner (II.39,
II.40 in UNCTAD, 2008, Vol 2). On the contrary, UNCTAD recommends countries
report outward MNE activities based on immediate owners (II.41 (ii)). I simply
cannot find any information about their second argument.

\section{Descriptive statistics}

In this section I describe the data I have constructed.

\section{Tables}%

%TCIMACRO{\TeXButton{reset_counter}{\setcounter{table}{0}
%\renewcommand{\thetable}{A\arabic{table}}}}%
%BeginExpansion
\setcounter{table}{0}
\renewcommand{\thetable}{A\arabic{table}}%
%EndExpansion
%

%TCIMACRO{\TeXButton{frequency_by_host}{\newpage\input
%{tables/frequency_by_host.tex}}}%
%BeginExpansion
\newpage\input{tables/frequency_by_host.tex}%
%EndExpansion


\section{Figures}%

%TCIMACRO{\TeXButton{reset_counter}{\setcounter{figure}{0}
%\renewcommand{\thefigure}{A\arabic{figure}}}}%
%BeginExpansion
\setcounter{figure}{0}
\renewcommand{\thefigure}{A\arabic{figure}}%
%EndExpansion
%

%TCIMACRO{\TeXButton{cf_cbp_only}{\begin{figure}[ptbh]\caption
%{Counterfactual labor shares: CBP only}\label{fig:cf_labor_share_cbp_only}
%\centering\includegraphics[scale=0.9]{{cf_labor_share_cbp_only.pdf}%
%}\end{figure}}}%
%BeginExpansion
\begin{figure}[ptbh]\caption{Counterfactual labor shares: CBP only}%
\label{fig:cf_labor_share_cbp_only}
\centering\includegraphics[scale=0.9]{{cf_labor_share_cbp_only.pdf}%
}\end{figure}%
%EndExpansion

\end{document}