%2multibyte Version: 5.50.0.2960 CodePage: 65001
% define the absolute path for table fragments

\documentclass[notitlepage,11pt]{article}%
\usepackage{amsmath}
\usepackage{natbib}
\usepackage{color}
\usepackage{geometry}
\usepackage[onehalfspacing]{setspace}
\usepackage{hyperref}
\usepackage{amsfonts}
\usepackage{amssymb}
\usepackage{caption}
\usepackage{tabularx}
\usepackage{graphicx}%
\setcounter{MaxMatrixCols}{30}
%TCIDATA{OutputFilter=latex2.dll}
%TCIDATA{Version=5.50.0.2960}
%TCIDATA{Codepage=65001}
%TCIDATA{CSTFile=aer.cst}
%TCIDATA{Created=Tuesday, July 23, 2013 22:04:26}
%TCIDATA{LastRevised=Tuesday, August 02, 2016 09:49:56}
%TCIDATA{<META NAME="GraphicsSave" CONTENT="32">}
%TCIDATA{<META NAME="SaveForMode" CONTENT="1">}
%TCIDATA{BibliographyScheme=BibTeX}
%TCIDATA{<META NAME="DocumentShell" CONTENT="Standard LaTeX\Blank - Standard LaTeX Article">}
%TCIDATA{Language=American English}
%TCIDATA{ComputeDefs=
%$\lambda$
%}
%BeginMSIPreambleData
\providecommand{\U}[1]{\protect\rule{.1in}{.1in}}
%EndMSIPreambleData
\newtheorem{theorem}{Theorem}
\newtheorem{acknowledgement}[theorem]{Acknowledgement}
\newtheorem{algorithm}[theorem]{Algorithm}
\newtheorem{assumption}{Assumption}
\newtheorem{axiom}[theorem]{Axiom}
\newtheorem{case}[theorem]{Case}
\newtheorem{claim}[theorem]{Claim}
\newtheorem{conclusion}[theorem]{Conclusion}
\newtheorem{condition}[theorem]{Condition}
\newtheorem{conjecture}[theorem]{Conjecture}
\newtheorem{corollary}{Corollary}
\newtheorem{criterion}[theorem]{Criterion}
\newtheorem{definition}{Definition}
\newtheorem{example}[theorem]{Example}
\newtheorem{exercise}[theorem]{Exercise}
\newtheorem{lemma}{Lemma}
\newtheorem{notation}[theorem]{Notation}
\newtheorem{problem}[theorem]{Problem}
\newtheorem{proposition}{Proposition}
\newtheorem{remark}[theorem]{Remark}
\newtheorem{solution}[theorem]{Solution}
\newtheorem{summary}[theorem]{Summary}
\newenvironment{proof}[1][Proof]{\noindent \textbf{#1.} }{\  \rule{0.5em}{0.5em}}
\renewcommand{\bibAnnoteFile}[1]{\IfFileExists{#1}{\begin{quotation}\noindent \texts c{Key:} #1\\
\textsc{Annotation:}\  \input{#1}\end{quotation}}{}}
\renewcommand{\bibAnnote}[2]{\begin{quotation}\noindent \textsc{Key:} #1\\
\textsc{Annotation:}\ #2\end{quotation}}
\newcommand \fnote[1]{\captionsetup{font=small}\caption*{#1}}
\geometry{left=1.5in,right=1.5in,top=1.5in,bottom=1.5in}
\RequirePackage{threeparttable}
\RequirePackage{booktabs}
\makeatletter
\def\input@path{{C:/Users/robsunch/Dropbox/Projects/FactorIntensity/Writing/paper/v6/}}
\makeatother
\graphicspath{{C:/Users/robsunch/Dropbox/Projects/FactorIntensity/Writing/paper/v6/figures/}}
\begin{document}

\title{The Factor Bias in Multinational Production and the Labor Share}
\author{Chang Sun%
%TCIMACRO{\TeXButton{thanks}{\thanks
%{I thank Gene Grossman, Oleg Itskhoki, Eduardo Morales, Ezra Oberfield, Steve Redding and Esteban Rossi-Hansberg for helpful discussions.
%I thank Natalia Ramondo for generously providing her data on multinational production.
%Financial support from the International Economics Section at Princeton University
%is greatly appreciated.}}}%
%BeginExpansion
\thanks
{I thank Gene Grossman, Oleg Itskhoki, Eduardo Morales, Ezra Oberfield, Steve Redding and Esteban Rossi-Hansberg for helpful discussions.
I thank Natalia Ramondo for generously providing her data on multinational production.
Financial support from the International Economics Section at Princeton University
is greatly appreciated.}%
%EndExpansion
\\\emph{Preliminary and Incomplete}}
\maketitle

\begin{abstract}
How does multinational production (MP) affect the distribution of income
between capital and labor? This paper argues that multinational firms can
affect the labor shares in the host countries because their affiliates have
different capital intensity from the local firms. I first document two
empirical regularities: multinational affiliates' capital intensity is
positively correlated with their home countries' capital abundance, and large
firms are more capital intensive. I then incorporate two novel mechanisms into
an otherwise standard multi-country general equilibrium trade and MP model to
rationalize the empirical patterns. Calibrating the model to both firm-level
and aggregate moments of 37 countries, I conduct counterfactual analysis and
find that in the past decade, the increase in multinational activities leads
to sizable reduction in the labor shares. The increase in MP activities not
only explains a large fraction of the average decline in the labor shares, but
also explains its variation across countries.

\end{abstract}

\section{Introduction}

Multinational firms have been playing an increasingly prominent role in the
global economy. The ratio of multinational sales to world GDP increased from
23\% in 1990 to 54\% in 2008 (author's calculation based on numbers in Table
I.5, \cite{unctad_world_2011}). Policy makers worldwide, especially those in
developing countries, are interested in attracting more multinational
production (MP) since multinational firms usually have more advanced
production technologies and might benefit the host countries in various ways
(\cite{javorcik_does_2004}, \cite{harrison_trade_2010}). Following this line
of thinking, the new generation of quantitative models of MP studies its
implications focusing on the heterogeneity in firm's Hicks-neutral
productivity (e.g., \cite{arkolakis_innovation_2013},
\cite{tintelnot_global_2014}). The heterogeneity in factor usage of MP,
however, has received little attention, in spite of the large heterogeneity
between multinational firms and local firms in the host country.

To examine the implication of the factor bias of MP on aggregate outcomes
especially the division of income between capital owners and workers (a.k.a,
the labor shares), I document two empirical regularities about the
capital-labor ratio of firms in 22 countries, including multinationals and
local firms. First, within the same country of production and industry, firms
originating from capital abundant countries are more capital intensive, which
I refer to as the "technology origin" effect in this paper. Second, larger
firms are more capital intensive, which I refer to as the size effect.
Multinational firms can bring technologies of different factor bias into the
host countries either because their technologies have different technology
origins, or because they are larger firms that use more capital-intensive
production techniques.

Building on the technology origin and size effects, I develop a multi-country
general equilibrium model of MP and trade where firms are heterogenous not
only in their sizes but also in their capital-labor ratios. The model
introduces two novel mechanisms to match the two empirical regularities,
respectively. To match the technology origin effect, I allow the firm make
endogenous choices of production techniques from a menu of technologies with
different capital- and labor-augmenting productivities. To match the
correlation between firm size and capital intensity, I assume technologies are
capital-biased, i.e., more productive technologies use relatively more capital
than labor.\footnote{See \cite{burstein_international_2015} for a similar idea
applied to firms' skill intensities.} At the same time, the structure of the
model is rich enough to match aggregate statistics such as the bilateral MP
and trade shares. Therefore, the model can be disciplined by both firm level
and aggregate level statistics, and it is well suited to study the impact of
MP on factor prices and income shares through multinational firms' factor bias.

The model has rich implications for understanding the distributional
consequences of MP liberalization, both analytically and quantitatively. After
a reduction in inward MP frictions, the size effect reduces the relative
demand for labor (thus the equilibrium labor shares) because the increased
activity of large (multinational) firms shifts the distribution of firms
towards large and capital-intensive ones. The "technology origin" effect leads
to a change in the relative demand for labor because multinational firms
originating from countries with different capital abundance use inherently
different technology in terms of capital intensity. The technology origin
effect tends to reduce the labor shares in capital-scarce countries while
increase the labor shares in capital-abundant ones.

To understand how MP liberalization has impacted the labor shares in recent
years, I parameterize a 37-country version of the model to exactly match,
among other moments of the data, the firm-level technology origin effect and
size effect and aggregate MP and trade shares in 1996-2001. I then perform
counterfactual analyses to study the effect of a reduction in MP frictions
such that the MP shares match those in a later period, 2006-2011. Over the
decade, many countries in my sample, especially the less-developed Eastern
European countries, saw large increases in inward MP activities. Associated
with the influx of multinational activities, the average country's labor share
declined by 1.4 percentage points, which is comparable to the average decline
of labor shares in the data (1.8 percentage points). At the same time, the
predicted changes in labor shares also explain the differential pace of labor
share decline in the data. The predicted and realized values are positively
correlated, and the predicted values explain about 19\% of the variation in
the data.

My paper contributes to a large literature on international technology
diffusion through multinational production. (\cite{ramondo_trade_2013},
\cite{arkolakis_innovation_2013}, \cite{tintelnot_global_2014}) In these
papers, technologies are modeled as Hicks-neutral productivities which can be
used in production locations other than the home country. This paper differs
from the previous literature by introducing factor bias as an additional
dimension of the technology. Since multinational technologies have different
factor bias than the local technologies, MP not only impacts the efficiency of
production, but also alters the relative demand for factors, thus the income shares.

The technology origin effect in the paper is closely related to the recent
literature on directed technical change (\cite{acemoglu_patterns_2003};
\cite{acemoglu_labor-_2003}; \cite{acemoglu_offshoring_2012}) and an earlier
empirical literature on "inappropriate technology"
(\cite{mason_observations_1973}, \cite{morley_limited_1977}), which tries to
test whether multinational firms from advanced countries are using
"inappropriately" capital-intensive production technologies in the developing
countries. The key insight from the two literatures is that technologies are
catered to the factor prices in the country where they are most likely to be
applied. As a theoretical contribution, I embed the idea of endogenous
technology choice in a quantitative model of multinational production to
rationalize the technology origin effect in the data. On the empirical front,
comparing to the case studies in the 1970s, I use a much better firm level
dataset and better econometric techniques to provide evidence supporting the
"inappropriate technology" hypothesis.

The exploration of the size effect, on the other hand, contributes to the
literature on the implication of factor-biased productivities. In a recent
paper, \cite{burstein_international_2015} argue that trade liberalization
leads to an increase in skill-premium because more productive firms are more
skill intensive and trade reallocate factors towards more productive firms
within sectors, which they refer to as the "skill-biased productivity"
mechanism. Though it is well known that larger firms are more capital
intensive (see \cite{bernard_firms_2007}), previous research has not
considered the implication of the "capital-biased productivity" mechanism in a
setting of global firms. I embed this mechanism into a multi-country, general
equilibrium trade and MP model and quantify its importance in understanding
the distributional consequences of trade and MP liberalizations.

The counterfactual analyses show MP liberalization is crucial in understanding
the global decline of labor shares. \cite{karabarbounis_global_2014} document
a global decline in the labor shares in the past three decades and argue the
global trend can be explained by the decline in the prices of investment
goods. However, as \cite{oberfield_micro_2014} point out, mechanisms that work
solely through factor prices cannot account for the labor share's decline if
the elasticity of substitution between capital and labor is estimated using
their more micro-based approach. According to their analysis for US
manufacturing sector since 1970, the bias of technical change within
industries has increased and accounts for most of the decline in the labor
share. My paper can be viewed as a further attempt to understand the technical
change caused by forces of globalization. The technology origin effect and the
size effect are both important in understanding the direction of technical
change in countries hosting a lot of multinational production.

The predictions from the quantitative model are quite different from an old
literature on capital flows and income distribution (see
\cite{caves_multinational_2007} for a summary). That literature views MP as a
reallocation of capital: a net outflow (inflow) of capital can cause a
relative increase (decline) of capital rewards in the country of study.
However, as is shown in the paper, the factor bias of multinational firms can
lead to changes in the labor shares without net flows of capital. This also
shows the importance of using information on bilateral MP sales rather than
just the net flow of capital to predict the effect of MP on income distribution.

My paper also contributes to a small but growing literature on firm's
heterogeneity in input usage. Following the seminal work by Melitz (2003), the
literature has focused much on firms' heterogeneity in their Hicks-neutral
productivities. The recent literature has acknowledged firms' heterogeneity in
other dimensions such as input usage.\footnote{See, for example,
\cite{crozet_firm-level_2013}, \cite{joaquin_blaum_non-homothetic_2015} and
\cite{burstein_international_2015}. Meanwhile, a different but related
literature tries to empirically estimate factor-augmenting productivities
using techniques developed by \cite{olley_dynamics_1996}. See
\cite{ulrich_doraszelski_measuring_2015} and
\cite{hongsong_zhang_non-neutral_2015} for example.} I show that the capital
intensity of the firm is systematically correlated with its home country's
capital abundance and its own size. The quantitative model rationalizes both
empirical regularities and can be used to understand the distributional
consequences of MP. Of course, multinational firms may differ from domestic
firms in their relative usage of other inputs, such as skill intensity, which
my data unfortunately cannot speak to. However, my quantitative framework can
be used to analyze the impact of MP on the skill premium when data permits.

The remainder of the paper is organized as follows. In Section 2, I document
two empirical regularities that are consistent with the technology origin
effect and capital-biased productivity. I develop my model of firms'
technology choices, trade and MP decisions in the next section. I then
calibrate the model and perform counterfactual analysis in sections 4 and 5. I
conclude in Section 6. Proofs and additional results relegated to the appendix.

\section{Empirical Regularities\label{sec:empirical_regularities}}

In this section, I explore the determinants of firm's capital intensity using
the Orbis database which covers firms, including multinationals, from many
countries. I document two empirical regularities focusing on firms within a
narrowly-defined industry. First, firms' capital intensity is positively
correlated with their home countries' capital abundance, which I refer to as
the "technology origin effect". Second, large firms are more capital
intensive, which I refer to as the "size effects".

\subsection{Firm-level Data}

To explore the determinants of firm's capital intensity, I use Orbis, the
global firm level database maintained by Bureau van Dijk (BvD). The database
covers balance sheet and income statement information for millions of firms
all around the world. Moreover, it provides a unique opportunity to examine
multinational firms' capital intensity since BvD records ownership links
between firms and identifies the "Global Ultimate Owner" (GUO) of a firm when
there is sufficient information to construct the "ownership tree" of the firm.
For the data I downloaded from the database, the majority of the ownership
links are updated in 2013, and I use the balance sheet data in 2012 for the
analysis since the 2013 data from many countries had not been collected by BvD
at the time of my study.

Before any statistical analyses, I clean the data in several steps to (1)
exclude firms with missing or abnormal values in total assets, employment and
wage bill (2) exclude multinational affiliates located in or originating from
tax havens (3) drop host-country-industry cells and home countries with too
few observations. The detailed steps can be found in the online appendix.

The data cleaning procedures leave me with more than 2.6 million firms from 23
host and 24 home countries.\footnote{I define the "home" country of a
multinational affiliate to be the country of its GUO and the home country of a
firm not belonging to any multinational group to simply be where it operates.}
I identify a multinational affiliate if the nationality of the firm's GUO is
different from where the firm operates. Among the 2.6 million firms, about
60,000 belong to a multinational group while approximately 40,000 are in a
foreign country. As expected, large and developed countries such as the United
States and Germany are home to a large number of multinational firms in the
data. However, the data also covers multinationals from less-developed
countries such as Romania, Bulgaria and the Czech Republic. Detailed industry
codes (four-digit NACE Revision 2) allow me to focus on variation within
narrowly-defined industries. Together with firms operating only domestically,
the dataset provides a good opportunity to explore the heterogeneity in
capital intensity, especially that of multinational firms.

\subsection{Technology Origin Effect}

In this subsection I document the technology origin effect: firms originating
from capital-abundant countries are more capital intensive compared to firms
from labor-abundant countries, conditional on operating in the same host
country and industry. The key assumption is within a narrowly-defined
industry, firms operating in the same country face the same factor prices so
any residual differences in capital intensity reflect different factor biases
in their technologies. The idea of comparing firms producing in the same
country but with different home countries dates back to the "inappropriate
technology" literature in the 1970s (\cite{mason_observations_1973},
\cite{morley_limited_1977}). Equipped with a much larger and richer dataset, I
run the following regression%
\[
\log\left(  \frac{K_{f}}{wL_{f}}\right)  =\delta_{s\left(  f\right)  \times
l\left(  f\right)  }+\beta\log\left(  \frac{K_{i\left(  f\right)  }%
}{hL_{i\left(  f\right)  }}\right)  +X_{f}+\varepsilon_{f}.
\]


In the above regression, $K_{f}/wL_{f}$ is the capital-labor ratio of firm
$f$, while $s\left(  f\right)  $, $l\left(  f\right)  $ and $i\left(
f\right)  $ refer to the sector, host country and home country of the firm,
respectively. To construct the capital-labor ratio, I use the wage bill
instead of the number of employees, to control for different worker skills
across firms. \footnote{For the practice of using the wage bill to measure the
efficiency units of labor, see, for example, \cite{hsieh_misallocation_2009}.
As discussed in the calibration section, using wage bill as a measure of labor
also makes my estimate of the elasticity of substitution between capital and
labor comparable to that estimated in \cite{oberfield_micro_2014}.} I control
for the substitution between $K$ and $L$ due to factor prices and the
differences of technologies across industries using the
producing-country-industry fixed effects $\delta_{s\left(  f\right)  \times
l\left(  f\right)  }$.\footnote{In principle, the industry-country specific
fixed effects also allow industry-country specific factor prices, which is
assumed away in my quantitative model but allowed in the reduced form
exercises.} The key independent variable is the ratio of capital stock to
human capital $K_{i\left(  f\right)  }/hL_{i\left(  f\right)  }$, a measure of
the firm's home country capital abundance.\footnote{This variable is
constructed using data on real capital stock, an index of average human
capital and total employment from Penn World Table 8.0. I adjust the total
employment by human capital in order to better measure the efficiency units of
labor in each country, and also try to be consistent with the use of wage bill
on the left-hand side of the equation. Note that I use wage to adjust labor
for the dependent variable to account for the heterogeneity in the skill of
work force across firms within a production country while I adjust for the
average skill of workers in the home country using the PWT human capital
measure. I cannot use wage in each country to replace $h$ since factor prices
can differ across countries due to different factor abundance. In this case,
it is not proper to think of the factor price to be a good adjustment factor
for the quality of the factor, which is exactly the case in the parameterized
model.}%

%TCIMACRO{\TeXButton{tech_origin}{\input{tables/tech_origin.tex}}}%
%BeginExpansion
\input{tables/tech_origin.tex}%
%EndExpansion


Table \ref{tab:tech_origin} shows the technology origin effect $\beta$ is
positive and is of both statistical and economic significance. As is shown in
the baseline specification of column 1, the elasticity of firms' capital
intensity with respect to the capital abundance of the home country is 0.244,
with a standard error of 0.044.\footnote{To address potential correlation of
the error term among firms from the same home or host country, I cluster the
standard errors at both the home and host country level.} This means
conditional on the producing country and industry, a firm originating from US
is 33\% more capital intensive than a firm originating from Bulgaria, one of
the least capital abundant countries in my sample.

In columns 2-4, I show the results are not simply driven by selection bias.
Since more productive firms are more capital intensive
(\cite{bernard_firms_2007}, also see the "size effect" in the next subsection)
and there are likely selection into multinational production and into the
Orbis database, the technology origin effect in column 1 might be
over-estimated if either (1) the barrier to invest in foreign countries are
larger for multinational firms from capital-abundant countries so they are a
more selected group of firms or (2) small firms from capital-abundant
countries are more likely to be missing in the data. I address this problem by
controlling for different proxies of firms' size and productivity. Columns 2
and 3 control for the firm's revenue and material cost per dollar of wage
bill, respectively. The technology origin effect is approximately the same and
still significant at 5\% level. In column 4, I focus on subsidiaries of
multinational firms, including subsidiaries in their home countries. The
sample consists of mostly large firms and is less likely to subject to
selection bias caused by the Orbis database. Again, I find the coefficient to
be positive and its magnitude to be close to the baseline specification in
column 1.

In columns 5 and 6, I show the results are not driven by the possibility that
multinational subsidiaries from more capital abundant countries have access to
cheaper loans. One key assumption of the empirical test is that firms
producing in the same host country face the same factor prices, or,
conditional on the host country fixed effect, the factor prices are not
correlated with home country capital abundance. With the firm-level data at
hand, I construct a firm-level measure of borrowing rate, defined as 2009-2013
average of interest paid per dollar of total liability. I also use the
debt-to-equity ratio as an additional control for the firm's access to
external financing. Controlling the two firm-level measures of financing costs
reduces the technology origin effect a little but still leaves the
coefficients significant and economically meaningful.

The results are robust to alternative definitions of "technology origin"
countries. In the main specifications, I use the Global Ultimate Owner (GUO)
to define the home country of a multinational affiliate. In the data, the GUO
can be at the very top of the "ownership tree" and may not have much direct
interaction with the affiliate. Alternatively, I can look at controlling
shareholders\footnote{A controlling shareholder is a shareholder that has the
majority of shares of the affiliate in a particular layer.} within a certain
number of layers and also require the shareholders to be in the same industry
as the affiliates. The results can be found in \ref{tab:alter_tech_origin}%
\ and are largely unchanged.

In Table \ref{tab:tech_origin_by_ind}, I perform the regression in Column (4)
of Table \ref{tab:tech_origin} for each industry separately. Clearly, there is
heterogeneity across industries but the majority of the coefficients are
positive. For the largest two industries, manufacturing and wholesale/retail,
the technology origin effects are estimated to be positive and significant.
The results for wholesale/retail sector also suggests that the technology
origin effect is not only driven by quality specialization (firms from rich
countries produce higher quality goods thus are more capital intensive) since
\cite{nir_jaimovich_trading_2015} recently show that labor intensity, if
anything, is positively correlated with service quality in the retail industry.%

%TCIMACRO{\TeXButton{tech_origin_by_ind}{\input{tables/tech_origin_by_ind.tex}%
%}}%
%BeginExpansion
\input{tables/tech_origin_by_ind.tex}%
%EndExpansion


To summarize, firm-level evidence reveals that firms from capital abundant
countries are more capital-intensive thus the "technology origin" of the firm
is an important determinant of firm's capital intensity. This pattern is
clearly missing in models with only heterogeneity in the Hicks-neutral
productivities but it is consistent with the old literature of "inappropriate
technology".(\cite{mason_observations_1973}, \cite{morley_limited_1977}) As
discussed in section \ref{sec:model}, a natural explanation is that firms
develop technologies that are more "appropriate" domestically since in
expectation they produce most of the goods there. \ 

\subsection{Size Effect}

The second empirical regularity reveals a positive correlation between firm's
size and capital-labor ratio. I estimate the elasticity of firm's
capital-labor ratio with respect to its size, measured by revenue. I use
revenue as a measure of firm size because measures such as total assets and
labor are used to calculate the left-hand variable and measurement errors can
cause mechanical correlations. The previous section reveals that the
technology origin plays a crucial role in determining the capital-labor ratio,
so in Table \ref{tab:rev_coef} I control for home country fixed effect when
estimating the elasticity whenever possible. In all regressions, I also
include host-country-industry fixed effects to control for capital intensity
differences induced by factor price differences across countries and
technology differences between sectors. In practice, I control for the
interaction of the two sets of fixed effects to isolate the size effect within
a host-country, home-country and industry cell.

Column 1-3 in Table \ref{tab:rev_coef} estimate the elasticity for
non-multinational firms, multinational firms and all firms, respectively.
Since the sample in Column 1 only contains local firms, their home and host
countries are the same so I only control for country-industry fixed effects.
Despite the differences in the samples, all three regressions give similar
estimates. The elasticity of firm's capital-labor ratio with respect to its
size is between 0.05 and 0.07, indicating that larger firms are using more
capital-intensive production techniques.%

%TCIMACRO{\TeXButton{rev_coef}{\input{tables/rev_coef.tex}}}%
%BeginExpansion
\input{tables/rev_coef.tex}%
%EndExpansion


Table \ref{tab:rev_coef_by_ind} shows the corresponding estimates by sectors
where sectors are defined similarly as in Table \ref{tab:tech_origin_by_ind}.
The elasticity is clearly heterogeneous across industries. However, the
estimates are typically between 0.01 and 0.1. In the calibration, I pick the
number estimated using the subsample of multinational firms, which is a
central value in the range of industry-specific estimates.%

%TCIMACRO{\TeXButton{rev_coef_by_ind}{\input{tables/rev_coef_by_ind.tex}}}%
%BeginExpansion
\input{tables/rev_coef_by_ind.tex}%
%EndExpansion


\section{Model\label{sec:model}}

In this section, I introduce two novel mechanisms into an otherwise standard
multi-country general equilibrium MP and trade model
(\cite{arkolakis_innovation_2013}, henceforth ARRY) to rationalize the two
empirical regularities. The first mechanism allows firms to choose a pair of
capital- and labor-augmenting productivity endogenously before they make MP
decisions, which induces them to cater their technology to domestic factor
prices. The second mechanism is a form of technology-capital complementarity.
I assume the capital- and labor-augmenting productivities increase at
different paces with the overall efficiency of the firm such that more
advanced technologies are more capital intensive in the fashion of Burstein
and Vogel (2015). I refer to the second mechanism as "capital-biased
productivity" (CBP henceforth). Though the general model with CBP delivers no
analytical gravity equations, the special case without CBP generates similar
expressions for trade and MP shares as in ARRY despite the new mechanism of
endogenous technology choice. This allows me to highlight the technology
origin effect and characterize and the impact of multinational production in a
two-region (North and South) special case.

\subsection{Environment}

There are $N$ countries and each country $i$ is endowed with two factors of
production, capital $K_{i}$ and labor $L_{i}$. There is only one sector with a
continuum of firms, each producing a different variety and engaging in
monopolistic competition. Consumers have CES preferences so demand for a
particular variety available in country $i$ is%
\[
q\left(  \omega\right)  =\frac{X_{i}}{P_{i}^{1-\sigma}}p\left(  \omega\right)
^{-\sigma},\omega\in\Omega_{i},
\]
where $X_{i}$ is the total expenditure and $\Omega_{i}$ is the set of
varieties available in country $i$. The price index $P_{i}$ is%
\[
P_{i}=\left(  \int_{\omega\in\Omega_{i}}p_{i}\left(  \omega\right)
^{1-\sigma}d\omega\right)  ^{1/\left(  1-\sigma\right)  }.
\]


A firm pays an entry cost $F_{ei}$ to headquarter in country $i$. To introduce
endogenous choice of technology, I assume that at the same time of entry, the
firm develops a production technique characterized by a pair of capital- and
labor-augmenting productivities $\left(  a,b\right)  $ from a "technology
menu". The technology menu is a set $\Theta\equiv\left\{  \left(  a,b\right)
|\theta\left(  a,b\right)  \leq1\right\}  $ where function $\theta$ is
strictly increasing and continuously differentiable in $\left(  a,b\right)  $.
Its frontier captures the trade-off between choosing a technology with high
capital-augmenting productivity and a technology with high labor-augmenting productivity.

In the quantitative exercises, I use the "CES" technology menu from
\cite{caselli_world_2006} (also see \cite{oberfield_micro_2014}) where%
\[
\theta\left(  a,b\right)  =\left[  a^{1-\eta}+b^{1-\eta}\right]  ^{1/\left(
1-\eta\right)  }.
\]
The parameter $\eta$ governs the shape of the technology frontier thus the
trade-off between capital- and labor-augmenting productivities. Since both
factor-augmenting productivities reduce the marginal cost, a firm always
chooses $\left(  a,b\right)  $ on the technology frontier $\theta\left(
a,b\right)  =1$. This means a firm has to trade off between technologies with
high-$a$ or high-$b$. The larger the $\eta$, the less a firm has to sacrifice
one factor-augmenting productivity for the other. When $\eta\rightarrow
-\infty$, the trade-off is the strongest and the technology menu collapses to
a singleton $\left(  a,b\right)  =\left(  1,1\right)  $.

\subsection{The firm's problem}

The timing of a firm's decisions is as follows. First, they pay a fixed cost
of $F_{ei}$ to headquarter in country $i$ and choose a technique $\left(
a,b\right)  \in\Theta$. Second, they draw a Hicks-neutral "core productivity"
from a Pareto distribution%
\[
\phi\sim F_{i}\left(  \phi\right)  =1-\left(  \phi/\phi_{\min}\right)  ^{-k}.
\]
After $\phi$ is realized, the firm has to decide which market to serve. The
firm can serve the market by producing in any country and then export to the
destination country. Regardless of the production location, the firm has to
pay $F_{n}$ units of marketing costs to access market $n$. Finally, the
location specific Hicks-neutral productivities $\mathbf{z}=\left(  z_{1}%
,z_{2},\dots,z_{N}\right)  $ are drawn from independent Frechet
distributions\footnote{In the online appendix, I consider the case of
multivariate Frechet distributions where the productivity draws are correlated
across countries. However, that version of the model is observationally
equivalent to the version described here. Therefore, given the data we observe
(MP and trade shares), there is no loss of generality to assume independent
productivity draws.}%
\[
z_{l}\sim\exp\left(  -T_{il}z^{-\theta}\right)  \text{, }l=1,\dots,N\text{,}%
\]
and the firm decides from which production country to serve a particular
market that they have access to.

The structure of the two-tier productivity draws is borrowed from ARRY. It
allows for rich patterns of MP and trade. To serve a foreign market, a firm
can either export from the home country, or produce in the destination market
and sell locally, or produce in a third country and export from there ("bridge
MP"). Despite the richness of the model, it still generates analytical
gravity-type of expressions for MP and trade volumes under certain
assumptions, which are useful for deriving analytical insights in section
\ref{sec:toe_theory}.

Firm in country $l$ produces using capital and labor according to the CES
production function%
\[
q=\phi z_{l}\left(  \lambda_{k}^{1/\varepsilon}\left(  a\phi^{-\xi/2}K\right)
^{\frac{\varepsilon-1}{\varepsilon}}+\left(  1-\lambda_{k}\right)
^{1/\varepsilon}\left(  b\phi^{\xi/2}L\right)  ^{\frac{\varepsilon
-1}{\varepsilon}}\right)  ^{\frac{\varepsilon}{\varepsilon-1}},\xi\in\left(
0,2\right)  ,
\]
where $\lambda_{k}$ is a parameter to adjust capital share for all firms and
$\varepsilon$ is the elasticity of substitution between capital and labor. The
two new mechanisms to generate heterogeneous capital intensity can be seen
from the capital- and labor-augmenting productivities $a\phi^{-\xi/2}$ and
$b\phi^{\xi/2}$. First, firms can optimally choose $\left(  a,b\right)  $
which is the endogenous technology choice mechanism. Second, the "core
productivity" adjusts these terms differently, with elasticities $-\xi/2$ and
$\xi/2$, respectively. To further understand these terms, I can rewrite the
production function and put the core productivity $\phi$ inside the
parathenses so that the full capita- and labor-augmenting productivities
become%
\begin{equation}
\left(  a\phi^{1-\xi/2},b\phi^{1+\xi/2}\right)  . \label{full_factor_aug_prod}%
\end{equation}
Thus, when $\xi\in\left(  0,2\right)  $, both productivities increase with
$\phi$ but the labor-augmenting productivity increases relatively more.
Combining with an elasticity $\varepsilon$ below one, which is the case in my
calibration, firms with higher overall efficiency are more capital intensive
due to the second mechanism. I refer to it as "capital-biased productivity",
similar to the "skill-biased productivity" studied in
\cite{burstein_international_2015}.

The marginal cost to produce one unit of goods in country $l$ becomes%
\[
C_{l}\left(  \phi,\mathbf{z},a,b\right)  =\frac{1}{\phi z_{l}}\left(
\lambda_{k}\left(  \frac{r_{l}}{a\phi^{-\xi/2}}\right)  ^{1-\varepsilon
}+\left(  1-\lambda_{k}\right)  \left(  \frac{w_{l}}{b\phi^{\xi/2}}\right)
^{1-\varepsilon}\right)  ^{1/\left(  1-\varepsilon\right)  },
\]
which not only depends on the factor-augmenting productivities but also
depends on the factor prices $\left(  r_{l},w_{l}\right)  $ in the production
country. To serve market $n$ from country $l$, the firm also incurs iceberg
trade and MP costs%
\[
C_{iln}\left(  \phi,\mathbf{z},a,b\right)  =\gamma_{il}C_{l}\left(
\phi,\mathbf{z},a,b\right)  \tau_{ln},
\]
where $\gamma_{il}$ and $\tau_{ln}$ are iceberg MP and trade costs that
captures the friction in MP and trade, respectively. When a technology
originating from country $i$ is applied in a foreign country $l$, it becomes
less efficient and its marginal cost is scaled by $\gamma_{il}$. It is a
convenient way to capture various impediments that multinationals
face.\footnote{Many quantitative MP models adopt the iceburg MP costs. See
\cite{arkolakis_innovation_2013}, \cite{ramondo_trade_2013} and
\cite{tintelnot_global_2014}.} In the calibration, I discipline all the
bilateral costs with data on MP and trade.

The firm's problem can be summarized into three stages

\begin{enumerate}
\item Entry and technology choice $\left(  a,b\right)  ;$

\item The core productivity $\phi$ is realized and the firm decides which
markets to serve by paying the marketing costs;

\item The country specific productivities $z$ is realized and firm decide
where to produce and how to serve each market.
\end{enumerate}

The firm's problem can be solved backwards from the last stage. Given
technology choice $\left(  a,b\right)  $ and the set of markets that it has
access to, the firm chooses the country with the lowest cost of production%
\[
l=\arg\min_{m}C_{imn}\left(  \phi,\mathbf{z},a,b\right)  .
\]
Using the property of Frechet distribution, one can integrate over the
distribution of $z$ and obtain the the expected operating profit from market
$n$ at the second stage%
\[
\pi_{i\cdot n}\left(  \phi,a,b\right)  =\frac{\tilde{\sigma}^{1-\sigma}X_{n}%
}{\sigma P_{n}^{1-\sigma}}\Gamma\left(  \frac{\theta-\sigma+1}{\theta}\right)
\phi^{\sigma-1}\Psi_{in}\left(  \phi,a,b\right)  ^{\frac{\sigma-1}{\theta}}%
\]
where $\tilde{\sigma}\equiv\sigma/\left(  \sigma-1\right)  $, $\Gamma\left(
\cdot\right)  $ is the gamma function, $X_{n}$ is the total absorption in
market $n$, and $\Psi_{in}\left(  \phi,a,b\right)  $ is a term that summarizes
the profitability of market $n$ (for detailed derivations, see the online
appendix). Specifically,
\[
\Psi_{in}\left(  \phi,a,b\right)  \equiv\sum_{l}T_{il}\zeta_{iln}\left(
\phi,a,b\right)  ^{-\theta},
\]
where I define $\xi_{iln}\left(  \phi,a,b\right)  \equiv C_{l}\left(
\phi,\mathbf{z},a,b\right)  \phi z_{l}$ as the marginal cost normalized by
core productivity $\phi$ and country-specific productivity $z_{l}$, which
depends on the marginal cost of combined factors and trade and MP frictions.
Therefore, the profitability term $\Psi_{in}$ summarizes how (1) the
innovation capacity $T_{il}$ (2) input costs in all potential production
locations and (3) MP and trade frictions govern the expected profit in a
destination market at the second stage.

In the second stage, the firm chooses the markets that it will serve. Given
the expected operating profit in the last stage, a firm enters market $n$ if
and only if the expected profit from that market is larger than the $F$ units
of marketing costs, paid using the composite good available in the destination
market $n$%
\[
\pi_{i\cdot n}\left(  \phi,a,b\right)  \geq P_{n}F.
\]


Under the assumption $\xi<2$, a higher core productivity $\phi$ implies both
higher capital- and labor-augmenting productivity thus lower marginal costs in
all countries (see expressions \ref{full_factor_aug_prod}). Thus, we have the
following lemma

\begin{lemma}
There exists a unique cutoff $\phi_{in}^{\ast}\geq\phi_{\min}$ such that for
$\phi\geq\phi_{in}^{\ast}$ the firm enters market $n$ and it does not if
$\phi<\phi_{in}^{\ast}$.
\end{lemma}

Unlike \cite{arkolakis_innovation_2013}, however, there is no closed-form
expression for $\phi_{in}^{\ast}$ since $\phi$ affects the marginal cost not
only through the overall efficiency but also through the factor bias. When I
shut down the capital-biased productivity mechanism and set $\xi=0$, I recover
closed-form expression for $\phi_{in}^{\ast}$ and thus other aggregate
variables such as the trade and MP shares.

In the first stage, the firm chooses the optimal technology $\left(
a,b\right)  $ by maximizing the expected global profit%
\[
E_{\phi}\left[  \pi_{i}\left(  \phi,a,b\right)  \right]  \equiv E_{\phi
}\left[  \sum_{n}S_{in}\left(  \phi\right)  \left(  \pi_{i\cdot n}\left(
\phi,a,b\right)  -P_{n}F\right)  \right]
\]
where $S_{in}\left(  \phi\right)  $ indicates whether the firm decides to
serve market $n$ in the second stage%
\[
S_{in}\left(  \phi\right)  \equiv\mathbf{1}\left[  \pi_{i\cdot n}\left(
\phi,a,b\right)  \geq P_{n}F\right]  .
\]
It is clear that all firms from the same home country will face the same
technology choice problem in the first stage. In equilibrium the optimal
technology is home-country specific $\left(  a_{i},b_{i}\right)  $, which
determines the technology origin effects. This result comes from the timing
assumption that firms choose $\left(  a,b\right)  $ before any productivity
shocks are realized. It is the simplest way to allow firms to choose
technology endogenously since the only information that the firm has when
choosing $\left(  a,b\right)  $ is the location of its headquarter $i$.
However, together with additional structure on the MP costs, it can help to
match the technology origin effect as I observe in the data.

\subsection{Aggregation and equilibrium}

In this subsection, I derive expressions for aggregate variables and define
the general equilibrium of the model. The expressions are useful for the
calibration and those in a special case without the CBP mechanism ($\xi=0$)
can be used to study the technology origin effect analytically.

Similar to Burstein and Vogel (2015), there is no analytical gravity equation
due to the capital-biased productivity in this model (this can be seen from
the zero cutoff profit condition). Aggregate variables can only be expressed
in terms of integrations of firm level variables over the distribution of core
productivity $\phi$. Conditional on $\phi$ and the firm entering market $n$,
the probability that country $l$ becomes the lowest cost production location
is (see Appendix A for derivation)%
\[
\psi_{iln}\left(  \phi,a,b\right)  \equiv\frac{T_{il}\zeta_{iln}\left(
\phi,a,b\right)  ^{-\theta}}{\sum_{m}T_{im}\zeta_{imn}\left(  \phi,a,b\right)
^{-\theta}},
\]
and the expected sales from country $l$ to $n$ by affiliates owned by country
$i$ firms%
\[
X_{iln}\left(  \phi\right)  =\sigma\psi_{iln}\left(  \phi\right)  \pi_{i\cdot
n}\left(  \phi\right)  .
\]


The aggregate sales is an integration over all firms headquartered in country
$i$%
\[
X_{iln}=M_{i}\int S_{in}\left(  \phi\right)  X_{iln}\left(  \phi\right)
dF_{i}\left(  \phi\right)  .
\]
I define trade shares as the ratio of goods produced in country $l$ sold to
market $n$ by firms headquartered all around the world to the total absorption
in country $n$%
\[
\lambda_{ln}^{T}=\frac{\sum_{i}X_{iln}}{\sum_{i,l}X_{iln}}.
\]
And the MP shares as the share of output produced in country $l$ by firms
headquartered in country $i$ in the total output of country $l$%
\[
\lambda_{il}^{M}=\sum_{n}X_{iln}/Y_{l},
\]
where $Y_{l}\equiv\sum_{i,n}X_{iln}$ is the total output.

Consumers in market $n$ can purchase goods produced by firms from all
different origins thus the price index is%
\begin{equation}
P_{n}^{1-\sigma}=\tilde{\sigma}^{1-\sigma}\Gamma\left(  \frac{\theta-\sigma
+1}{\theta}\right)  \sum_{i}M_{i}\int S_{in}\left(  \phi\right)  \phi
^{\sigma-1}\Psi_{in}\left(  \phi\right)  ^{\frac{\sigma-1}{\theta}}%
dF_{i}\left(  \phi\right)  . \label{price_index}%
\end{equation}


\textbf{General} \textbf{Equilibrium} An equilibrium of the model is a vector
of $\left\{  \left(  a_{i},b_{i}\right)  ,r_{i},w_{i},P_{i},X_{i}%
,M_{i}\right\}  $ such that

\begin{enumerate}
\item Firms choose optimal technologies to maximize global expected profit%
\[
\left(  a_{i},b_{i}\right)  =\arg\max_{\left(  a,b\right)  \in\Theta}E_{\phi
}\left[  \pi_{i}\left(  \phi,a,b\right)  \right]
\]


\item Net profit is non-positive due to free entry%
\[
E_{\phi}\left[  \pi_{i}\left(  \phi,a,b\right)  \right]  -P_{i}F_{ei}\leq0
\]
and $E_{\phi}\left[  \pi_{i}\left(  \phi,a,b\right)  \right]  -P_{i}F_{ei}=0$
when $M_{i}>0$.

\item Capital and labor markets clear%
\begin{align*}
K_{i}  &  =\frac{1}{\tilde{\sigma}}\sum_{j,n}M_{j}\int S_{jn}\left(
\phi\right)  X_{jin}\left(  \phi\right)  \frac{\kappa_{ji}\left(  \phi\right)
}{r_{i}}dF_{j}\left(  \phi\right) \\
L_{i}  &  =\frac{1}{\tilde{\sigma}}\sum_{j,n}M_{j}\int S_{jn}\left(
\phi\right)  X_{jin}\left(  \phi\right)  \frac{1-\kappa_{ji}\left(
\phi\right)  }{w_{i}}dF_{j}\left(  \phi\right)
\end{align*}
where $\kappa_{ji}\left(  \phi\right)  $ is the capital share of firms
producing in $i$ from country $j$%
\[
\kappa_{ji}\left(  \phi\right)  =\frac{\frac{\lambda_{k}}{1-\lambda_{k}%
}\left(  \frac{a_{i}}{b_{i}}\right)  ^{\varepsilon-1}\left(  \frac{r_{l}%
}{w_{l}}\right)  ^{1-\varepsilon}\phi^{\xi\left(  1-\varepsilon\right)  }%
}{\frac{\lambda_{k}}{1-\lambda_{k}}\left(  \frac{a_{i}}{b_{i}}\right)
^{\varepsilon-1}\left(  \frac{r_{l}}{w_{l}}\right)  ^{1-\varepsilon}\phi
^{\xi\left(  1-\varepsilon\right)  }+1}.
\]


\item Goods market clear%
\[
X_{i}+\Delta_{i}=r_{i}K_{i}+w_{i}L_{i}+P_{i}\sum_{j}M_{j}F_{ji}E_{\phi}\left(
S_{ji}\left(  \phi\right)  \right)  +M_{i}P_{i}F_{ei}%
\]
where $\Delta_{i}$ is the exogenous current account surplus which I allow in
the quantitative exercise.

\item Price index satisfies equation (\ref{price_index}).
\end{enumerate}

\subsection{Determinants of capital intensity}

The model has rich implications for firm's capital intensity. Consider an
affiliate producing in country $l$ from country $i$. Solving the optimal input
usage problem I obtain%
\[
\frac{K_{il}\left(  \phi\right)  }{L_{il}\left(  \phi\right)  }=\frac
{\lambda_{k}}{1-\lambda_{k}}\left(  \frac{a_{i}}{b_{i}}\right)  ^{\varepsilon
-1}\phi^{\xi\left(  1-\varepsilon\right)  }\left(  \frac{r_{l}}{w_{l}}\right)
^{-\varepsilon}.
\]
Thus, a firm's capital intensity can be decomposed into the technology origin
effect $\left(  a_{i}/b_{i}\right)  ^{\varepsilon-1}$, a firm productivity
effect $\phi^{\xi\left(  1-\varepsilon\right)  }$ and the usual factor price
effect $\left(  r_{l}/w_{l}\right)  ^{-\varepsilon}$. One can view the choice
of $\left(  a_{i},b_{i}\right)  $ as the \emph{extensive substitution} between
capital and labor since the firm would respond to factor prices by optimizing
its technology choice in the first stage. Correspondingly, I call the usual
factor price effect $\left(  r_{l}/w_{l}\right)  ^{-\varepsilon}$ the
\emph{intensive substitution }since firms can adjust its capital intensity
after the technology $\left(  a_{i},b_{i}\right)  $ is chosen (see Oberfield
and Raval (2014)).

It is also clear from this equation that multinational firm data is crucial
for the identification of the intensive and extensive substitution. If we only
have local firms in multiple countries, the home and production countries are
always identical for each firm, and thus it is impossible to separately
identify the two margins of substitution. In this situation, the differences
in factor prices $\left(  r_{i},w_{i}\right)  $ leads firms to choose
different capital-labor ratio both because of the intensive substitution term
and its impact on the ex-ante technology choice $\left(  a_{i},b_{i}\right)
$. However, when we have data on multinational firms, it is possible to
separate the two margins because the dataset contains firms with $i\neq l$.

\subsection{Technology origin effect \label{sec:toe_theory}}

Under simplifying assumptions, I obtain sharper analytical results about the
technology origin effect. To highlight this mechanism, I assume there is no
capital-biased productivity in this subsection, i.e., $\xi=0$. Thus, there is
no heterogeneity in capital intensity between firms due to different $\phi$.
In this case, the only difference between my model and ARRY is that there are
two factors in my model and the firm can also choose $\left(  a,b\right)  $ ex
ante. Thus I restore all expressions for aggregate variables in ARRY (see the
online appendix). However, even with this assumption, the equilibrium in this
multi-country model is still hard to characterize. To obtain sharper
analytical results, I make the following additional assumptions

\begin{enumerate}
\item The world consists of two regions, North and South. Each region has a
number of symmetric countries. All Northern countries have endowments $\left(
K_{N},L_{N}\right)  $ while the endowments of the Southern countries are
$\left(  K_{S},L_{S}\right)  $. The North is more capital abundant
$K_{N}/L_{N}>K_{S}/L_{S}.$

\item Entry costs $F_{ei}$ and market access costs $F_{i}$ are the same within
each region.

\item MP and trade costs are symmetric%
\begin{align*}
\gamma_{ii}  &  =1,\gamma_{il}=\gamma>1\text{ for }i\neq l,\\
\tau_{ll}  &  =1,\tau_{ln}=\tau>1\text{ for }l\neq n.
\end{align*}


\item Trade balances $\Delta_{i}$ are the same within each region.

\item The intensive and extensive elasticities of substitution satisfy the
restriction $\varepsilon+\eta<2$.
\end{enumerate}

The last assumption ensures that the extensive substitution force is not too
strong for the firm to produce with only one input. Values of $\varepsilon$
and $\eta$ in my calibration satisfy this restriction and the sum is way below 2.

Under these additional assumptions, the model predicts a technology origin
effect - firms from the North choose a technology $\left(  a_{N},b_{N}\right)
$ that is more capital intensive than the Southern technology $\left(
a_{S},b_{S}\right)  $. The intuition comes from the fact that bilateral MP
costs are greater than one. It implies that production in other countries is
less efficient than that in the home country. Therefore, when choosing optimal
technology, firms give more weight to the expected revenue obtained from
producing in the home market. Firms originating in capital-abundant countries
will choose a more capital-intensive technology and vice versa for firms from
labor-abundant countries. The result resonates with the market size effect in
\cite{acemoglu_patterns_2003}, but is derived in a model of multinational
production where the barriers to MP play a crucial role.

With these simplifying assumptions, the following proposition characterizes
the partial equilibrium technology origin effect given factor prices:

\begin{proposition}
[Technology Origin Effect]\label{prop1} Assume foreign trade and MP costs
satisfy $\gamma\geq\tau>1$ or $\tau=\infty,\gamma>1$;

Then in a symmetric equilibrium (all equilibrium objects are the same within
each region)

\begin{enumerate}
\item the North has relatively cheap capital $r_{N}/w_{N}<r_{S}/w_{S}$

\item an optimal technology chosen by a Northern firm $\left(  a_{N}%
,b_{N}\right)  $ must be more capital-intensive than one chosen by a Southern
firm $\left(  a_{S},b_{S}\right)  $%
\[
\frac{K\left(  r_{l},w_{l};a_{N},b_{N}\right)  }{L\left(  r_{l},w_{l}%
;a_{N},b_{N}\right)  }\geq\frac{K\left(  r_{l},w_{l};a_{S},b_{S}\right)
}{L\left(  r_{l},w_{l};a_{S},b_{S}\right)  }.
\]
In terms of the factor augmenting productivities, $a_{N}/b_{N}\leq a_{S}%
/b_{S}$ when capital and labor are complements ($\varepsilon<1$) and
$a_{N}/b_{N}\geq a_{S}/b_{S}$ when capital and labor are substitutes
($\varepsilon>1$).

\item Northern firms enjoy a cost advantage in the North while Southern firms
enjoy a cost advantage in the South%
\begin{align*}
C\left(  r_{N},w_{N};a_{N},b_{N}\right)   &  \leq C\left(  r_{N},w_{N}%
;a_{S},b_{S}\right)  ,\\
C\left(  r_{S},w_{S};a_{S},b_{S}\right)   &  \leq C\left(  r_{S},w_{S}%
;a_{N},b_{N}\right)  .
\end{align*}

\end{enumerate}
\end{proposition}

\begin{proof}
See appendix.
\end{proof}

The model also provides a supply-side explanation for the fact that firms
invest relatively more in countries with income levels similar to their home
country (\cite{fajgelbaum_linder_2014}). When $\xi=0$, the normalized marginal
cost $\zeta_{iln}\left(  \phi,a_{i},b_{i}\right)  $ no longer depends on
$\phi$ and can be written as%
\[
\xi_{iln}\left(  a_{i},b_{i}\right)  =\gamma_{il}C_{l}\left(  a_{i}%
,b_{i}\right)  \tau_{ln}\text{.}%
\]
Similar to the iceberg MP cost $\gamma_{il}$, the middle term $C_{l}\left(
a_{i},b_{i}\right)  $ is also home-host country specific. Though the exogenous
MP costs $\gamma_{il}$ are symmetric (same for within-region MP and
cross-region MP), the endogenous choice of $\left(  a_{i},b_{i}\right)  $
leads to differences in $C_{l}\left(  a_{i},b_{i}\right)  $ for within-region
MP and cross-region MP because of the mismatch between technologies and factor
prices between the two regions. This creates an endogenous barrier of MP
between the North and the South, which provides a supply-side explanation for
the fact that there is relatively more investment between countries with
similar level of development (\cite{fajgelbaum_linder_2014}).

Part 1 of the proposition establishes that the scarce factor is indeed more
expensive, which is consistent with the prediction from a one-sector,
two-factor, closed-economy model, despite the complication of technology
choice, trade and MP. The intuition of the result can be illustrated as
follows. Suppose the South has a relatively lower capital rental rate. Then
Southern firms will adopt a relatively capital-intensive technology according
to Lemma \ref{prop1}. Moreover, since firms do not incur the MP cost at home,
there are relatively more Southern firms operating in the South. These two
results establish that, at the extensive margins (choices of technology and
location of production), relative demand for capital is higher in the South.
The intensive substitution further strengthens the relative demand for capital
in the South. This, however, contradicts the factor market clearing conditions.

What is the impact of MP in a world where firms develop technologies that
cater to domestic prices as in \ref{prop1}? The following proposition states
that the relative prices $r/w$ diverge after MP liberalization under
additional assumption.

\begin{proposition}
\label{prop3}Assume

\begin{enumerate}
\item trade is frictionless $\tau=1$,

\item the restriction $\varepsilon<k/\left(  1-\rho\right)  +1$ is satisfied

\item optimal technology choice is unique
\end{enumerate}

Then if MP becomes frictionless (from $\gamma>1$ to $\gamma=1$), the relative
factor prices $r/w$ in the two regions will diverge.
\end{proposition}

\begin{proof}
See appendix.
\end{proof}

The intuition for the proposition comes from the fact that the "total
elasticity" of substitution is a combination of the extensive and the
intensive\ elasticities of substitution. The liberalization in MP eliminates
the home-market effect in technology choice, and all firms adopt the same
technology. The extensive substitution no longer adjusts factor prices. When
the "total" elasticity of substitution drops, to make the factor markets clear
in both regions, the relative factor prices have to diverge.

The results on factor prices contrast with those in \cite{helpman_simple_1984}
which predict that multinational activities lead to an expansion of the factor
price equalization (FPE) set and thus can make the relative factor prices
converge. The \cite{helpman_simple_1984} model focuses on vertical FDI and the
separation of production and headquarter services adds another "sector" into
the economy. This causes the expansion of the FPE set since now the country
rich in capital can substitute into the most capital-intensive sector -
headquarter services. My model studies horizontal FDI only and the possibility
that firms tailor their technology to their global investment opportunities
leads technology to diverge when there are barriers to MP. Thus, compared to
the case where MP is frictionless, the "total elasticity" is larger in the
frictional world and the required factor prices to clear the factor markets
are less different across regions.

\section{Calibration}

To understand the quantitative importance of the technology origin effect and
capital-biased productivities, I calibrate the model to match both firm-level
and 1996-2001 aggregate data for 37 countries including both developed and
developing countries, which represent 91\% of world GDP, 95\% of world inward
FDI stocks and 99\% of world outward FDI stocks in my data source.

I estimate the technology origin effect and the firm size effect on firms'
capital intensity to discipline the shape of the technology menu $\eta$ and
the capital-biased productivity parameter $\xi$. The model also suggests a
firm-level regression for estimating the usual elasticity of (intensive)
substitution $\varepsilon$. I target the other parameters of the model (trade
costs and MP costs) to aggregate moments so that the trade and MP shares are
perfectly matched. Since the calibration involves firm-level statistics
(regression coefficients) which cannot be obtained using numerical
integration, I use simulated method of moments to calibrate the model. I then
discuss the model's fit in terms of additional statistics that I do not
target, with a focus on the factor prices since they are also the main outcome
variables I examine in the counterfactuals.

\subsection{Parameters calibrated without solving the model}

Two of the parameters are calibrated without solving the model. For the demand
elasticity $\sigma$, I simply follow ARRY and calibrate the demand elasticity
$\sigma$ to be 4, which is a common value in the literature (see also
\cite{bernard_plants_2003}). For the elasticity of substitution in the CES
production function, I directly estimate it from the structural equation of
firm's relative demand for capital and labor%
\[
\frac{K_{f}}{L_{f}}=\phi^{\left(  1-\varepsilon\right)  \xi}\left(
\frac{a_{i}}{b_{i}}\right)  ^{\varepsilon-1}\left(  \frac{r_{l}}{w_{l}%
}\right)  ^{-\varepsilon},
\]
where $f$ is an affiliate, and $i$ and $l$ denote the home and host countries
as before. According to the model, both the core productivity $\phi$ and the
endogenous choice of technology $\left(  a_{i},b_{i}\right)  $ are specific to
a parent firm. Therefore I can control these unobservables with a parent firm
fixed effect. The extent to which the affiliates adjust their capital-labor
ratio across production locations $l$ is then informative about the intensive
substitution $\varepsilon$.

In practice, I follow \cite{oberfield_micro_2014} and run the following
regression for multinational affiliates%
\[
\log\left(  \frac{r_{l\left(  f\right)  }K_{f}}{wL_{f}}\right)  =\left(
1-\varepsilon\right)  \log\left(  \frac{r_{l\left(  f\right)  }}{w_{l\left(
f\right)  }}\right)  +\delta_{g\left(  f\right)  \times s\left(  f\right)
}+u_{f}\text{,}%
\]
where $l\left(  f\right)  ,g\left(  f\right)  $ and $s\left(  f\right)  $
denote the host country, parent firm and industry of an affiliate. There are
several differences between the estimation equation and the structural
equation implied by the model. First, the model abstracts from multiple
industries so here I add industry fixed effects to control for technology
differences across industries. Second, I use the total wage bill reported by
the affiliate $wL_{f}$ instead of multiplying country-level average wage
$w_{l}$ with the number of employees of the affiliate $L_{f}$ to control for
worker skill differences across affiliates. Finally, since the host-country
rental rate (backed out using the labor share data from
\cite{karabarbounis_global_2014}) appears both on the left- and right-hand
side of the equation, I instrument $\log\left(  r_{l\left(  f\right)
}/w_{l\left(  f\right)  }\right)  $ with endowment $\log\left(  K_{l\left(
f\right)  }/L_{l\left(  f\right)  }\right)  $ to avoid mechanical correlation
caused by measurement error in $r_{l\left(  f\right)  }$.

Another issue with this regression is that the data lacks a measure of real
capital stock $K_{f}$. I construct a host-country-specific asset deflator and
then deflate firms' total assets using the deflator. The asset deflator
assumes the firm's capital stock has been growing at a constant rate (same as
that of the national aggregate capital stock) for a 10-year period. Together
with a constant growth rate of investment prices and deflation rate, I can
derive an expression for the asset deflator (see the appendix for detailed
derivation and data used in each component). I also experiment with different
assumptions on firms' ages and the estimated elasticity does not vary much
(see Table \ref{tab:elas_int}). My preferred IV estimate is%
\[
\widehat{\log\left(  \frac{r_{l\left(  f\right)  }K_{f}}{wL_{f}}\right)
}=\underset{\underset{\left(  0.108\right)  }{0.453}}{\left(
\widehat{1-\varepsilon}\right)  }\log\left(  \frac{r_{l\left(  f\right)  }%
}{w_{l\left(  f\right)  }}\right)  +\delta_{g\left(  f\right)  \times s\left(
f\right)  },
\]
which suggests the intensive elasticity of substitution is 0.547 with a
standard error 0.108. The first stage is strong with an F statistic 80.2. The
estimate is well in line with those in Oberfield and Raval (2014) who use a
similar cross-sectional approach to identify $\varepsilon$.

\subsection{Parameters calibrated to match endogenous outcomes from the model}

All the other parameters of the model, $\gamma_{il}$, $\tau_{ln}$, $k$,
$\theta$, $\eta$, $\xi$, $\lambda_{k}$ are calibrated to match endogenous
outcomes from the model, which I describe in the rest of this subsection. The
location parameter of the Frechet distribution $T_{il}$ cannot be separately
identified from the iceberg MP costs using the moments I target so I normalize
$T_{il}$ to 1 for all $i$ and $l$. I also normalize the lower bound of the
core productivity $\phi_{\min}$ and the marketing costs $F$ to 1. The entry
costs $F_{ei}$ affects the mass of firms in each country, which does not have
any clear empirical counterparts. Together with the normalization of $F$,
$F_{ei}$ will determine the probability of firms serving its home market: if
$F_{ei}$ is high, there are few firms compete in the domestic market and a
large fraction of the entrants will serve the domestic market. This again has
no clear empirical counterparts and I simply assume this fraction is 0.7 in
all the countries in my sample. Experimenting with lower values, I find the
normalization does not affect the calibration of other parameters and the
counterfacutals much.\footnote{Results are available upon requests.} I pick a
relatively high number such that I do not need to simulate too many firms to
generate a large enough panel in the simulated method of moments procedures,
minimizing the computational burden.

\textbf{Trade and MP shares: }I target the trade and MP costs $\left\{
\tau_{ln},\gamma_{il}\right\}  $ to the trade and MP shares $\left\{
\lambda_{ln}^{T},\lambda_{il}^{M}\right\}  $ (see equation ), normalizing the
domestic costs $\tau_{ii}$ and $\gamma_{ii}$ to 1. I obtain the trade flows
$X_{\cdot ln}$ from BACI and the MP sales $X_{il\cdot}$ from
\cite{ramondo_multinational_2015}. For countries with missing nonfinancial
total output $Y_{l}$ I extrapolate their GDP using a log-linear equation. All
country pairs in my sample have positive bilateral trade flows while some of
them have zero MP sales. I simply assign MP costs $\gamma_{il}$ to be infinity
for these countries. Detailed data sources and the extrapolation results can
be found in the online appendix.

\textbf{Average labor share: }The parameter $\lambda_{k}$ is common across
countries and determines the average labor share. The higher \ $\lambda_{k}$
is, the lower the labor shares. Among the sample countries, the average labor
share is 0.52, and I target $\lambda_{k}$ to match this number. The calibrated
$\lambda_{k}$ is 0.298.

\textbf{Restricted and Unrestricted Trade Elasticities}: As is shown in
\cite{arkolakis_innovation_2013}, the Pareto shape parameter $k$ and the
Frechet shape parameter $\theta$ are well disciplined by the "unrestricted"
and "restricted" trade elasticities, respectively. The "restricted" trade
elasticity $\beta^{r}$ is estimated using the three-way sales%
\[
\log X_{iln}=\delta_{il}+\delta_{in}-\beta^{r}\tau_{ln}+u_{iln}.
\]
In ARRY, $\beta^{r}$ equals the Frechet shape parameter $\theta$ regardless of
other model parameters. In my model, because of the CBP mechanism, there is no
analytical gravity and $\beta^{r}$ can be different from $\theta$. I use the
estimate $\hat{\beta}^{r}=10.9$ from ARRY but calibrate my model such that the
estimated $\hat{\beta}^{r}$ using the model generated three-way sales
$X_{iln}$ and calibrated $\tau_{ln}$ matches 10.9. The "unrestricted" trade
elasticity $\beta^{u}$ is estimated using the usual two-way trade flows%
\[
\log X_{\cdot ln}=\delta_{l}+\delta_{n}-\beta^{u}\tau_{ln}+u_{ln}.
\]
Again I use the empirical estimate $\hat{\beta}^{u}=4.3$ from ARRY and ensure
that $\beta^{u}$ estimated using data $X_{\cdot ln}$ and calibrated $\tau
_{ln}$ matches 4.3. The calibrated $\theta$ is 10.927, very close to
$\hat{\beta}^{r}$, which suggests that $\beta^{r}$ still pins down $\theta$
tightly despite the complication of the CBP mechanism. The calibrated $k$ is
4.184, which is also very close to the calibrated value in ARRY (4.0).

\textbf{Technology origin effect and size effect:} As discussed before, the
elasticity of extensive substitution $\eta$ governs the shape of the
technology menu thus the strength of the technology origin effect; the
parameter $\xi$ governs how the core productivity $\phi$ affects capital- and
labor-augmenting productivities differently, thus determines the size effect.
In section \ref{sec:empirical_regularities}, I run different specifications to
check the robustness of the two effects. In my calibration, I pick one set of
estimates and check sensitivity of the calibration and counterfactuals later.
Particularly, I use the estimates based on (1) the multinational subsample and
(2) affiliates whose home and host countries are all in my 37 country sample:%
\begin{equation}
\log\left(  \frac{K_{f}}{wL_{f}}\right)  =\delta_{s\left(  f\right)  \times
l\left(  f\right)  }+\underset{\left(  0.108\right)  }{0.274}\log\left(
\frac{K_{i\left(  f\right)  }}{hL_{i\left(  f\right)  }}\right)
+\underset{\left(  0.0143\right)  }{0.0575}\log\left(  X_{f}\right)
+\varepsilon_{f}, \label{firm_level_reg}%
\end{equation}
where $X_{f}$ is the affiliate's revenue.

Different from previous moments, the TOE and size effect are regression
coefficients from an affiliate level dataset. Therefore, for each guess of
model parameters, I solve for the general equilibrium variables and then
simulate a panel of multinational affiliates (see details in the next
subsection). I then run the above regression in the simulated data and adjust
parameters $\eta$ and $\xi$ to match the two regression coefficients. The
calibrated value of $\eta$ is 0.693 and the value of $\xi$ is 0.658 as shown
in Table \ref{tab:calib_baseline}.%

%TCIMACRO{\TeXButton{calib_baseline}{\input{tables/calib_baseline.tex}}}%
%BeginExpansion
\input{tables/calib_baseline.tex}%
%EndExpansion


\subsection{Algorithm}

The general equilibrium model involves six equilibrium variables $\left(
\delta_{i},r_{i},w_{i},P_{i},X_{i},M_{i}\right)  $ for each country where I
define $\delta_{i}$ to be%
\[
\delta_{i}\equiv\left(  \varepsilon-1\right)  \log\left(  a_{i}/b_{i}\right)
\text{.}%
\]
Since a firm always chooses $\left(  a_{i},b_{i}\right)  $ on the technology
frontier $\theta\left(  a_{i},b_{i}\right)  =1$, I can express $a_{i},b_{i}$
as functions of $\delta_{i}$. As is shown in Table \ref{tab:calib_baseline}, I
need to calibrate $2N\left(  N-1\right)  $ parameters of trade and MP costs,
$N$ parameters of entry costs $F_{ei}$ and five additional parameters $\left(
\eta,\xi,k,\rho,\lambda_{k}\right)  $ by solving the model.

I calibrate the model with a two-loop procedure. Given a set of outer loop
parameters $\left(  \eta,\xi,k,\theta,\lambda_{k}\right)  $, I iterate over
guesses of $\left(  \delta_{i},r_{i},w_{i},P_{i},X_{i},M_{i},F_{ei}\right)  $
and the trade and MP costs $\left\{  \tau_{ln},\gamma_{il}\right\}  $ such
that (1) all equilibrium conditions are satisfied and (2) trade and MP shares
are exactly the same as those in the data and (3) the probability of a firm
serving its domestic country is 0.7 in all countries. Note that to solve this
inner loop, I need to perform numerical integrations to obtain aggregate
variables such as sales and factor demand. I use a 20-node Gauss--Legendre
quadrature to obtain high precision. Given the large number of parameters, I
use an adjustment approach (also see \cite{burstein_international_2015}) to
reduce the computational burden. Intuitively, I increase prices if there are
excess demands, increase trade and MP costs if the trade and MP shares are too
high, and increase $F_{ei}$ if the probability of a firm serving its domestic
country is higher than 0.7. The detailed algorithm can be found in the online appendix.

The outer loop iterates over guesses of $\left(  \eta,\xi,k,\theta,\lambda
_{k}\right)  $ until the five corresponding moments are exactly matched. The
labor shares are the most straightforward to calculate. The restricted and
unrestricted trade elasticities are estimated using $X_{iln},X_{\cdot ln}$ and
calibrated $\tau_{ln}$ obtained from the inner loop. To obtain the technology
origin effect and size effect from the model, I simulate 20,000 firms in each
country. Each firm is characterized by its home country $i$, its core
productivity $\phi$ and a vector of productivity shocks $\mathbf{z}$. Firms
choose the markets to serve and from which country to serve a particular
market according to the model. Thus, for each firm, I can determine the size
of their affiliate in each country $l$. In the end, I obtain a panel of firms.
Similar to the data, some of the firms are multinationals while some only
operate in their domestic countries. I run the firm-level regression
\ref{firm_level_reg} and estimate the two coefficients using the simulated
data for the same set of home and host countries. The outer loop again uses an
intuitive adjustment approach as the inner loop. The entire calibration
typically takes from one hour to ten hours on a 20-core cluster, based on the
choice of the initial guesses.

\subsection{Model Fit}

The calibration produces thousands of iceberg trade and MP costs $\gamma_{il}$
and $\tau_{ln}$. Figure \ref{fig:gravity} plots the calibrated log values
against log of distance between countries. As expected, there is a strong
correlation between the iceberg costs and distance. There also seems to be
more variation in the trade costs than the MP costs. One reason why the
average MP cost seem to be smaller than the average trade cost is that for
country pairs with zero bilateral MP sales I set $\gamma_{il}=\infty$ and they
are not represented on the graph, while in my sample all bilateral trade flows
are positive and very small trade flows can imply very large trade costs
$\tau_{ln}$.%

%TCIMACRO{\TeXButton{gravity}{\begin{figure}[ptbh]\caption{Gravity of $\tau
%_{ln}$ and $\gamma_{il}$}\label{fig:gravity}
%\centering\includegraphics[scale=0.8]{{gravity.pdf}}\end{figure}}}%
%BeginExpansion
\begin{figure}[ptbh]\caption{Gravity of $\tau_{ln}$ and $\gamma_{il}$}%
\label{fig:gravity}
\centering\includegraphics[scale=0.8]{{gravity.pdf}}\end{figure}%
%EndExpansion


The calibrated value of $\eta$, 0.693, shows there is at least some extensive
substitution, which can be expected if I want to match the technology origin
effect as in the data. To get a sense of how large the extensive substitution
is, I consider firms in a closed economy, choosing both $\left(  a,b\right)  $
and $\left(  K,L\right)  $ given the domestic factor prices $\left(
r,w\right)  $. As is shown in Oberfield and Raval (2014), one can define a
"total elasticity" $\varepsilon^{tot}$ as the response of the relative factor
demand to factor prices, taking into account the firm's endogenous response
through $\left(  a,b\right)  $. Using the first-order conditions, one has%
\begin{align*}
\frac{K}{L}  &  =\left(  \frac{a}{b}\right)  ^{\varepsilon-1}\left(  \frac
{r}{w}\right)  ^{-\varepsilon}\\
&  =\left(  \frac{A}{B}\right)  ^{\varepsilon^{tot}-1}\left(  \frac{r}%
{w}\right)  ^{-\varepsilon^{tot}},
\end{align*}
where $\varepsilon^{tot}$ is a combination of the intensive and extensive
margins%
\[
\frac{1}{\varepsilon^{tot}-1}=\frac{1}{\varepsilon-1}+\frac{1}{\eta-1}.
\]
The calibrated values of $\varepsilon$ and $\eta$ suggests the total
elasticity is 0.82. Thus, without the extensive margin, the estimated
elasticity $\varepsilon=0.547$ will underestimate the "total elasticity" by
0.27. Therefore, ignoring the extensive elasticity would predict much higher
dispersion in $r_{l}/w_{l}$ across countries.%

%TCIMACRO{\TeXButton{eqm_price_fit}{\begin{figure}[ptbh]\caption
%{Model Fit - Factor Prices}\label{fig:eqm_price_fit}
%\centering\includegraphics[scale=0.8]{{eqm_price_fit.pdf}}\end{figure}}}%
%BeginExpansion
\begin{figure}[ptbh]\caption{Model Fit - Factor Prices}\label
{fig:eqm_price_fit}
\centering\includegraphics[scale=0.8]{{eqm_price_fit.pdf}}\end{figure}%
%EndExpansion


The model fits the relative factor prices in each country well. As Figure
\ref{fig:eqm_price_fit} shows, the predicted values of $\log\left(
r/w\right)  $ are highly correlated with those in the data, with a correlation
coefficient as high as 0.9. Though the match is not perfect, one can be
confident that the model captures the broad variation in factor prices across
countries. To be consistent, in the counterfactual exercises below, I always
compare the counterfactual factor prices with the factor prices in the
calibrated baseline.

\section{Counterfactuals}

The main research question in this paper is how MP liberalization can affect
distribution of income between capital and labor, especially through the two
new mechanisms I study. In this section, I use the calibrated model to conduct
two counterfactuals to evaluate the impact of MP on the labor shares. The
first counterfactual is more hypothetical but can be used to illustrate the
workings of the model. In this experiment, I study the impact of a unilateral
MP cost decline in each of the countries in my sample. Second, I use my model
to mimic the MP liberalization from 1996-2011 to a later period 2006-2011 and
examine its implications on the labor shares. In both counterfactuals, MP
liberalization reduces the labor shares in most countries especially in
capital scarce countries. The MP liberalization in the past decade helps to
explain not only the average decline in the labor shares but also the
variation in the decline across countries. I use the model to further
decompose the relative importance for the two mechanisms I introduced in my
model. I found both mechanisms are important to account for the decline of
labor shares after MP liberalization, with the technology origin effect
relatively more important for understanding the changes in less-developed countries.

\subsection{Unilateral MP liberalization}

In the first counterfactual, I consider a scenario where each country reduces
their inward MP costs by 10 log points but keeping all other parameters
unchanged. This counterfactual mimics a country conducting unilateral MP
liberalization without bilateral agreements that provide reciprocal
incentives. In particular, for each country $l$, I reduce the bilateral MP
costs $\gamma_{il}$ by 10 log points except for self-investment cost
$\gamma_{ll}$ which is kept at its original value, one. The bilateral MP costs
among other countries are also kept the same and I solve the general
equilibrium of the global economy for each country $l$'s reform.%

%TCIMACRO{\TeXButton{fig:cf_unilateral_mp_lib}{\begin{figure}[ptbh]\caption
%{Unilateral MP liberalization}\label{fig:cf_unilateral_mp_lib}
%\centering\includegraphics[scale=0.9]{{cf_unilateral_mp_lib.pdf}}\end{figure}%
%}}%
%BeginExpansion
\begin{figure}[ptbh]\caption{Unilateral MP liberalization}\label
{fig:cf_unilateral_mp_lib}
\centering\includegraphics[scale=0.9]{{cf_unilateral_mp_lib.pdf}}\end{figure}%
%EndExpansion


Figure \ref{fig:cf_unilateral_mp_lib} illustrates the results of the first
counterfactual experiment. Each dot in both panels represents a different
general equilibrium that results from the country's unilateral MP
liberalization and statistics are computed for that country only, leaving the
other 36 countries in the background. Panel a plots the change in total inward
MP shares $\sum_{i\neq l}X_{il\cdot}/Y_{l}$, which is positive for all
countries when they liberalize MP unilaterally. The effect of the same
10-log-point decline in $\gamma_{il}$ on total inward MP shares is non-linear
in the sense that countries with higher inward MP shares initially (thus lower
$\gamma_{il}$) see an even larger increase in inward multinational activities
with the same decline in $\gamma_{il}$.

Panel b of Figure \ref{fig:cf_unilateral_mp_lib} illustrates how the two
mechanisms in my model predict the changes in labor shares after the
unilateral MP liberalization. All countries see a decline in their labor
shares after the liberalization but the effect is heterogeneous across
countries. The key variable to explain the heterogeneity is the change in
inward MP shares induced by the reduction in $\gamma_{il}$. On average, the
larger the increase in inward MP shares, the larger the decline of the labor
shares. The correlation is stronger for the half of the sample countries that
are less capital abundant (blue dots) than those capital abundant ones (green
dots). For example, Germany has the largest increase in the total inward MP
share after their unilateral MP liberalization but only sees a decline in the
labor share by less than one percentage point, while capital scarce countries
such as Hungary and Slovakia see much larger declines in their labor shares.
Inituitively, the capital-biased productivity mechanism works similarly in
capital scarce and capital abundant countries. It works essentially through
selection: more MP intensifies competition in the country and shifts the
distribution of firms towards large and capital-intensive ones. In contrast,
the technology origin effect only matters when the host country has different
capital abundance than the home countries. Since the majority of FDI is from
capital abundant countries, the capital scarce countries will be affected the
most by the technology origin effect of MP, which causes a stronger
relationship between the increase of multinational activities and the decline
of labor shares.

To summarize, a unilateral reduction in countries' inward MP costs increases
multinational activities in that country. The CBP mechanism contributes to a
labor share decline to all countries. The technology origin effect matters
relatively more for capital scarce countries because they receive more MP from
capital abundant countries after the liberalization, whose technologies are
inherently more capital intensive than the ones used by the domestic firms in
the host country.

\subsection{MP liberalization up to 2011}

The first counterfactual is useful to think about a hypothetical MP
liberalization but it is hard to judge how reasonable a 10-log-point decrease
in $\gamma_{il}$ is. In the second counterfatual, I use MP data in a later
period to calibrate the decline in the inward MP costs for each country. I
collect data on multinational activities from OECD and Eurostat up to 2011,
and compute the average total inward MP shares for each country between 2006
and 2011, after 10 years of the baseline period. For each host country $l$, I
then calibrate the common change in the bilateral MP costs $\tilde{\gamma}%
_{l}=\gamma_{il}^{06-11}/\gamma_{il}^{96-01}$ for all $i\neq l$. The data for
the new period only covers 19 out of the 37 countries in the original sample.
For countries that do not have information on total inward MP shares, I simply
assume they have the average decline $\tilde{\gamma}_{l}$ of the other
countries.\footnote{I also experiment with the assumption that the MP costs
did not change at all for these 18 countries without data. The results are
available upon request and the results for the 19 countries with data barely
change.} When I examine the results from the counterfactual, I only focus on
the 19 countries with data in the later period.

Among the 19 countries with both 1996-2001 and 2006-2011 data, the average
growth in inward MP shares is 10.2 percentage points. East European countries
such as Romania, Bulgaria and Slovakia see large growth in inward MP shares,
which can be as large as 30 percentage points. Correspondingly, the calibrated
decline in MP costs are the largest for these countries ($\tilde{\gamma}_{l}$
smaller than 1).\footnote{The average $\tilde{\gamma}_{l}$ indicates a
8.2-log-point decline, which suggests the first counterfactual (10-log-point
decline) is quantitatively reasonable.} The detailed changes in MP shares and
calibrated decline in MP costs can be found in Table \ref{tab:cf_baseline}.
The influx of foreign FDI leads to a decline in labor shares in the majority
of the countries. The average decline predicted by the model is 1.4 percentage
points while it is 1.8 percentage points in the data.

The MP liberalization over this period not only helps to explain the average
decline in the labor shares, but also helps to explain the variation of the
decline across countries. Figure \ref{fig:cf_labor_share_baseline} correlates
the change in the labor shares and the change in the total inward MP shares
both in the data (panel a) and in the model (panel b). The model suggests a
tight relationship between the change in the labor shares and the change in
the total inward MP shares. In the data, there is much more variation in the
change of the labor shares but there is also a significant negative
correlation between the change in the labor shares and the change in the
inward MP shares. The predicted regression coefficient from the model is also
very close to that from the data. The predicted change in the labor shares is
significantly correlated with the change in the labor shares in the data, with
a correlation coefficient 0.34. This suggests the model is useful to account
for different paces of labor share decline across countries.%

%TCIMACRO{\TeXButton{fig:cf_labor_share_baseline}{\begin{figure}[ptbh]\caption
%{Counterfactual of baseline model}\label{fig:cf_labor_share_baseline}
%\centering\includegraphics[scale=0.9]{{cf_labor_share_baseline.pdf}%
%}\end{figure}}}%
%BeginExpansion
\begin{figure}[ptbh]\caption{Counterfactual of baseline model}\label
{fig:cf_labor_share_baseline}
\centering\includegraphics[scale=0.9]{{cf_labor_share_baseline.pdf}%
}\end{figure}%
%EndExpansion


With the calibrated model at hand, I next consider three alternative setups to
further illustrate the workings of the model. First, I shut down the
technology origin effect and see how the capital biased productivity mechanism
only can help explain the labor share decline in the data. Second, I use an
alternative model where the technology origin effects are exogenous to
decompose the technology transfer and technology change in the baseline
counterfactual. Finally, I consider a version of the model in which capital is
fully mobile across countries to check the robustness of the predictions.

\subsubsection{Shut down technology origin effect}

With both mechanisms incorporated in the quantitative model, I can shut down
one and study the relative importance of each mechanism. To shut down the
technology origin effect (TOE), I re-calibrate the model assuming there is no
scope to choose different technologies, i.e., $\eta\rightarrow-\infty$ thus
$\left(  a,b\right)  =\left(  1,1\right)  $ for firms from any country. The
new calibration targets all the moments except for the technology origin
effect. I then calculate the same counterfactual experiment, reducing the
foreign MP costs such that the MP shares matches those in 2006-2011.

In Figure \ref{fig:baseline_vs_cbp_only}, I plot the percentage point changes
in labor shares in the baseline model (CBP + TOE) against the model with only
CBP (capital-biased productivities). For countries that are not hugely
impacted by the inward MP, the CBP mechanism explains the majority of the
decline in labor shares. However, for less developed countries that saw large
increases in inward MP (Romania, Bulgaria and Slovakia), the TOE mechanism
reduces the labor shares beyond the CBP mechanism. Panel B Figure
\ref{fig:cf_cbp_only} shows the counterfactual change in the labor shares with
only the CBP mechanism. The average decline in the labor shares is 0.85
percentage points, and the correlation between the changes in MP and changes
in labor shares is much weaker. Thus, both mechanisms are important to
understand the impact of MP on the distribution of income between capital and
labor, with the TOE mechanism especially important for less developed countries.%

%TCIMACRO{\TeXButton{fig:baseline_vs_cbp_only}{\begin{figure}[ptbh]\caption
%{Decompose CBP and TOE}\label{fig:baseline_vs_cbp_only}
%\centering\includegraphics[scale=0.9]{{baseline_vs_cbp_only.pdf}}\end{figure}%
%}}%
%BeginExpansion
\begin{figure}[ptbh]\caption{Decompose CBP and TOE}\label
{fig:baseline_vs_cbp_only}
\centering\includegraphics[scale=0.9]{{baseline_vs_cbp_only.pdf}}\end{figure}%
%EndExpansion


\subsubsection{Technology transfer v.s. technology change\newline}

To generate technology differences across countries, I assume firms
endogenously choose technologies to maximize their expected global profits.
Theoretically, after a MP liberalization, firms not only transfer their
technology to foreign countries more via MP, but also reoptimize their
technology $\left(  a,b\right)  $. It is straightforward that the technology
transfer will impact the labor shares in the host countries, but the
endogenous change in $\left(  a,b\right)  $ will affect the labor shares both
at home and in the foreign countries. In the extreme case where MP becomes
frictionless, the factor biases of technologies developed in different
countries converge, and such technology change must induce huge responses in
factor prices. The second alternative model is calibrated to illustrate the
quantitative importance of the two channels given the frictions in the recent period.

In the alternative setup, I assume there is no endogenous choice of $\left(
a,b\right)  $ but the capital share parameter $\lambda_{k}$ is now home
country specific and increases with the home country's capital abundance in an
exogenous fashion. It still generates the "technology origin effect" but firms
no longer change their technology after MP liberalization. In particular, I
allow the capital share parameter $\lambda_{k}$ to depend on the home country
$i$ in an exogenous way%
\[
\log\left(  \frac{\lambda_{ki}}{1-\lambda_{ki}}\right)  =\alpha_{0}^{\lambda
}+\alpha_{1}^{\lambda}\log\left(  K_{i}/L_{i}\right)  \text{.}%
\]
Thus the alternative model does not have the elasticity of extensive
substitution $\eta$ but has a new parameter $\alpha_{1}^{\lambda}$ (note
$\alpha_{0}^{\lambda}$ determines the average labor share in the baseline
model and in the alternative model as well). I re-calibrate this model
targeting exactly the same moments as in the baseline. $\alpha_{1}^{\lambda}$
is calibrated to be 0.264, which is consistent with the technology origin
effect I find in the data.

After I calibrate the model, I conduct the same counterfactual experiment to
match the increase in MP shares in the 10-year period. In Figure
\ref{fig:baseline_vs_exogenous_toe}, I plot the changes in labor shares in the
alternative calibration against the changes in the baseline calibration
(endogenous choice of $\left(  a,b\right)  $). The two counterfactuals on the
labor shares align very well with each other. This means that the impact of MP
on the labor share through the TOE mechanism relies mostly on the transfer of
technologies with different factor bias and that the technology change plays
almost no role. The intuition behind this is that even though the inflow of MP
activities into the host countries is large, it is still small comparing to
the size of the home countries (especially the largest home countries such as
US and Germany). Thus the potential of foreign production is still too small
to alter firms' choices of $\left(  a_{i},b_{i}\right)  $ quantitatively. The
response of endogenous technology choice to MP liberalization does not add
much to the change in the labor shares, at least with the current level of MP
costs, but could play an important role when MP costs further decline in the future.%

%TCIMACRO{\TeXButton{fig:baseline_vs_exogenous_toe}{\begin{figure}%
%[ptbh]\caption{Exogenous vs Endogenous TOE}\label
%{fig:baseline_vs_exogenous_toe}
%\centering\includegraphics[scale=0.9]{{baseline_vs_exogenous_toe.pdf}%
%}\end{figure}}}%
%BeginExpansion
\begin{figure}[ptbh]\caption{Exogenous vs Endogenous TOE}\label
{fig:baseline_vs_exogenous_toe}
\centering\includegraphics[scale=0.9]{{baseline_vs_exogenous_toe.pdf}%
}\end{figure}%
%EndExpansion


\subsubsection{Capital mobility}

In the baseline calibration, I assume that there is no mobility in capital
across countries. The debate about the mobility of capital across countries
has not been settled yet. The international macro literature has documented
the lack of capital flows towards developing countries (\cite{lucas_why_1990})
while \cite{caselli_marginal_2007} argue that the capital rental rates are
similar across countries after proper accounting so there is essentially no
incentive to move capital across borders. Instead of modeling the friction of
capital mobility across countries, here I simply consider an extreme version
of the model where capital markets are perfectly integrated. This leads to an
equalization of rental rates $r_{i}$ across countries but still leaves room
for different relative factor prices across countries since the model features
monopolistic competition and increasing return to scales and typically in such
a model, countries with different sizes will have different wages. The
differences in relative factor prices give firms incentives to choose
different technologies $\left(  a_{i},b_{i}\right)  $ and the technology
origin effect still operates besides the capital-biased productivity mechanism.

In practice, I recalibrate the model with the same parameters and moments but
allow capital to be perfectly mobile across countries. I can conduct the same
counterfactual to match the new inward MP shares in 2006-2011. Table
\ref{tab:cf_mobileK_selected} compares the change in labor shares in the
baseline model and in this alternative setup for countries with largest
declines of labor shares. The change of labor shares is smaller due to the
cross-country flows of capital. However, one still sees a decline of labor
share for most of the countries, with a reduction in the magnitude between 1/2
to 2/3. Thus, changing the mobility of capital in the model only changes the
predictions on the labor shares quantitatively but not qualitatively. The
decline in most countries are still sizable comparing to those in the data.
The new mechanisms introduced in my model still help to explain the average
decline of labor shares and the variation across countries in the 10-year
period, as long as factor prices are not equalized across countries.%

%TCIMACRO{\TeXButton{cf_mobileK_selected}{\input
%{tables/cf_mobileK_selected.tex}}}%
%BeginExpansion
\input{tables/cf_mobileK_selected.tex}%
%EndExpansion


\section{Conclusion}

Multinational firms differ in many dimensions from domestic firms. This paper
focuses on the factor bias of multinational production. I document two
empirical regularities about firm's capital intensity (1) firms from capital
abundant countries are more capital intensive and (2) larger firms are more
capital intensive. The empirical results suggest multinational production may
impact the labor shares in the host countries. Based on the empirical
evidence, I build a quantifiable general equilibrium trade and MP model
incorporating two novel mechanisms to match both empirical regularities. A
reduction in MP costs leads to more MP activities thus changes the relative
demand for capital and labor in the host countries through both mechanisms. I
find the increase of multinational activities leads to sizable decline in the
labor shares in the host countries through both mechanisms. The increase in MP
activities not only explains the average change in the labor shares, but also
the variation across countries.

The paper provides a first step to understanding the heterogeneity in input
usage across firms, especially firms from different countries. The framework
can be applied to choices of other factor usage, such as skilled labor v.s.
unskilled labor. I focus on capital and labor here largely because of data
constraint. More detailed information about multinational firm operation is
needed to study other factors. Moreover, other aspects of firms' technologies
such as pollution are also worth studying given the debate about whether
multinational firms increase or reduce pollution in the host countries.
(\cite{wang_foreign_2014})

The quantitative model highlights the technology transfer within the firm but
abstracts from other vehicles of technology transfer and technology diffusion.
The calibrated multinational production costs may overestimate the barriers to
transfer technologies across countries. Incorporating different channels of
technology transfer can be quantitatively important, and I leave it for future research.

\bibliographystyle{econometrica}
\bibliography{myLib160429}


\appendix{}

\section{Theory Appendix}

\subsection{Proofs for Section \ref{sec:model}}

Under the assumption that $\xi=0$, I obtain the following expressions for
operating profits conditional on $\phi$ in market $n$
\[
\pi_{i\cdot n}\left(  \phi\right)  =\frac{\tilde{\sigma}^{1-\sigma}X_{n}%
}{\sigma P_{n}^{1-\sigma}}\Gamma\left(  \frac{\theta-\sigma+1}{\theta}\right)
\phi^{\sigma-1}\Psi_{in}^{\frac{\sigma-1}{\theta}},
\]
where $\Psi_{in}$ does not depend on $\phi$ but depends on $\left(
a,b\right)  $ (omitted in the notation for simplicity)%
\[
\Psi_{in}\equiv\sum_{l}T_{il}\left(  \gamma_{il}C_{l}\left(  a,b\right)
\tau_{ln}\right)  ^{-\theta}%
\]
The zero profit cutoff is%
\begin{align*}
\pi_{i\cdot n}\left(  \phi_{in}^{\ast}\right)   &  =P_{n}F_{n}\\
&  \Rightarrow\phi_{in}^{\ast}=\Gamma^{-1/\left(  \sigma-1\right)  }\Psi
_{in}^{-1/\theta}\frac{\tilde{\sigma}}{P_{n}}\left(  \frac{\sigma P_{n}F_{n}%
}{X_{n}}\right)  ^{1/\left(  \sigma-1\right)  }%
\end{align*}
and the expected global profit%
\begin{align*}
E_{\phi}\left[  \pi_{i}\left(  \phi,a,b\right)  \right]   &  =\sum_{n}%
\int_{\phi_{in}^{\ast}}\left(  \pi_{i\cdot n}\left(  \phi\right)  -P_{n}%
F_{n}\right)  dF_{i}\left(  \phi\right) \\
&  =\frac{\left(  \sigma-1\right)  \tilde{\sigma}^{-k}\sigma^{-\frac{k}%
{\sigma-1}}}{k-\sigma+1}\Gamma^{k/\left(  \sigma-1\right)  }\phi_{i,\min}%
^{k}\sum_{n}X_{n}^{\frac{k}{\sigma-1}}P_{n}^{k}\Psi_{in}\left(  a,b\right)
^{\frac{k}{\theta}}\left(  P_{n}F_{n}\right)  ^{\frac{\sigma-k-1}{\sigma-1}}%
\end{align*}
Denote%
\[
D_{n}\equiv\frac{\left(  \sigma-1\right)  \tilde{\sigma}^{-k}\sigma^{-\frac
{k}{\sigma-1}}}{k-\sigma+1}\Gamma^{k/\left(  \sigma-1\right)  }X_{n}^{\frac
{k}{\sigma-1}}P_{n}^{k}\left(  P_{n}F_{n}\right)  ^{\frac{\sigma-k-1}%
{\sigma-1}}%
\]


\begin{lemma}
The optimal technology choice $\left(  a,b\right)  $ must be on the boundary
of the technology menu, i.e.,%
\[
\theta\left(  a,b\right)  =1.
\]
For an optimal interior technology $\left(  a_{i}^{\ast},b_{i}^{\ast}\right)
$, the following condition must hold%
\[
\frac{\partial\pi_{i}\left(  a_{i}^{\ast},b_{i}^{\ast}\right)  /\partial
a}{\partial\pi_{i}\left(  a_{i}^{\ast},b_{i}^{\ast}\right)  /\partial b}%
=\frac{\partial\theta\left(  a_{i}^{\ast},b_{i}^{\ast}\right)  /\partial
a}{\partial\theta\left(  a_{i}^{\ast},b_{i}^{\ast}\right)  /\partial
b},i=N,S.
\]

\end{lemma}

\begin{proof}
This is a direct result of the properties of the production function, the
profit function and the technology menu. Recall that the production function
is strictly increasing in $a$ and $b$%
\[
q=zF\left(  aK,bL\right)  .
\]
Thus, the marginal cost strictly decreases with $a$ and $b$. The expected
global profit is $\pi_{i}=\sum_{n}D_{n}\Psi_{in}$ where $\Psi_{in}%
\equiv\left(  \sum_{m}\left(  T_{im}\zeta_{imn}^{-k}\right)  ^{1/\left(
1-\rho\right)  }\right)  ^{1-\rho}$ and $\zeta_{iln}\equiv\gamma_{il}%
C_{l}\left(  a,b\right)  \tau_{ln}$. Clearly, $\zeta_{iln}$ strictly decreases
in $a$ and $b$, thus $\Psi_{in}$ and $\pi_{i}$ strictly increase in $a$ and
$b$.

On the other hand, the function that determines the boundary of the technology
menu, $\theta\left(  a,b\right)  $, strictly increases in $a$ and $b$.
Consider $\left(  a,b\right)  $ such that $\theta\left(  a,b\right)
=\lambda<1$, then there exists $\left(  a^{\prime},b^{\prime}\right)  >$
$\left(  a,b\right)  $ such that $\theta\left(  a^{\prime},b^{\prime}\right)
<1$. It allows the firm to have a lower cost of production in any country,
which increases the profit. Thus, the optimal technology choice $\left(
a,b\right)  $ must occur on the boundary.

The first order conditions imply the iso-profit curve is tangent to the
technology boundary at any optimal technology $\left(  a_{i}^{\ast}%
,b_{i}^{\ast}\right)  $. Since both $\theta$ and $\pi_{i}$ are continuously
differentiable and strictly increasing in $\left(  a,b\right)  $, I invert
$\theta\left(  a,b\right)  =1$ and $\pi_{i}\left(  a,b\right)  =\pi_{i}\left(
a_{i}^{\ast},b_{i}^{\ast}\right)  $\footnote{Suppose $\exists$ $b_{1}\neq
b_{2}$ such that $\pi_{i}\left(  a,b_{1}\right)  =\pi_{i}\left(
a,b_{2}\right)  $. By mean value theorem, $\partial\pi_{i}/\partial b=\left[
\pi_{i}\left(  a,b_{2}\right)  -\pi_{i}\left(  a,b_{1}\right)  \right]
/\left(  b_{2}-b_{1}\right)  =0$ which contradicts $\partial\pi_{i}/\partial
b>0$.} and get%
\[
b=\theta^{-1}\left(  a\right)  ,\,b=\pi_{i}^{-1}\left(  a\right)  .
\]


Consider $f_{i}\left(  a\right)  \equiv\pi_{i}^{-1}\left(  a\right)
-\theta^{-1}\left(  a\right)  $, which is a continuously differentiable
function in $a$ (because $C$ and $\theta$ are continuously differentiable). By
construction, $f_{i}\left(  a_{i}^{\ast}\right)  =b_{i}^{\ast}-b_{i}^{\ast}%
=0$. In addition, for any value of $a$, we must have $f\left(  a\right)
\geq0$; otherwise, $\left(  a,\pi_{i}^{-1}\left(  a\right)  \right)  $ is in
the interior of the technology menu and we can always find a technology that
delivers a higher expected global profit than $\pi_{i}\left(  a_{i}^{\ast
},b_{i}^{\ast}\right)  $. Thus, $f^{\prime}\left(  a\right)  $ must be zero at
$a_{i}^{\ast}$. This implies $\frac{\partial}{\partial a}\pi_{i}^{-1}\left(
a_{i}^{\ast}\right)  -\frac{\partial}{\partial a}\theta^{-1}\left(
a_{i}^{\ast}\right)  =0$. Since $\theta\left(  a,\theta^{-1}\left(  a\right)
\right)  =1$, $\pi_{i}\left(  a,\pi_{i}^{-1}\left(  a\right)  \right)
=\pi_{i}\left(  a_{i}^{\ast},b_{i}^{\ast}\right)  $, I obtain%
\[
\frac{\partial\pi_{i}/\partial a}{\partial\pi_{i}/\partial b}=-\frac{\partial
}{\partial a}\pi_{i}^{-1}\left(  a_{i}^{\ast}\right)  =-\frac{\partial
}{\partial a}\theta^{-1}\left(  a_{i}^{\ast}\right)  =\frac{\partial
\theta/\partial a}{\partial\theta/\partial b}.
\]

\end{proof}

\begin{proof}
[Proof of Proposition 1]When there are $N_{N}$ symmetric Northern countries
and $N_{S}$ symmetric Southern countries, the expected gross global profit of
a Northern firm can be simplified to (denote $\tilde{k}\equiv k/\left(
1-\rho\right)  $)%
\begin{equation}
\pi_{N}\left(  a,b\right)  =D_{N}\Psi_{Nh}+\left(  N_{N}-1\right)  D_{N}%
\Psi_{NN}+N_{S}D_{S}\Psi_{NS}, \label{pi_n}%
\end{equation}
where
\begin{align*}
\Psi_{Nh}  &  \equiv\left(  C_{N}^{-\tilde{k}}+\left(  N_{N}-1\right)  \left(
\gamma C_{N}\tau\right)  ^{-\tilde{k}}+N_{S}\left(  \gamma C_{S}\tau\right)
^{-\tilde{k}}\right)  ^{1-\rho},\\
\Psi_{NN}  &  \equiv\left(  \left(  C_{N}\tau\right)  ^{-\tilde{k}}+\left(
\gamma C_{N}\right)  ^{-\tilde{k}}+\left(  N_{N}-2\right)  \left(  \gamma
C_{N}\tau\right)  ^{-\tilde{k}}+N_{S}\left(  \gamma C_{S}\tau\right)
^{-\tilde{k}}\right)  ^{1-\rho},\\
\Psi_{NS}  &  \equiv\left(  \left(  C_{N}\tau\right)  ^{-\tilde{k}}+\left(
N_{N}-1\right)  \left(  \gamma C_{N}\tau\right)  ^{-\tilde{k}}+\left(  \gamma
C_{S}\right)  ^{-\tilde{k}}+\left(  N_{S}-1\right)  \left(  \gamma C_{S}%
\tau\right)  ^{-\tilde{k}}\right)  ^{1-\rho}.
\end{align*}
The first term in equation (\ref{pi_n}) is the expected profit from the
domestic market, the second term is the total expected profit from other
Northern markets, and the third term is the total expected profit from selling
to the South. The expressions for $\Psi$'s are also intuitive. $\Psi_{Nh}$
determines the expected profit from serving the home market through (1)
selling from itself (2) selling from a subsidiary in another Northern country
and (3) selling from a subsidiary in a Southern country. The expressions for
$\Psi_{NN}$ and $\Psi_{NS}$ can be interpreted in the same way. The global
expected profit depends on technology choice through the production costs
$C_{N}$ and $C_{S}$. For a typical Southern firm $i\in S$, the expression is
similar%
\begin{equation}
\pi_{S}\left(  a,b\right)  =D_{S}\Psi_{Sh}+\left(  N_{S}-1\right)  D_{S}%
\Psi_{SS}+N_{N}D_{N}\Psi_{SN}, \label{pi_s}%
\end{equation}
where%
\begin{align*}
&  \Psi_{Sh}\equiv\left(  C_{S}^{-\tilde{k}}+\left(  N_{S}-1\right)  \left(
\gamma C_{S}\tau\right)  ^{-\tilde{k}}+N_{N}\left(  \gamma C_{N}\tau\right)
^{-\tilde{k}}\right)  ^{1-\rho},\\
&  \Psi_{SS}\equiv\left(  \left(  C_{S}\tau\right)  ^{-\tilde{k}}+\left(
\gamma C_{S}\right)  ^{-\tilde{k}}+\left(  N_{S}-2\right)  \left(  \gamma
C_{S}\tau\right)  ^{-\tilde{k}}+N_{N}\left(  \gamma C_{N}\tau\right)
^{-\tilde{k}}\right)  ^{1-\rho},\\
&  \Psi_{SN}\equiv\left(  \left(  C_{S}\tau\right)  ^{-\tilde{k}}+\left(
N_{S}-1\right)  \left(  \gamma C_{S}\tau\right)  ^{-\tilde{k}}+\left(  \gamma
C_{N}\right)  ^{-\tilde{k}}+\left(  N_{N}-1\right)  \left(  \gamma C_{N}%
\tau\right)  ^{-\tilde{k}}\right)  ^{1-\rho}.
\end{align*}


Consider the derivative of expected profit with respect to $C_{N}$ and $C_{S}$%
\begin{align*}
\partial\pi_{N}/\partial C_{N}  &  =-kD_{N}\left(  1+\left(  N_{N}-1\right)
\left(  \gamma\tau\right)  ^{-\tilde{k}}\right)  \Psi_{Nh}^{\frac{-\rho
}{1-\rho}}C_{N}^{-\tilde{k}-1}\\
&  -kD_{N}\left(  N_{N}-1\right)  \left(  \tau^{-\tilde{k}}+\gamma^{-\tilde
{k}}+\left(  N_{N}-2\right)  \left(  \gamma\tau\right)  ^{-\tilde{k}}\right)
\Psi_{NN}^{\frac{-\rho}{1-\rho}}C_{N}^{-\tilde{k}-1}\\
&  -kD_{S}N_{S}\tau^{-\tilde{k}}\left(  1+\left(  N_{N}-1\right)
\gamma^{-\tilde{k}}\right)  \Psi_{NS}^{\frac{-\rho}{1-\rho}}C_{N}^{-\tilde
{k}-1},
\end{align*}
and%
\begin{align*}
\partial\pi_{N}/\partial C_{S}  &  =-kD_{N}N_{S}\left(  \gamma\tau\right)
^{-\tilde{k}}\Psi_{Nh}^{\frac{-\rho}{1-\rho}}C_{S}^{-\tilde{k}-1}\\
&  -kD_{N}\left(  N_{N}-1\right)  N_{S}\left(  \gamma\tau\right)  ^{-\tilde
{k}}\Psi_{NN}^{\frac{-\rho}{1-\rho}}C_{S}^{-\tilde{k}-1}\\
&  -kD_{S}N_{S}\gamma^{-\tilde{k}}\left(  1+\left(  N_{S}-1\right)
\tau^{-\tilde{k}}\right)  \Psi_{NS}^{\frac{-\rho}{1-\rho}}C_{S}^{-\tilde{k}%
-1}.
\end{align*}


What matters for the choice of technology is the ratio of the two%
\[
\frac{\partial\pi_{N}/\partial C_{N}}{\partial\pi_{N}/\partial C_{S}}%
=\frac{A_{N}}{B_{N}}\left(  \frac{C_{N}}{C_{S}}\right)  ^{-\tilde{k}-1},
\]
where%
\begin{align*}
A_{N}  &  =D_{N}\left(  1+\left(  N_{N}-1\right)  \left(  \gamma\tau\right)
^{-\tilde{k}}\right)  \Psi_{Nh}^{\frac{-\rho}{1-\rho}}\\
&  +D_{N}\left(  N_{N}-1\right)  \left(  \tau^{-\tilde{k}}+\gamma^{-\tilde{k}%
}+\left(  N_{N}-2\right)  \left(  \gamma\tau\right)  ^{-\tilde{k}}\right)
\Psi_{NN}^{\frac{-\rho}{1-\rho}}\\
&  +D_{S}N_{S}\tau^{-\tilde{k}}\left(  1+\left(  N_{N}-1\right)
\gamma^{-\tilde{k}}\right)  \Psi_{NS}^{\frac{-\rho}{1-\rho}},
\end{align*}
and%
\begin{align*}
B_{N}  &  =D_{N}N_{S}\left(  \gamma\tau\right)  ^{-\tilde{k}}\Psi_{Nh}%
^{\frac{-\rho}{1-\rho}}\\
&  +D_{N}\left(  N_{N}-1\right)  N_{S}\left(  \gamma\tau\right)  ^{-\tilde{k}%
}\Psi_{NN}^{\frac{-\rho}{1-\rho}}\\
&  +D_{S}N_{S}\gamma^{-\tilde{k}}\left(  1+\left(  N_{S}-1\right)
\tau^{-\tilde{k}}\right)  \Psi_{NS}^{\frac{-\rho}{1-\rho}}.
\end{align*}


For a typical Southern firm the expressions are similar%
\[
\frac{\partial\pi_{S}/\partial C_{N}}{\partial\pi_{S}/\partial C_{S}}%
=\frac{A_{S}}{B_{S}}\left(  \frac{C_{N}}{C_{S}}\right)  ^{-\tilde{k}-1},
\]
where%
\begin{align*}
A_{S}  &  =D_{S}N_{N}\left(  \gamma\tau\right)  ^{-\tilde{k}}\Psi_{Sh}%
^{\frac{-\rho}{1-\rho}}\\
&  +D_{S}\left(  N_{S}-1\right)  N_{N}\left(  \gamma\tau\right)  ^{-\tilde{k}%
}\Psi_{SS}^{\frac{-\rho}{1-\rho}}\\
&  +D_{N}N_{N}\gamma^{-\tilde{k}}\left(  1+\left(  N_{N}-1\right)
\tau^{-\tilde{k}}\right)  \Psi_{SN}^{\frac{-\rho}{1-\rho}},
\end{align*}
and%
\begin{align*}
B_{S}  &  =D_{S}\left(  1+\left(  N_{S}-1\right)  \left(  \gamma\tau\right)
^{-\tilde{k}}\right)  \Psi_{Sh}^{\frac{-\rho}{1-\rho}}\\
&  +D_{S}\left(  N_{S}-1\right)  \left(  \tau^{-\tilde{k}}+\gamma^{-\tilde{k}%
}+\left(  N_{S}-2\right)  \left(  \gamma\tau\right)  ^{-\tilde{k}}\right)
\Psi_{SS}^{\frac{-\rho}{1-\rho}}\\
&  +D_{N}N_{N}\tau^{-\tilde{k}}\left(  1+\left(  N_{S}-1\right)
\gamma^{-\tilde{k}}\right)  \Psi_{SN}^{\frac{-\rho}{1-\rho}}.
\end{align*}


To compare $A_{N}/B_{N}$ and $A_{S}/B_{S}$, consider $A_{N}B_{S}-B_{N}A_{S}$.
Collecting terms, one can show that when $\gamma>1$,

\begin{itemize}
\item the coefficient before $D_{N}^{2}$ is positive;

\item the coefficient before $D_{S}^{2}$ is positive;

\item the coefficient before $D_{N}D_{S}$ equals%
\begin{align}
&  \lambda_{Nh,Sh}\Psi_{Nh}^{\frac{-\rho}{1-\rho}}\Psi_{Sh}^{\frac{-\rho
}{1-\rho}}+\lambda_{Nh,SS}\Psi_{Nh}^{\frac{-\rho}{1-\rho}}\Psi_{SS}%
^{\frac{-\rho}{1-\rho}}+\lambda_{NN,Sh}\Psi_{NN}^{\frac{-\rho}{1-\rho}}%
\Psi_{Sh}^{\frac{-\rho}{1-\rho}}\label{whatSign}\\
&  +\lambda_{NN,SS}\Psi_{NN}^{\frac{-\rho}{1-\rho}}\Psi_{SS}^{\frac{-\rho
}{1-\rho}}+\lambda_{NS,SN}\Psi_{NS}^{\frac{-\rho}{1-\rho}}\Psi_{SN}%
^{\frac{-\rho}{1-\rho}},\nonumber
\end{align}
where $\lambda$'s are coefficients that involve $\gamma,\tau,N_{N},N_{S}$. It
is straightforward to show that the first four coefficients are all positive,
while the last coefficient $\lambda_{NS,SN}$ has the same sign as $\gamma
-\tau$. All the expressions for these coefficients can be found in the online
appendix. When $\gamma>\tau$ or $\tau\rightarrow\infty$, one can show that
$A_{N}B_{S}-B_{N}A_{S}>0$ thus
\[
\frac{\partial\pi_{N}/\partial C_{N}}{\partial\pi_{N}/\partial C_{S}}%
>\frac{\partial\pi_{S}/\partial C_{N}}{\partial\pi_{S}/\partial C_{S}}%
\]
for any $\left(  C_{N},C_{S}\right)  $. After all the tedious algebra, I
conclude that firms care relatively more about the production cost in its home
country. This is because the foreign MP cost $\gamma$ is greater than 1, which
makes the firm sees the foreign market as less important.
\end{itemize}

Now it becomes straightforward to determine the sign of $\frac{\partial\pi
_{N}/\partial a}{\partial\pi_{N}/\partial b}-\frac{\partial\pi_{S}/\partial
a}{\partial\pi_{S}/\partial b}$. Using the chain rule,%
\[
\frac{\partial\pi_{N}}{\partial a}=\frac{\partial\pi_{N}}{\partial C_{N}}%
\frac{\partial C_{N}}{\partial a}+\frac{\partial\pi_{N}}{\partial C_{S}}%
\frac{\partial C_{S}}{\partial a},
\]
and similar expressions hold for $\partial\pi_{N}/\partial b$, $\partial
\pi_{S}/\partial a$ and $\partial\pi_{S}/\partial b$. Thus%
\[
\frac{\partial\pi_{N}/\partial a}{\partial\pi_{N}/\partial b}=\frac
{\frac{\partial\pi_{N}/\partial C_{N}}{\partial\pi_{N}/\partial C_{S}}%
\frac{\partial C_{N}}{\partial a}+\frac{\partial C_{S}}{\partial a}}%
{\frac{\partial\pi_{N}/\partial C_{N}}{\partial\pi_{N}/\partial C_{S}}%
\frac{\partial C_{N}}{\partial b}+\frac{\partial C_{S}}{\partial b}}%
\]
and%
\[
\frac{\partial\pi_{S}/\partial a}{\partial\pi_{S}/\partial b}=\frac
{\frac{\partial\pi_{S}/\partial C_{N}}{\partial\pi_{S}/\partial C_{S}}%
\frac{\partial C_{N}}{\partial a}+\frac{\partial C_{S}}{\partial a}}%
{\frac{\partial\pi_{S}/\partial C_{N}}{\partial\pi_{S}/\partial C_{S}}%
\frac{\partial C_{N}}{\partial b}+\frac{\partial C_{S}}{\partial b}}.
\]


Denote $\sigma_{N}\equiv\frac{\partial\pi_{N}/\partial C_{N}}{\partial\pi
_{N}/\partial C_{S}}$ and $\sigma_{S}\equiv\frac{\partial\pi_{S}/\partial
C_{N}}{\partial\pi_{S}/\partial C_{S}}$, the difference between $\frac
{\partial\pi_{N}/\partial a}{\partial\pi_{N}/\partial b}$ and $\frac
{\partial\pi_{S}/\partial a}{\partial\pi_{S}/\partial b}$ is%
\begin{align*}
\frac{\partial\pi_{N}/\partial a}{\partial\pi_{N}/\partial b}-\frac
{\partial\pi_{S}/\partial a}{\partial\pi_{S}/\partial b}  &  =\frac{\sigma
_{N}\frac{\partial C_{N}}{\partial a}+\frac{\partial C_{S}}{\partial a}%
}{\sigma_{N}\frac{\partial C_{N}}{\partial b}+\frac{\partial C_{S}}{\partial
b}}-\frac{\sigma_{S}\frac{\partial C_{N}}{\partial a}+\frac{\partial C_{S}%
}{\partial a}}{\sigma_{S}\frac{\partial C_{N}}{\partial b}+\frac{\partial
C_{S}}{\partial b}}\\
&  =\frac{\partial C_{N}}{\partial b}\frac{\partial C_{S}}{\partial b}%
\frac{\left(  \sigma_{N}-\sigma_{S}\right)  \left(  \frac{\partial
C_{N}/\partial a}{\partial C_{N}/\partial b}-\frac{\partial C_{S}/\partial
a}{\partial C_{S}/\partial b}\right)  }{\left(  \sigma_{N}\frac{\partial
C_{N}}{\partial b}+\frac{\partial C_{S}}{\partial b}\right)  \left(
\sigma_{S}\frac{\partial C_{N}}{\partial b}+\frac{\partial C_{S}}{\partial
b}\right)  }.
\end{align*}
Since $\sigma_{N}>\sigma_{S}$, the sign of the above expression is determined
by the second term in the numerator. Consider the cost function dual to the
production function $F\left(  aK,bL\right)  $%
\[
C\left(  r,w;a,b\right)  =\min_{K,L}rK+wL\text{ s.t. }F\left(  aK,bL\right)
\geq1.
\]


The optimization of factor usage can be transformed into one with production
function $F\left(  K,L\right)  $ and corresponding factor prices are $\left(
r/a,w/b\right)  $. Define $\tilde{C}\left(  r,w\right)  \equiv C\left(
r,w;1,1\right)  $, which is the cost function for the production $F\left(
K,L\right)  $. When $\left(  a,b\right)  \neq\left(  1,1\right)  $, and factor
prices are $\left(  r/a,w/b\right)  $, the optimal factor demand $\left(
\tilde{K},\tilde{L}\right)  $ satisfies that%
\[
\frac{F_{K}\left(  \tilde{K},\tilde{L}\right)  }{F_{L}\left(  \tilde{K}%
,\tilde{L}\right)  }=\frac{r/a}{w/b},F\left(  \tilde{K},\tilde{L}\right)  =1.
\]
When it comes to the original problem where $\left(  a,b\right)  \neq\left(
1,1\right)  $%
\[
\frac{aF_{K}\left(  aK,bL\right)  }{bF_{L}\left(  aK,bL\right)  }=\frac{r}%
{w},F\left(  aK,bL\right)  =1.
\]
Comparing the two set of conditions, there is a relationship between $\left(
K,L\right)  $ and $\left(  \tilde{K},\tilde{L}\right)  $:
\[
K=\tilde{K}/a,L=\tilde{L}/b.
\]
Thus the cost function
\begin{align*}
C\left(  r,w;a,b\right)   &  =rK+wL\\
&  =r\frac{\tilde{K}}{a}+w\frac{\tilde{L}}{b}\\
&  =\tilde{C}\left(  r/a,w/b\right)  .
\end{align*}
The derivative of $C$ with respect to technology choice%
\begin{align*}
\frac{\partial C\left(  r,w;a,b\right)  }{\partial a}  &  =\frac
{\partial\tilde{C}\left(  r/a,w/b\right)  }{\partial a}=-\tilde{C}_{r}\frac
{r}{a^{2}},\\
\frac{\partial C\left(  r,w;a,b\right)  }{\partial b}  &  =\frac
{\partial\tilde{C}\left(  r/a,w/b\right)  }{\partial b}=-\tilde{C}_{w}\frac
{w}{b^{2}}.
\end{align*}
By Shephard's Lemma%
\begin{align*}
\tilde{C}_{r}  &  =\tilde{K}\left(  r/a,w/b\right)  =aK,\\
\tilde{C}_{w}  &  =\tilde{L}\left(  r/a,w/b\right)  =bL,
\end{align*}
and%
\[
\frac{\partial C/\partial a}{\partial C/\partial b}=\frac{rK/a}{wL/b}.
\]


Now consider the impact of factor price on $\frac{\partial C/\partial
a}{\partial C/\partial b}$
\begin{align*}
\frac{\partial\ln\left(  \frac{\partial C/\partial a}{\partial C/\partial
b}\right)  }{\partial\ln\left(  r/w\right)  }  &  =1-\left(  -\frac
{\partial\ln\left(  K/L\right)  }{\partial\ln\left(  r/w\right)  }\right) \\
&  \equiv1-\varepsilon\left(  r/w\right)  ,
\end{align*}
where the second term is the elasticity between capital and labor at the
factor prices $\left(  r,w\right)  $. Suppose the elasticity is always above
one, the left hand side is always negative. Therefore,
\[
\frac{\partial C_{N}/\partial a}{\partial C_{N}/\partial b}-\frac{\partial
C_{S}/\partial a}{\partial C_{S}/\partial b}>0\text{ and }\frac{\partial
\pi_{N}/\partial a}{\partial\pi_{N}/\partial b}-\frac{\partial\pi_{S}/\partial
a}{\partial\pi_{S}/\partial b}>0.
\]
When the elasticity is always below one, vice versa.

Now I can characterize the optimal technology under the constraint
$\theta\left(  a,b\right)  =1$. First consider $\varepsilon\left(  r/w\right)
<1$ and denote $\left(  a_{N},b_{N}\right)  =\arg\max\pi_{N}$, $\left(
a_{S},b_{S}\right)  =\arg\max\pi_{S}$. Suppose $a_{N}\geq a_{S}$. Similar to
the proof of Lemma A.1, I invert the two iso-profit curves $\pi_{i}\left(
a,b\right)  =\pi_{i}\left(  a_{i},b_{i}\right)  ,i=N,S$ and get $b=\pi
_{i}^{-1}\left(  a\right)  $. Define a function%
\[
g\left(  a\right)  \equiv\pi_{N}^{-1}\left(  a\right)  -\pi_{S}^{-1}\left(
a\right)
\]
and its derivative%
\[
g^{\prime}\left(  a\right)  =-\frac{\partial\pi_{N}/\partial a}{\partial
\pi_{N}/\partial b}+\frac{\partial\pi_{S}/\partial a}{\partial\pi_{S}/\partial
b}>0
\]
for any $\left(  a,b\right)  $.

On the other hand, $g\left(  a_{S}\right)  =\pi_{N}^{-1}\left(  a_{S}\right)
-b_{S}=\pi_{N}^{-1}\left(  a_{S}\right)  -\theta^{-1}\left(  a_{S}\right)
\geq0$, and $g\left(  a_{N}\right)  =b_{N}-$ $\pi_{S}^{-1}\left(
a_{N}\right)  \leq0$. $g\left(  a\right)  $ is strictly increasing implies
$a_{S}\geq a_{N}$. Suppose $a_{S}=a_{N}$, then both the Northern and Southern
firms choose exactly the same technology $\left(  \tilde{a},\tilde{b}\right)
$. By Lemma A.1,%
\[
\frac{\partial\pi_{N}/\partial a}{\partial\pi_{N}/\partial b}=\frac
{\partial\theta/\partial a}{\partial\theta/\partial b}=\frac{\partial\pi
_{S}/\partial a}{\partial\pi_{S}/\partial b},
\]
which contradicts $\frac{\partial\pi_{N}/\partial a}{\partial\pi_{N}/\partial
b}<\frac{\partial\pi_{S}/\partial a}{\partial\pi_{S}/\partial b}$. Thus,
$a_{S}$ must be greater than $a_{N}$. Since the optimal technology occur along
$\theta\left(  a,b\right)  =1$, we immediately get $b_{S}<b_{N}$. To compare
capital intensity of production in the same location $l$ by firms' origin, I
apply the transformation of factor demand from $\left(  \tilde{K},\tilde
{L}\right)  $ to $\left(  K,L\right)  $ as above and consider the derivative
of $K/L$ with respect to $a/b$%
\begin{align*}
\frac{\partial\ln\frac{K\left(  r_{l},w_{l};a,b\right)  }{L\left(  r_{l}%
,w_{l};a,b\right)  }}{\partial\ln\left(  a/b\right)  }  &  =\frac{\partial
\ln\frac{\tilde{K}\left(  r_{l}/a,w_{l}/b\right)  /a}{\tilde{L}\left(
r_{l}/a,w_{l}/b\right)  /b}}{\partial\ln\left(  a/b\right)  }\\
&  =-1-\frac{\partial\ln\frac{\tilde{K}\left(  r_{l}/a,w_{l}/b\right)
}{\tilde{L}\left(  r_{l}/a,w_{l}/b\right)  }}{\partial\ln\frac{r_{l}/a}%
{w_{l}/b}}\\
&  \equiv-1+\varepsilon\left(  \frac{r_{l}/a}{w_{l}/b}\right)  <0,
\end{align*}
where $\varepsilon\left(  \frac{r_{l}/a}{w_{l}/b}\right)  $ the elasticity of
capital and labor of $F\left(  K,L\right)  $ evaluated at $\left(
r_{l}/a,w_{l}/b\right)  $. Since $a_{N}/b_{N}<a_{S}/b_{S}$, we conclude%
\[
\frac{K\left(  r_{l},w_{l};a_{N},b_{N}\right)  }{L\left(  r_{l},w_{l}%
;a_{N},b_{N}\right)  }>\frac{K\left(  r_{l},w_{l};a_{S},b_{S}\right)
}{L\left(  r_{l},w_{l};a_{S},b_{S}\right)  }.
\]


For the case $\varepsilon\left(  \frac{r_{l}/a}{w_{l}/b}\right)  >1$, we know
$\frac{\partial\pi_{N}/\partial a}{\partial\pi_{N}/\partial b}$ is larger than
$\frac{\partial\pi_{S}/\partial a}{\partial\pi_{S}/\partial b}$, thus
$g\left(  a\right)  $ strictly decreases with $a$ and $a_{S}\leq a_{N}$. Using
the same argument, I exclude the case $a_{S}=a_{N}$. This again implies
$a_{S}<a_{N}$ and $b_{S}>b_{N}$. The Northern firms are more capital intensive
since%
\[
\frac{\partial\ln\frac{K\left(  r_{l},w_{l};a,b\right)  }{L\left(  r_{l}%
,w_{l};a,b\right)  }}{\partial\ln\left(  a/b\right)  }=-1+\varepsilon
>0\text{.}%
\]


The last part of the proposition is to show that under the optimal choice of
technology, the cost of production is relatively small in the domestic market,
compared to a firm originated in the opposite region. I first prove that%
\begin{equation}
\frac{C\left(  r_{N},w_{N};a_{N},b_{N}\right)  }{C\left(  r_{N},w_{N}%
;a_{S},b_{S}\right)  }\leq\frac{C\left(  r_{S},w_{S};a_{N},b_{N}\right)
}{C\left(  r_{S},w_{S};a_{S},b_{S}\right)  }. \label{relativeCosts}%
\end{equation}


I again use the transformation $\tilde{C}$ developed above:%
\[
\frac{C\left(  r_{N},w_{N};a_{N},b_{N}\right)  }{C\left(  r_{N},w_{N}%
;a_{S},b_{S}\right)  }=\frac{\tilde{C}\left(  r_{N}/a_{N},w_{N}/b_{N}\right)
}{\tilde{C}\left(  r_{N}/a_{S},w_{N}/b_{S}\right)  }=\frac{\tilde{C}\left(
\frac{r_{N}/a_{N}}{w_{N}/b_{N}},1\right)  }{\tilde{C}\left(  \frac{r_{N}%
/a_{S}}{w_{N}/b_{S}},1\right)  }.
\]


Denote $r_{1}\equiv\frac{r_{N}/a_{N}}{w_{N}/b_{N}}$, $r_{2}\equiv\frac
{r_{N}/a_{S}}{w_{N}/b_{S}}$, $r_{3}\equiv\frac{r_{S}/a_{N}}{w_{S}/b_{N}}$,
$r_{4}\equiv\frac{r_{S}/a_{S}}{w_{S}/b_{S}}$. The inequality (\ref{prop1.2})
is equivalent to
\[
\ln\tilde{C}\left(  \ln\left(  r_{1}\right)  ,1\right)  +\ln\tilde{C}\left(
\ln\left(  r_{4}\right)  ,1\right)  \leq\ln\tilde{C}\left(  \ln\left(
r_{2}\right)  ,1\right)  +\ln\tilde{C}\left(  \ln\left(  r_{3}\right)
,1\right)  .
\]
Note that $\ln r_{1}+\ln r_{4}=\ln r_{2}+\ln r_{3}$. Suppose $\varepsilon<1$,
the technology choices satisfy $a_{N}/b_{N}\leq a_{S}/b_{S}$ thus $\ln
r_{3}>\ln r_{1},\ln r_{4}>\ln r_{2}$. To establish the above inequality, one
just needs to show $\ln\tilde{C}\left(  r,1\right)  $ is convex in $\ln r$.%
\begin{align*}
\frac{\partial\ln\tilde{C}\left(  r,1\right)  }{\partial\ln r}  &
=\frac{r\tilde{K}\left(  r,1\right)  }{\tilde{C}\left(  r,1\right)  }\\
&  =1-\frac{\tilde{L}\left(  r,1\right)  }{r\tilde{K}\left(  r,1\right)
+\tilde{L}\left(  r,1\right)  }\\
&  =1-\frac{1}{r\tilde{K}/\tilde{L}+1}%
\end{align*}
and%
\begin{align*}
\frac{\partial^{2}\ln\tilde{C}\left(  r,1\right)  }{\partial\left(  \ln
r\right)  ^{2}}  &  =r\frac{\tilde{K}/\tilde{L}+r\frac{\partial\left(
\tilde{K}/\tilde{L}\right)  }{\partial r}}{\left(  r\tilde{K}/\tilde
{L}+1\right)  ^{2}}\\
&  =r\frac{1+\frac{\partial\ln\left(  \tilde{K}/\tilde{L}\right)  }%
{\partial\ln r}}{\left(  r\tilde{K}/\tilde{L}+1\right)  ^{2}}>0
\end{align*}
Thus I prove the inequality (\ref{relativeCosts}). When $\varepsilon>1$, we
have $a_{N}/b_{N}\geq a_{S}/b_{S}$ thus $\ln r_{4}>\ln r_{2},\ln r_{3}>\ln
r_{1}$. Correspondingly $\ln\tilde{C}\left(  r,1\right)  $ is concave in $\ln
r$. The inequality holds as well.

For simplicity, I denote $C_{il}=C\left(  r_{l},w_{l};a_{i},b_{i}\right)  $.
Suppose $C_{NN}>C_{SN}$, then $C_{SS}<C_{NS}$ for inequality
(\ref{relativeCosts}) to hold. This implies that the technology of the
Northern firms is strictly worse than that of the Southern firms $\left(
C_{NN},C_{NS}\right)  >\left(  C_{SN},C_{SS}\right)  $. The Northern firms can
adopt the Southern technology to improve their profits. Thus $\left(
C_{NN},C_{NS}\right)  $ is not optimal and I obtain a contradiction. Thus,
Northern firms must have cost advantage in the North and so do Southern firms
in the South.
\end{proof}

\begin{proof}
[Proof of Corollary 1]Under the assumption of CES production function, one can
show that the optimal technology must occur in the interior of $\theta\left(
a,b\right)  =1$. This is equivalent to show that the maximum of profit does
not occur when $\alpha\equiv a/b$ goes to $+\infty$ or $0$. Rewrite the
technology choice in terms of $\alpha$%
\begin{align*}
b  &  =\theta\left(  \alpha,1\right)  ^{-1},\\
a  &  =\alpha\theta\left(  \alpha,1\right)  ^{-1}.
\end{align*}
Now the marginal cost can be written as%
\[
C=\left(  \left(  \alpha/A\right)  ^{1-\eta}+\left(  1/B\right)  ^{1-\eta
}\right)  ^{1/\left(  1-\eta\right)  }\left(  r^{1-\varepsilon}\alpha
^{\varepsilon-1}+w^{1-\varepsilon}\right)  ^{1/\left(  1-\varepsilon\right)
}.
\]
Taking derivatives of $\log C$ with respect to $\log\alpha$%
\[
\frac{\partial\log C}{\partial\log\alpha}=\frac{\left(  \frac{A}{B}\right)
^{\eta-1}\alpha^{\varepsilon-1}\left[  \alpha^{2-\eta-\varepsilon}-\left(
\frac{A}{B}\right)  ^{1-\eta}\left(  \frac{r}{w}\right)  ^{1-\varepsilon
}\right]  }{\left[  \left(  \frac{A}{B}\right)  ^{\eta-1}\alpha^{1-\eta
}+1\right]  \left[  \left(  \frac{r}{w}\right)  ^{1-\varepsilon}%
\alpha^{\varepsilon-1}+1\right]  },
\]
the sign of which depends on $\alpha^{2-\eta-\varepsilon}-\left(  \frac{A}%
{B}\right)  ^{1-\eta}\left(  \frac{r}{w}\right)  ^{1-\varepsilon}$. Denote the
zero of this expression $\alpha^{\ast}\left(  r/w\right)  $ (an asterisk is
used since $\alpha^{\ast}\left(  r/w\right)  $ would be the optimal technology
choice in a world without MP)%
\[
\alpha^{\ast}\left(  r/w\right)  =\left(  \frac{A}{B}\right)  ^{\frac{1-\eta
}{2-\varepsilon-\eta}}\left(  \frac{r}{w}\right)  ^{\frac{1-\varepsilon
}{2-\varepsilon-\eta}}.
\]
When $\alpha>\alpha^{\ast}$, $\frac{\partial\log C}{\partial\log\alpha}>0$ and
vice versa when $\alpha<\alpha^{\ast}$. When there is a symmetric North and a
symmetric South region, it is straightforward to show that the optimal
technology must occur between $\alpha^{\ast}\left(  r_{N}/w_{N}\right)  $ and
$\alpha^{\ast}\left(  r_{S}/w_{S}\right)  $. If $\alpha>\max\left\{
\alpha^{\ast}\left(  r_{N}/w_{N}\right)  ,\alpha^{\ast}\left(  r_{S}%
/w_{S}\right)  \right\}  $, both $C_{N}$ and $C_{S}$ can be reduced by
reducing $\alpha$; if $\alpha<\min\left\{  \alpha^{\ast}\left(  r_{N}%
/w_{N}\right)  ,\alpha^{\ast}\left(  r_{S}/w_{S}\right)  \right\}  $, both
$C_{N}$ and $C_{S}$ can be reduced by increasing $\alpha$. Thus the optimal
technology must be an interior solution. Also note that the sign of
$2-\eta-\varepsilon$ is crucial for establishing this result, which is assumed
to be positive in the assumption.

Next, I establish the inequality that $C_{NN}\leq C_{SN}$ and $C_{SS}\leq
C_{NS}$. For a typical firm, I can transform its technology choice problem
into a problem of choosing marginal costs of production in the two regions
$\left(  C_{N},C_{S}\right)  $. An interior solution must occur where there is
a trade off between minimizing $C_{N}$ and $C_{S}$ thus $\frac{dC_{N}}{dC_{S}%
}\leq0$. When production function is CES, one can show that
\[
\frac{dC_{N}}{dC_{S}}=\left(  \frac{C_{N}}{C_{S}}\right)  ^{\varepsilon}%
\frac{r_{N}^{1-\varepsilon}-w_{N}^{1-\varepsilon}\left(  \frac{A}{B}\right)
^{\eta-1}\left(  \frac{a}{b}\right)  ^{2-\varepsilon-\eta}}{r_{S}%
^{1-\varepsilon}-w_{S}^{1-\varepsilon}\left(  \frac{A}{B}\right)  ^{\eta
-1}\left(  \frac{a}{b}\right)  ^{2-\varepsilon-\eta}}.
\]
The trade off $dC_{N}/dC_{S}\leq0$ only when $a/b$ is between $\left(
\frac{A}{B}\right)  ^{\frac{1-\eta}{2-\eta-\varepsilon}}\left(  \frac{r_{S}%
}{w_{S}}\right)  ^{\frac{1-\varepsilon}{2-\eta-\varepsilon}}$and $\left(
\frac{A}{B}\right)  ^{\frac{1-\eta}{2-\eta-\varepsilon}}\left(  \frac{r_{N}%
}{w_{N}}\right)  ^{\frac{1-\varepsilon}{2-\eta-\varepsilon}}$. Outside this
region, we can change $\left(  a,b\right)  $ to reduce both $C_{N}$ and
$C_{S}$ and increase the expected profit. Thus the optimum $\left(
C_{NN},C_{NS}\right)  $ for Northern firms and $\left(  C_{SN},C_{SS}\right)
$ for Southern firms can only occur on this segment. Combining with the result
that $C_{NN}/C_{NS}\leq C_{SN}/C_{SS}$, I obtain $C_{NN}\leq C_{SN}$ and
$C_{SS}\leq C_{NS}$.
\end{proof}

\begin{proof}
[Proof of Proposition 2]Suppose the opposite: capital is relatively cheap in
the South $r_{N}/w_{N}>r_{S}/w_{S}$. In the proof I will derive a
contradiction that the relative demand for capital is larger in the South
under this condition for the case $\varepsilon<1$. (the proof for
$\varepsilon>1$ is similar). \qquad When $r_{N}/w_{N}>r_{S}/w_{S}$,
Proposition \ref{prop1} implies the Northern firms must be more labor
intensive $a_{N}/b_{N}>a_{S}/b_{S}$ and $C_{NN}C_{SS}<C_{NS}C_{SN}$. Recall
the factor market clearing conditions \ref{capitalMarket} and
\ref{laborMarket}, the relative factor demand in country $l$ is%
\begin{align*}
\frac{K_{l}}{L_{l}}  &  =\left(  \frac{r_{l}}{w_{l}}\right)  ^{-1}\frac
{\tilde{\sigma}^{-1}\sum_{i,n}\psi_{iln}\lambda_{in}^{E}X_{n}\kappa_{il}%
+\frac{1}{\tilde{\sigma}k}\sum_{n}\lambda_{ln}^{E}X_{n}\kappa_{ll}%
+\frac{k-\left(  \sigma-1\right)  }{\sigma k}X_{l}\kappa_{ll}}{\tilde{\sigma
}^{-1}\sum_{i,n}\psi_{iln}\lambda_{in}^{E}X_{n}\left(  1-\kappa_{il}\right)
+\frac{1}{\tilde{\sigma}k}\sum_{n}\lambda_{ln}^{E}X_{n}\left(  1-\kappa
_{ll}\right)  +\frac{k-\left(  \sigma-1\right)  }{\sigma k}X_{l}\left(
1-\kappa_{ll}\right)  }\\
&  =\left(  \frac{r_{l}}{w_{l}}\right)  ^{-\varepsilon}\frac{\tilde{\sigma
}^{-1}\sum_{i,n}\psi_{iln}\lambda_{in}^{E}X_{n}C_{il}^{\varepsilon-1}%
a_{i}^{\varepsilon-1}+\left(  \frac{1}{\tilde{\sigma}k}\sum_{n}\lambda
_{ln}^{E}X_{n}+\frac{k-\left(  \sigma-1\right)  }{\sigma k}X_{l}\right)
C_{ll}^{\varepsilon-1}a_{l}^{\varepsilon-1}}{\tilde{\sigma}^{-1}\sum_{i,n}%
\psi_{iln}\lambda_{in}^{E}X_{n}C_{il}^{\varepsilon-1}b_{i}^{\varepsilon
-1}+\left(  \frac{1}{\tilde{\sigma}k}\sum_{n}\lambda_{ln}^{E}X_{n}%
+\frac{k-\left(  \sigma-1\right)  }{\sigma k}X_{l}\right)  C_{ll}%
^{\varepsilon-1}b_{l}^{\varepsilon-1}}%
\end{align*}
where the CES cost function has been applied.

Define $K_{N}^{e}\equiv\left(  \tilde{\sigma}k\right)  ^{-1}\sum_{i,n}%
\psi_{iNn}\lambda_{in}^{E}X_{n}C_{iN}^{\varepsilon-1}a_{i}^{\varepsilon
-1}+D_{N}^{e}C_{NN}^{\varepsilon-1}a_{N}^{\varepsilon-1}$ where%
\[
D_{N}^{e}\equiv\frac{1}{\tilde{\sigma}k^{2}}\sum_{n}\lambda_{Nn}^{E}%
X_{n}+\frac{k-\left(  \sigma-1\right)  }{\sigma k^{2}}X_{N}%
\]
represents the demand from marketing and entry. Using the demand shifter
(\ref{demandShifter}), the first part of $K_{N}^{e}$ can be rewritten as%
\[
\left(  \tilde{\sigma}k\right)  ^{-1}\sum_{i,n}\psi_{iln}\lambda_{in}^{E}%
X_{n}C_{il}^{\varepsilon-1}a_{i}^{\varepsilon-1}=\sum_{i,n}\left(  \gamma
_{il}\tau_{ln}\right)  ^{-\tilde{k}}M_{i}D_{n}\Psi_{in}^{\frac{-\rho}{1-\rho}%
}C_{il}^{-\tilde{k}+\varepsilon-1}a_{i}^{\varepsilon-1}%
\]


Consider a typical Northern country, its factor demand can be decomposed into:
(1) home firms production and sales to (1.1) home market (1.2) another
Northern market (1.3) a Souther market; (2) production of subsidiaries
headquartered in another Northern country and sales to (2.1) the host
(Northern) country (2.2) the source (Northern) country (2.3) a third Northern
country (2.4) a Southern country; (3) production of subsidiaries headquartered
in a Southern country and sales to (3.1) the source Southern country (3.2)
another Southern country (3.3) the host (Northern) country (3.4) another
Northern country; (4) production of goods used in marketing and entry.
Collecting terms in $K_{N}^{e}$%
\begin{align*}
K_{N}^{e}  &  =M_{N}D_{N}\left(  1+\left(  N_{N}-1\right)  \left(  \gamma
\tau\right)  ^{-\tilde{k}}\right)  \left(  C_{NN}^{-\tilde{k}}+\left(
N_{N}-1\right)  \left(  \gamma\tau C_{NN}\right)  ^{-\tilde{k}}+N_{S}\left(
\gamma\tau C_{NS}\right)  ^{-\tilde{k}}\right)  ^{-\rho}C_{NN}^{\varepsilon
-\tilde{k}-1}a_{N}^{\varepsilon-1}\\
&  +M_{N}D_{N}\left(  N_{N}-1\right)  \left(  \tau^{-\tilde{k}}+\gamma
^{-\tilde{k}}+\left(  N_{N}-2\right)  \left(  \gamma\tau\right)  ^{-\tilde{k}%
}\right)  \left(  \left(  \gamma C_{NN}\right)  ^{-\tilde{k}}+\left(  \tau
C_{NN}\right)  ^{-\tilde{k}}+\left(  N_{N}-2\right)  \left(  \gamma\tau
C_{NN}\right)  ^{-\tilde{k}}+N_{S}\left(  \gamma\tau C_{NS}\right)
^{-\tilde{k}}\right)  ^{-\rho}C_{NN}^{\varepsilon-\tilde{k}-1}a_{N}%
^{\varepsilon-1}\\
&  +M_{N}N_{S}D_{S}\left(  \tau^{-\tilde{k}}+\left(  N_{N}-1\right)  \left(
\gamma\tau\right)  ^{-\tilde{k}}\right)  \left(  \left(  \tau C_{NN}\right)
^{-\tilde{k}}+\left(  N_{N}-1\right)  \left(  \gamma\tau C_{NN}\right)
^{-\tilde{k}}+\left(  \gamma C_{NS}\right)  ^{-\tilde{k}}+\left(
N_{S}-1\right)  \left(  \gamma\tau C_{NS}\right)  ^{-\tilde{k}}\right)
^{-\rho}C_{NN}^{\varepsilon-\tilde{k}-1}a_{N}^{\varepsilon-1}\\
&  +N_{S}M_{S}D_{S}\left(  \gamma\tau\right)  ^{-\tilde{k}}\left(
C_{SS}^{-\tilde{k}}+\left(  N_{S}-1\right)  \left(  \gamma\tau C_{SS}\right)
^{-\tilde{k}}+N_{N}\left(  \gamma\tau C_{SN}\right)  ^{-\tilde{k}}\right)
^{-\rho}C_{SN}^{\varepsilon-\tilde{k}-1}a_{S}^{\varepsilon-1}\\
&  +N_{S}M_{S}\left(  N_{S}-1\right)  D_{S}\left(  \gamma\tau\right)
^{-\tilde{k}}\left(  \left(  \tau C_{SS}\right)  ^{-\tilde{k}}+\left(  \gamma
C_{SS}\right)  ^{-\tilde{k}}+\left(  N_{S}-2\right)  \left(  \gamma\tau
C_{SS}\right)  ^{-\tilde{k}}+N_{N}\left(  \gamma\tau C_{SN}\right)
^{-\tilde{k}}\right)  ^{-\rho}C_{SN}^{\varepsilon-\tilde{k}-1}a_{S}%
^{\varepsilon-1}\\
&  +N_{S}M_{S}D_{N}\left(  \gamma^{-\tilde{k}}+\left(  N_{N}-1\right)  \left(
\gamma\tau\right)  ^{-\tilde{k}}\right)  \left(  \left(  \tau C_{SS}\right)
^{-\tilde{k}}+\left(  N_{S}-1\right)  \left(  \gamma\tau C_{SS}\right)
^{-\tilde{k}}+\left(  \gamma C_{SN}\right)  ^{-\tilde{k}}+\left(
N_{N}-1\right)  \left(  \gamma\tau C_{SN}\right)  ^{-\tilde{k}}\right)
^{-\rho}C_{SN}^{\varepsilon-\tilde{k}-1}a_{S}^{\varepsilon-1}\\
&  +D_{N}^{e}C_{NN}^{\varepsilon-1}a_{N}^{\varepsilon-1}.
\end{align*}
The expression for $L_{N}^{e}$, $K_{S}^{e}$, $L_{S}^{e}$ can be derived in the
same way. My goal here is to derive $K_{N}^{e}/L_{N}^{e}\leq K_{S}^{e}%
/L_{S}^{e}$. If this holds, combining the fact that the first part in relative
factor demand $\left(  r_{N}/w_{N}\right)  ^{-\varepsilon}<\left(  r_{S}%
/w_{S}\right)  ^{-\varepsilon}$, I will obtain a contradiction $K_{N}%
/L_{N}<K_{S}/L_{S}$. I consider $K_{N}^{e}L_{S}^{e}-K_{S}^{e}L_{N}^{e}$ and
collect terms of $M_{N}^{2}D_{N}^{2}$, $M_{S}^{2}D_{S}^{2}$, $M_{S}^{2}%
D_{N}^{2}$, $M_{N}^{2}D_{S}^{2}$, $M_{N}^{2}D_{N}D_{S}$, $M_{S}^{2}D_{N}D_{S}%
$, $M_{N}M_{S}D_{N}^{2}$, $M_{N}M_{S}D_{S}^{2}$ and $M_{N}M_{S}D_{N}D_{S}$,
and terms involving the demand from entry and market access costs $D_{N}%
^{e}D_{S}^{e}$, $D_{N}^{e}M_{N}D_{N}$, $D_{N}^{e}M_{N}D_{S}$, $D_{N}^{e}%
M_{S}D_{N}$, $D_{N}^{e}M_{S}D_{S}$, $D_{S}^{e}M_{S}D_{S}$, $D_{S}^{e}%
M_{S}D_{N}$, $D_{S}^{e}M_{N}D_{S}$, $D_{S}^{e}M_{N}D_{N}$. It turns out that
all coefficients before all terms expect $M_{N}M_{S}D_{N}D_{S}$ are either
zero or negative using the results $a_{N}/b_{N}>a_{S}/b_{S}$ and $C_{NN}%
C_{SS}<C_{NS}C_{SN}$. The sign of the coefficient before $M_{N}M_{S}D_{N}%
D_{S}$ is the opposite to (\ref{whatSign}). Under the conditions of
Proposition \ref{prop1}, it must be negative. Thus, $K_{N}^{e}/L_{N}^{e}%
<K_{S}^{e}/L_{S}^{e}$ and a contradiction is obtained.
\end{proof}

\begin{proof}
[Proof of Proposition 3]Trade is frictionless $\tau=1$. The expected global
profit of a Northern firm can be written as%
\[
\pi_{N}=D_{N}\Psi_{Nh}+\left(  N_{N}-1\right)  D_{N}\Psi_{NN}+N_{S}D_{S}%
\Psi_{NS},
\]
where%
\[
\Psi_{Nh}=\left(  \left(  1+\left(  N_{N}-1\right)  \gamma^{-\tilde{k}%
}\right)  C_{N}^{-\tilde{k}}+N_{S}\gamma^{-\tilde{k}}C_{S}^{-\tilde{k}%
}\right)  ^{1-\rho}=\Psi_{NN}=\Psi_{NS}.
\]
Thus%
\[
\pi_{N}=\left(  N_{N}D_{N}+N_{S}D_{S}\right)  \Psi_{Nh}.
\]


Similarly, for a Southern firm,%
\[
\pi_{S}=\left(  N_{N}D_{N}+N_{S}D_{S}\right)  \Psi_{Sh},
\]
where%
\[
\Psi_{Sh}=\left(  \left(  1+\left(  N_{S}-1\right)  \gamma^{-\tilde{k}%
}\right)  C_{S}^{-\tilde{k}}+N_{N}\gamma^{-\tilde{k}}C_{N}^{-\tilde{k}%
}\right)  ^{1-\rho}.
\]


When $\gamma=1$, $\pi_{N}\left(  \delta\right)  =\pi_{S}\left(  \delta\right)
$. Then $\delta_{N}=\delta_{S}$ when the optimal technology is unique. When
$\gamma>1$, Proposition 1 implies that $\delta_{N}>\delta_{S}$.

Now consider the factor market clearing conditions which determine the
relative factor prices%
\[
\frac{K_{l}}{L_{l}}=\frac{K_{l}^{e}}{L_{l}^{e}}\left(  \frac{r_{l}}{w_{l}%
}\right)  ^{-\varepsilon},
\]
where $K_{l}^{e}$ and $L_{l}^{e}$ are defined as in the proof of Proposition 2.

Apply $\tau=1$, $\Psi_{Nh}=\Psi_{NN}=\Psi_{NS}$, and $\Psi_{Sh}=\Psi_{SS}%
=\Psi_{SN}$,%
\begin{align*}
K_{N}^{e}  &  =M_{N}\left(  N_{N}D_{N}+N_{S}D_{S}\right)  \left(  1+\left(
N_{N}-1\right)  \gamma^{-\tilde{k}}\right)  \Psi_{Nh}^{\frac{-\rho}{1-\rho}%
}C_{NN}^{\varepsilon-\tilde{k}-1}a_{N}^{\varepsilon-1}\\
&  +M_{S}\left(  N_{N}D_{N}+N_{S}D_{S}\right)  N_{S}\gamma^{-\tilde{k}}%
\Psi_{Sh}^{\frac{-\rho}{1-\rho}}C_{SN}^{\varepsilon-\tilde{k}-1}%
a_{S}^{\varepsilon-1}+D_{N}^{e}C_{NN}^{\varepsilon-1}a_{N}^{\varepsilon-1},
\end{align*}
and%
\begin{align*}
L_{N}^{e}  &  =M_{N}\left(  N_{N}D_{N}+N_{S}D_{S}\right)  \left(  1+\left(
N_{N}-1\right)  \gamma^{-\tilde{k}}\right)  \Psi_{Nh}^{\frac{-\rho}{1-\rho}%
}C_{NN}^{\varepsilon-\tilde{k}-1}b_{N}^{\varepsilon-1}\\
&  +M_{S}\left(  N_{N}D_{N}+N_{S}D_{S}\right)  N_{S}\gamma^{-\tilde{k}}%
\Psi_{Sh}^{\frac{-\rho}{1-\rho}}C_{SN}^{\varepsilon-\tilde{k}-1}%
b_{S}^{\varepsilon-1}+D_{N}^{e}C_{NN}^{\varepsilon-1}b_{N}^{\varepsilon-1}.
\end{align*}


The ratio of relative factor demand%
\begin{align*}
\frac{K_{N}^{e}}{L_{N}^{e}}  &  =\frac{M_{N}\left(  1+\left(  N_{N}-1\right)
\gamma^{-\tilde{k}}\right)  \Psi_{Nh}^{\frac{-\rho}{1-\rho}}C_{NN}%
^{\varepsilon-\tilde{k}-1}a_{N}^{\varepsilon-1}+M_{S}N_{S}\gamma^{-\tilde{k}%
}\Psi_{Sh}^{\frac{-\rho}{1-\rho}}C_{SN}^{\varepsilon-\tilde{k}-1}%
a_{S}^{\varepsilon-1}+D_{N}^{e}C_{NN}^{\varepsilon-1}a_{N}^{\varepsilon-1}%
}{M_{N}\left(  1+\left(  N_{N}-1\right)  \gamma^{-\tilde{k}}\right)  \Psi
_{Nh}^{\frac{-\rho}{1-\rho}}C_{NN}^{\varepsilon-\tilde{k}-1}b_{N}%
^{\varepsilon-1}+M_{S}N_{S}\gamma^{-\tilde{k}}\Psi_{Sh}^{\frac{-\rho}{1-\rho}%
}C_{SN}^{\varepsilon-\tilde{k}-1}b_{S}^{\varepsilon-1}+D_{N}^{e}%
C_{NN}^{\varepsilon-1}b_{N}^{\varepsilon-1}},\\
\frac{K_{S}^{e}}{L_{S}^{e}}  &  =\frac{M_{N}N_{N}\gamma^{-\tilde{k}}\Psi
_{Nh}^{\frac{-\rho}{1-\rho}}C_{NS}^{\varepsilon-\tilde{k}-1}a_{N}%
^{\varepsilon-1}+M_{S}\left(  1+\left(  N_{S}-1\right)  \gamma^{-\tilde{k}%
}\right)  \Psi_{Sh}^{\frac{-\rho}{1-\rho}}C_{SS}^{\varepsilon-\tilde{k}%
-1}a_{S}^{\varepsilon-1}+D_{S}^{e}C_{SS}^{\varepsilon-1}a_{S}^{\varepsilon-1}%
}{M_{N}N_{N}\gamma^{-\tilde{k}}\Psi_{Nh}^{\frac{-\rho}{1-\rho}}C_{NS}%
^{\varepsilon-\tilde{k}-1}b_{N}^{\varepsilon-1}+M_{S}\left(  1+\left(
N_{S}-1\right)  \gamma^{-\tilde{k}}\right)  \Psi_{Sh}^{\frac{-\rho}{1-\rho}%
}C_{SS}^{\varepsilon-\tilde{k}-1}b_{S}^{\varepsilon-1}+D_{S}^{e}%
C_{SS}^{\varepsilon-1}b_{S}^{\varepsilon-1}}.
\end{align*}


When $\gamma\rightarrow1$%
\[
\frac{K_{N}^{e}}{L_{N}^{e}}=\left(  \frac{a}{b}\right)  ^{\varepsilon-1}%
=\frac{K_{S}^{e}}{L_{S}^{e}}.
\]


To show relative factor prices converge when $\gamma>1$, I need to show that
\[
\frac{K_{N}^{e}}{L_{N}^{e}}>\frac{K_{S}^{e}}{L_{S}^{e}}.
\]


Now consider $K_{N}^{e}L_{S}^{e}-K_{S}^{e}L_{N}^{e}$ and one can show that it
can be simplified to\footnote{Detailed derivation can be found in the online
appendix.}%
\[
\left(  a_{N}^{\varepsilon-1}b_{S}^{\varepsilon-1}-a_{S}^{\varepsilon-1}%
b_{N}^{\varepsilon-1}\right)  \Xi,
\]
where $a_{N}^{\varepsilon-1}b_{S}^{\varepsilon-1}-a_{S}^{\varepsilon-1}%
b_{N}^{\varepsilon-1}$ is positive by Proposition 1 and
\begin{align*}
\Xi &  \equiv M_{N}\left(  1+\left(  N_{N}-1\right)  \gamma^{-\tilde{k}%
}\right)  \Psi_{Nh}^{\frac{-\rho}{1-\rho}}C_{NN}^{\varepsilon-\tilde{k}%
-1}\times M_{S}\left(  1+\left(  N_{S}-1\right)  \gamma^{-\tilde{k}}\right)
\Psi_{Sh}^{\frac{-\rho}{1-\rho}}C_{SS}^{\varepsilon-\tilde{k}-1}\\
&  +M_{S}N_{S}\gamma^{-\tilde{k}}\Psi_{Sh}^{\frac{-\rho}{1-\rho}}%
C_{SN}^{\varepsilon-\tilde{k}-1}\times M_{N}N_{N}\gamma^{-\tilde{k}}\Psi
_{Nh}^{\frac{-\rho}{1-\rho}}C_{NS}^{\varepsilon-\tilde{k}-1}\\
&  +D_{N}^{e}C_{NN}^{\varepsilon-1}D_{S}^{e}C_{SS}^{\varepsilon-1}\\
&  +D_{N}^{e}C_{NN}^{\varepsilon-1}M_{S}\left(  1+\left(  N_{S}-1\right)
\gamma^{-\tilde{k}}\right)  \Psi_{Sh}^{\frac{-\rho}{1-\rho}}C_{SS}%
^{\varepsilon-\tilde{k}-1}\\
&  +D_{S}^{e}C_{SS}^{\varepsilon-1}M_{N}\left(  1+\left(  N_{N}-1\right)
\gamma^{-\tilde{k}}\right)  \Psi_{Nh}^{\frac{-\rho}{1-\rho}}C_{NN}%
^{\varepsilon-\tilde{k}-1}.
\end{align*}
Using the restriction $\varepsilon-\tilde{k}-1<0$ and the result from
Proposition \ref{prop1} that $C_{NN}C_{SS}<C_{SN}C_{NS}$, I can show that the
sum of the first two terms are positive. Thus, $\Xi>0$ and $K_{N}^{e}%
/L_{N}^{e}>K_{S}^{e}/L_{S}^{e}$. Relative factor prices diverge after MP liberalization.
\end{proof}

\section{Data Appendix}

\subsection{Construction of asset deflators}

Since I only use the 2012 firm-level data, I cannot perform a perpetual
inventory method to calculate the real stock of capital. Consider the
perpetual inventory method for a typical firm in country $l$:%
\[
\tilde{K}_{lt}=I_{lt}+\frac{P_{lt}^{I}}{P_{lt-1}^{I}}\left(  1-\delta
_{l}\right)  \tilde{K}_{lt-1},
\]
where $I_{lt}$ is the \textbf{value} of investment in period $t$ at the price
of $P_{lt}^{I}$ and $\tilde{K}_{lt}$ is the \textbf{value} of capital stock at
the price of $P_{lt}^{I}$, at the end of period $t$. $\delta_{l}$ is the
country specific discount rate. Iterate backwards,%
\begin{align*}
\tilde{K}_{it}  &  =I_{lt}+\frac{P_{lt}^{I}}{P_{lt-1}^{I}}\left(  1-\delta
_{l}\right)  \left[  I_{lt-1}+\frac{P_{lt-1}^{I}}{P_{lt-2}^{I}}\left(
1-\delta_{l}\right)  \tilde{K}_{lt-2}\right] \\
&  =\sum_{j=0}^{\infty}\left(  1-\delta_{l}\right)  ^{j}\frac{P_{lt}^{I}%
}{P_{lt-j}^{I}}I_{lt-j}.
\end{align*}


The real stock of capital equals%
\[
K_{lt}\equiv\frac{\tilde{K}_{lt}}{P_{lt}^{I}}=\sum_{j=0}^{\infty}\left(
1-\delta_{l}\right)  ^{j}\frac{1}{P_{lt-j}^{I}}I_{lt-j}.
\]


However, in practice, the book value of past investment are not adjusted as
investment price changes over time. I approximate the book value of total
assets as follows%
\[
\tilde{K}_{lt}^{acct}\equiv I_{lt}+\left(  1-\delta_{l}\right)  \tilde
{K}_{lt-1}^{acct}=\sum_{j=0}^{\infty}\left(  1-\delta_{l}\right)  ^{j}%
I_{lt-j}.
\]
Crozet and Trionfetti (2013) simply deflate the accounting value of capital
stock using the price of investment (PWT 8.0, US 2005 = 1) and get%
\[
K_{lt}^{CT}\equiv\tilde{K}_{lt}^{acct}/P_{lt}^{I}=\sum_{j=0}^{\infty}\left(
1-\delta\right)  ^{j}\frac{1}{P_{lt}^{I}}I_{lt-j},
\]
which tends to underestimate the real capital stock if there is constant
inflation in investment prices. To properly adjust the real capital stock, I
assume that the economies are in steady states and \emph{the real capital
stock grows at a constant rate }$g_{l}$. This implies%
\begin{align*}
K_{lt}  &  =\left(  1+g_{l}\right)  K_{lt-1}=I_{lt}/P_{lt}^{I}+\left(
1-\delta_{l}\right)  K_{lt-1}\\
&  \Rightarrow I_{lt}/P_{lt}^{I}=\left(  g_{l}+\delta_{l}\right)  K_{lt-1}.
\end{align*}
Thus real investment grows at the same speed as capital stock. Also assume the
investment prices grow at constant rates $\pi_{l}$. I can rewrite the real
capital stock as
\begin{align*}
K_{lt}  &  =\sum_{j=0}^{\infty}\left(  1-\delta_{l}\right)  ^{j}\frac
{1}{P_{lt-j}^{I}}I_{lt-j}\\
&  =\frac{I_{lt}}{P_{lt}^{I}}\sum_{j=0}^{\infty}\left(  \frac{1-\delta_{l}%
}{1+g_{l}}\right)  ^{j}\\
&  =\frac{I_{lt}}{P_{lt}^{I}}\frac{1+g_{l}}{\delta_{l}+g_{l}},
\end{align*}
and
\begin{align*}
\tilde{K}_{lt}^{acct}  &  =\sum_{j=0}^{\infty}\left(  \frac{1-\delta_{l}%
}{1+g_{l}}\right)  ^{j}\frac{I_{lt}}{P_{lt}^{I}}P_{lt-j}^{I}\\
&  =\frac{I_{lt}}{P_{lt}^{I}}\sum_{j=0}^{\infty}P_{lt}^{I}\left(
\frac{1-\delta_{l}}{\left(  1+\pi_{l}\right)  \left(  1+g_{l}\right)
}\right)  ^{j}\\
&  =I_{lt}\frac{\left(  1+\pi_{l}\right)  \left(  1+g_{l}\right)  }{\left(
1+\pi_{l}\right)  \left(  1+g_{l}\right)  -\left(  1-\delta_{l}\right)  }.
\end{align*}
Thus the proper deflator is%
\[
\frac{\tilde{K}_{lt}^{acct}}{K_{lt}}=P_{lt}^{I}\frac{\left(  1+\pi_{l}\right)
\left(  \delta_{l}+g_{l}\right)  }{\pi_{l}+g_{l}+\pi_{l}g_{l}+\delta_{l}}.
\]
If the life span of firms is finite, say $T$, then the deflator should be%
\begin{align*}
\frac{\tilde{K}_{lt}^{acct}}{K_{lt}}  &  =P_{lt}^{I}\frac{\sum_{j=0}%
^{T-1}\left(  \frac{1-\delta_{l}}{\left(  1+\pi_{l}\right)  \left(
1+g_{l}\right)  }\right)  ^{j}}{\sum_{j=0}^{T-1}\left(  \frac{1-\delta_{l}%
}{1+g_{l}}\right)  ^{j}}\\
&  =P_{lt}^{I}\frac{\left(  1+\pi_{l}\right)  \left(  \delta_{l}+g_{l}\right)
}{\pi_{l}+g_{l}+\pi_{l}g_{l}+\delta_{l}}\frac{1-\left(  \frac{1-\delta_{l}%
}{\left(  1+\pi_{l}\right)  \left(  1+g_{l}\right)  }\right)  ^{T}}{1-\left(
\frac{1-\delta_{l}}{1+g_{l}}\right)  ^{T}}.
\end{align*}


In practice, I calculate $g_{l}$ and $\pi_{l}$ using a log-linear regression
of real capital stock and investment prices on time for the sample period
1990-2011, respectively. Then I extrapolate $P_{l,2012}^{I}$ from 2000-2011 to
2012 using country-specific growth rates. Firm age in all countries is assumed
to be 10 years.

In the reduced-form regressions in Section 2, it does not matter what deflator
I use since it is country specific and will be absorbed by the
country-industry fixed effects. However, when it comes to the estimation of
the intensive elasticity, the capital-labor ratio has to be comparable across
countries. For the estimate used in calibration, I assume firm age to be 10
years in all countries. I experiment with different $T$ and the results are
robust. (See Table A7)

\section{Tables}%

%TCIMACRO{\TeXButton{reset_counter}{\setcounter{table}{0}
%\renewcommand{\thetable}{A\arabic{table}}}}%
%BeginExpansion
\setcounter{table}{0}
\renewcommand{\thetable}{A\arabic{table}}%
%EndExpansion
%

%TCIMACRO{\TeXButton{frequency_by_host}{\newpage\input
%{tables/frequency_by_host.tex}}}%
%BeginExpansion
\newpage\input{tables/frequency_by_host.tex}%
%EndExpansion
%

%TCIMACRO{\TeXButton{frequency_by_home}{\newpage\input
%{tables/frequency_by_home.tex}}}%
%BeginExpansion
\newpage\input{tables/frequency_by_home.tex}%
%EndExpansion
%

%TCIMACRO{\TeXButton{tech_origin_mne}{\newpage\input
%{tables/tech_origin_mne.tex}}}%
%BeginExpansion
\newpage\input{tables/tech_origin_mne.tex}%
%EndExpansion
%

%TCIMACRO{\TeXButton{tech_origin_mne_common_sample}{\newpage\input
%{tables/tech_origin_mne_common_sample.tex}}}%
%BeginExpansion
\newpage\input{tables/tech_origin_mne_common_sample.tex}%
%EndExpansion
%

%TCIMACRO{\TeXButton{tech_origin_foreignLink}{\newpage\input
%{tables/tech_origin_foreignLink.tex}}}%
%BeginExpansion
\newpage\input{tables/tech_origin_foreignLink.tex}%
%EndExpansion
%

%TCIMACRO{\TeXButton{alter_tech_origin}{\newpage\input
%{tables/alter_tech_origin.tex}}}%
%BeginExpansion
\newpage\input{tables/alter_tech_origin.tex}%
%EndExpansion
%

%TCIMACRO{\TeXButton{ind_def}{\newpage\input{tables/ind_def.tex}}}%
%BeginExpansion
\newpage\input{tables/ind_def.tex}%
%EndExpansion
%

%TCIMACRO{\TeXButton{elas_int}{\newpage\input{tables/elas_int.tex}}}%
%BeginExpansion
\newpage\input{tables/elas_int.tex}%
%EndExpansion
%

%TCIMACRO{\TeXButton{cf_unilateral_mp_lib}{\newpage\input
%{tables/cf_unilateral_mp_lib.tex}}}%
%BeginExpansion
\newpage\input{tables/cf_unilateral_mp_lib.tex}%
%EndExpansion
%

%TCIMACRO{\TeXButton{cf_baseline}{\newpage\input{tables/cf_baseline.tex}}}%
%BeginExpansion
\newpage\input{tables/cf_baseline.tex}%
%EndExpansion
%

%TCIMACRO{\TeXButton{calib_cbp_only}{\newpage\input{tables/calib_cbp_only.tex}%
%}}%
%BeginExpansion
\newpage\input{tables/calib_cbp_only.tex}%
%EndExpansion
%

%TCIMACRO{\TeXButton{cf_cbp_only}{\newpage\input{tables/cf_cbp_only.tex}}}%
%BeginExpansion
\newpage\input{tables/cf_cbp_only.tex}%
%EndExpansion
%

%TCIMACRO{\TeXButton{calib_exogenous_toe}{\newpage\input
%{tables/calib_exogenous_toe.tex}}}%
%BeginExpansion
\newpage\input{tables/calib_exogenous_toe.tex}%
%EndExpansion
%

%TCIMACRO{\TeXButton{cf_exogenous_toe}{\newpage\input
%{tables/cf_exogenous_toe.tex}}}%
%BeginExpansion
\newpage\input{tables/cf_exogenous_toe.tex}%
%EndExpansion
%

%TCIMACRO{\TeXButton{calib_mobileK}{\newpage\input{tables/calib_mobileK.tex}}}%
%BeginExpansion
\newpage\input{tables/calib_mobileK.tex}%
%EndExpansion
%

%TCIMACRO{\TeXButton{cf_mobileK}{\newpage\input{tables/cf_mobileK.tex}}}%
%BeginExpansion
\newpage\input{tables/cf_mobileK.tex}%
%EndExpansion


\section{Figures}%

%TCIMACRO{\TeXButton{reset_counter}{\setcounter{figure}{0}
%\renewcommand{\thefigure}{A\arabic{figure}}}}%
%BeginExpansion
\setcounter{figure}{0}
\renewcommand{\thefigure}{A\arabic{figure}}%
%EndExpansion
%

%TCIMACRO{\TeXButton{cf_cbp_only}{\begin{figure}[ptbh]\caption
%{Counterfactual labor shares: CBP only}\label{fig:cf_labor_share_cbp_only}
%\centering\includegraphics[scale=0.9]{{cf_labor_share_cbp_only.pdf}%
%}\end{figure}}}%
%BeginExpansion
\begin{figure}[ptbh]\caption{Counterfactual labor shares: CBP only}%
\label{fig:cf_labor_share_cbp_only}
\centering\includegraphics[scale=0.9]{{cf_labor_share_cbp_only.pdf}%
}\end{figure}%
%EndExpansion



\end{document}