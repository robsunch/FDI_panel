%2multibyte Version: 5.50.0.2960 CodePage: 65001
% define the absolute path for table fragments

\documentclass[notitlepage,11pt]{article}%
\usepackage{amsmath}
\usepackage{natbib}
\usepackage{color}
\usepackage{geometry}
\usepackage[onehalfspacing]{setspace}
\usepackage{hyperref}
\usepackage{amsfonts}
\usepackage{amssymb}
\usepackage{caption}
\usepackage{tabularx}
\usepackage{graphicx}%
\setcounter{MaxMatrixCols}{30}
%TCIDATA{OutputFilter=latex2.dll}
%TCIDATA{Version=5.50.0.2960}
%TCIDATA{Codepage=65001}
%TCIDATA{CSTFile=aer.cst}
%TCIDATA{Created=Tuesday, July 23, 2013 22:04:26}
%TCIDATA{LastRevised=Tuesday, August 09, 2016 23:44:00}
%TCIDATA{<META NAME="GraphicsSave" CONTENT="32">}
%TCIDATA{<META NAME="SaveForMode" CONTENT="1">}
%TCIDATA{BibliographyScheme=BibTeX}
%TCIDATA{<META NAME="DocumentShell" CONTENT="Standard LaTeX\Blank - Standard LaTeX Article">}
%TCIDATA{Language=American English}
%TCIDATA{ComputeDefs=
%$\lambda$
%}
%BeginMSIPreambleData
\providecommand{\U}[1]{\protect\rule{.1in}{.1in}}
%EndMSIPreambleData
\newtheorem{theorem}{Theorem}
\newtheorem{acknowledgement}[theorem]{Acknowledgement}
\newtheorem{algorithm}[theorem]{Algorithm}
\newtheorem{assumption}{Assumption}
\newtheorem{axiom}[theorem]{Axiom}
\newtheorem{case}[theorem]{Case}
\newtheorem{claim}[theorem]{Claim}
\newtheorem{conclusion}[theorem]{Conclusion}
\newtheorem{condition}[theorem]{Condition}
\newtheorem{conjecture}[theorem]{Conjecture}
\newtheorem{corollary}{Corollary}
\newtheorem{criterion}[theorem]{Criterion}
\newtheorem{definition}{Definition}
\newtheorem{example}[theorem]{Example}
\newtheorem{exercise}[theorem]{Exercise}
\newtheorem{lemma}{Lemma}
\newtheorem{notation}[theorem]{Notation}
\newtheorem{problem}[theorem]{Problem}
\newtheorem{proposition}{Proposition}
\newtheorem{remark}[theorem]{Remark}
\newtheorem{solution}[theorem]{Solution}
\newtheorem{summary}[theorem]{Summary}
\newenvironment{proof}[1][Proof]{\noindent \textbf{#1.} }{\  \rule{0.5em}{0.5em}}
\renewcommand{\bibAnnoteFile}[1]{\IfFileExists{#1}{\begin{quotation}\noindent \texts c{Key:} #1\\
\textsc{Annotation:}\  \input{#1}\end{quotation}}{}}
\renewcommand{\bibAnnote}[2]{\begin{quotation}\noindent \textsc{Key:} #1\\
\textsc{Annotation:}\ #2\end{quotation}}
\newcommand \fnote[1]{\captionsetup{font=small}\caption*{#1}}
\geometry{left=1.5in,right=1.5in,top=1.5in,bottom=1.5in}
\RequirePackage{threeparttable}
\RequirePackage{booktabs}
\makeatletter
\def\input@path{{C:/Users/robsunch/Dropbox/Projects/FDI_panel/writing/paper/v1/}}
\makeatother
\graphicspath{{C:/Users/robsunch/Dropbox/Projects/FDI_panel/writing/paper/v1/figures/}}
\begin{document}

\title{A note on constructing panel data set on multinational production}

\section{{Introduction}}

There is no ready-to-use panel data set on multinational production. I combine
OECD and Eurostat FATS data set to construct one. The primary goal of the data
set is to provide bilateral statistics on total non-financial MP sales,
employment and number of firms used in my job market project : \emph{the
factor bias in multinational production and the labor share}. This note
details the data sources, choice of variables and extrapolation.

\section{Data sources for FATS}

\cite{ramondo_multinational_2015} use UNCTAD as a basic source of FATS
(foreign affiliate statistics) data. However, according to email exchange with
Masataka Fujita, the head of investment trends and issues branch at UNCTAD,
the organization has stopped producing the FATS series. In contrast, Eurostat
and OECD have expanded their effort providing FATS data. I use them as the
major data source for the construction of my panel data.

\subsection{Eurostat}

The following tables are downloaded from Eurostat and used in the construction
of my FATS data.%

\begin{tabular}
[c]{llll}%
Table & Direction & Years & Def of Agg Economy\\
fats\_96/fats\_sum & inward & 1999-2002 & NACE Rev. 1.1 C-K\_X\_J\\
fats\_g1b\_03 & inward & 2003-2007 & NACE Rev. 1.1 C-K\_X\_J\\
fats\_g1b\_08 & inward & 2008-2012 & NACE Rev2 B-N\_S95\_X\_K\\
fats\_out1 & outward & 2004-2006 & NACE Rev1 A-O\_X\_L\\
fats\_out2 & outward & 2007-2009 & NACE Rev1 A-O\_X\_L\\
fats\_out2\_r2 & outward & 2010-2012 & NACE Rev2 B-S\_X\_O
\end{tabular}


Definition of aggregate economy:

\begin{itemize}
\item NACE Rev. 1.1 C-K\_X\_J: Business economy - Industry and services
(except financial intermediation)

\item Nace Rev2 B-N\_S95\_X\_K: Foreign control of enterprises by controlling
countries (from 2008 onwards) NACE Rev 2

\item NACE Rev1 A-O\_X\_L: All NACE activities (except public administration;
activities of households and extra-territorial organizations)

\item Nace Rev2 B-S\_X\_O: Industry, construction and services (except public
administration, defense, compulsory social security)
\end{itemize}

Therefore, the inward FATS aggregate excludes the financial sector, while the
outward statistics include the financial sector, which needs adjustment.

For years before 2002, only 11 countries voluntarily report bilateral FATS
tables. Among them, Ireland and Germany do not report total values (sector
C-K\_X\_J). I aggregate industry level values to get the total. For Ireland, I
aggregate NACE Rev1.1 sectors \textquotedblleft C D E G H I
K\textquotedblright, requiring no missing values in any of the sectors.

Variable definition (inward FATS)%

\begin{tabular}
[c]{lll}%
Original code & Description & new name\\
V11110 & Number of enterprises & n\_ent\\
V12110 & Turnover or gross premiums written & rev\\
V12120 & Production value & prod\_v\\
V12150 & Value added at factor cost & vadd\\
V13110 & Total purchases of goods and services & purchase\\
V13310 & Personnel costs & psn\_cost\\
V15110 & Gross investment in tangible goods & inv\_tangi\\
V16110 & Number of persons employed & n\_psn\_emp\\
V16120 & Number of unpaid persons employed & n\_unpaid\_emp\\
V16130 & Number of employees & n\_emp
\end{tabular}


Variable definitions (outward FATS)%

\begin{tabular}
[c]{llll}%
Original code & Description & new name & inward FATS code\footnote{see FATS
manual P103.}\\
ENT & Number of enterprises & n\_ent & V11110\\
EMP & Number of persons employed & n\_psn\_emp & V16110\\
TUR & Turnover - Million ECU/EUR & rev & V12110
\end{tabular}


\subsubsection{Exchange rate adjustment}

Monetary values are in millions of Euros. For years before 1999, I first use
the table fats\_esms\_an1 to adjust ECU to local currencies and then convert
it to USD using exchange rate from WDI.

\subsection{OECD}

OECD also compiles data on FDI and multinational activities, based on
statistics reported by OECD countries. The multinational firm activity data
can be downloaded from OECD database -- Globalization -- Activity of
Multinationals. There are 12 tables in this section, and four of which focus
on bilateral aggregate statistics, which are used here.%

\begin{tabular}
[c]{lll}%
Table & Years & Industry\\
Inward, ISIC3 (service) & 1995-2008 & 10-74 minus 65-67\\
Inward, ISIC4 & 2008-2013 & sec B to N excl. K\\
Outward, ISIC3 (service) & 1995-2009 & 01-93 minus 75\\
Outward, ISIC4 & 2007-2013 & sec B to S excl. O
\end{tabular}


Note that only inward statistics ISIC4 exclude the financial sector. All the
other statistics need adjustment.

Variable definition (inward and outward, ISIC4)%

\begin{tabular}
[c]{lll}%
Original code & Description & new name\\
ENT & Number of enterprises & n\_ent\\
EMP & Number of persons employed & n\_psn\_emp\\
TUR & Turnover (Mil LCU) & rev\\
EMPE & Number of employees & n\_emp
\end{tabular}


Variable definition (inward and outward, ISIC3)%

\begin{tabular}
[c]{lll}%
Original code & Description & new name\\
NOE & Number of enterprises & n\_ent\\
EMP & Number of employees & n\_emp\\
TUR & Turnover (Mil LCU) & rev
\end{tabular}


\subsubsection{Exchange rate adjustment}

The monetary variables are in millions of local currency. According to the
meta data in the online database,

\begin{quotation}
For Euro area countries, national currency data is expressed in euro beginning
with the year of entry into the Economic and Monetary Union (EMU). For years
prior to the year of entry into EMU, data have been converted from the former
national currency using the appropriate irrevocable conversion rate. This
presentation facilitates comparisons within a country over time and ensures
that the historical evolution is preserved. Please note, however, that pre-EMU
euro are a notional unit and should not be used to form area aggregates or to
carry out cross-country comparisons.
\end{quotation}

\subsubsection{Overlapping years}

Which data shall we use for the overlapping years between ISIC3 and ISIC4?

Concerning inward FATS, for years before 2007 (including 2007), ISIC Rev 3
data provide most of the information, while for years after 2007, ISIC Rev 4
data provide most of the information. To obtain the most comprehensive
bilateral FATS, I use ISIC3 as the primary data in 2007, and complement with
ISIC4 data. For years 2008 and 2009, vice versa.

Concerning inward FATS, I use ISIC4 as the primary data for 2008 and
complement with ISIC3 (very few observations).

\subsubsection{Anomalies}

Germany in 2007 is an outliers. It seems dividing the monetary variables in
2007 makes the time series stable. data on outward MNE sales are outliers in
2007. For example, in the OECD database, table \textquotedblleft Outward
Activity of multinationals by industrial sector (manufacturing) -- ISIC Rev
3\textquotedblright\ shows that Germany's total outward sales (Total Business
Enterprise) is 1.497 million Euros in 2007. This number, however, is 1.486
billion Euros in the first table. Therefore, it seems the undocumented table
\textquotedblleft Outward Activity of Multinationals in ISIC Rev 3
(services)\textquotedblright\ present values that are 1000 times larger. Thus,
I rescale all values for Germany in 2007 by 1000 times. Slovenia is a similar
case. The only difference is that the ISIC3 data contains values of financial
sector revenues after 2007 which are used to complement ISIC4 data. These
years are also misscaled besides 2007.

\subsection{Measure of employment}

In OECD and Eurostat inward tables, there are two variables related to total
employment of multinational firms: (1) number of employees and (2) number of
persons employed. A detailed definition of these two concepts can be found on
Page 56-57 in the Eurostat Foreign AffiliaTes Statistics (FATS)
Recommendations Manual (2012 edition). The manual does not provide a direct
comparison between the two concepts but the key difference between the two
concepts seems to be that one is counted as an employee only when a contract
of employment is provided.

For country-pair-years which both variables are nonmissing, I can calculate
the difference between the two variables. Most of them are exactly the same,
and (2) is in general larger than (1). Since Eurostat outward FATS only
provide (2), I use (2) as the primary data. When possible, I use (2) as a
measure of employment and supplement (1) if (2) is missing or zero but (1) is positive.

\subsection{Exclude financial sector}

In contrast to inward FATS, outward FATS do not exclude financial sector. 

\section{Data sources for FDI}

\subsection{Eurostat}

There are two sources for FDI statistics in Eurostat, TEC tables and BOP tables.

tec00049\qquad Direct investment inward flows by main investing country

tec00051\qquad Direct investment inward stocks by main origin of investment

tec00053\qquad Direct investment outward flows by main destinations

tec00052\qquad Direct investment outward stocks by main destinations

bop\_fdi\_pos\_r2, bop\_fdi\_flow\_r2\qquad EU direct investment positions,
breakdown by country and economic activity (NACE Rev. 2), 2008-2012.

bop\_fdi\_pos, bop\_fdi\_flows\qquad NACE Rev.1, 1994 to 2009

\subsubsection{Exchange rate adjustment}

According to the meta data for BOP\_FDI tables (http://goo.gl/OLT4xr), the
data are in units of Euros/ECU (EUR million starting from 1999, ECU million
before 1999.

\subsection{OECD}

\subsection{UNCTAD}

UNCTAD bilateral FDI stocks and flows, 2001-2012

\section{Construction}

Given MNE activity data from OECD and Eurostat, and FDI stocks and flows data
from OECD, Eurostat and UNCTAD, I try to impute some of the missing bilateral
MNE activity variables (mainly revenue, but employment is also considered). I
use the following steps.

\subsection{Consolidate three sources of FDI statistics}

First I drop outliers by looking at year-to-year growth rates within a country
pair.I look for outliers within a home-host country pair. I compute both the
deviation from the log mean and the log change in a certain variable, and
define an observation to be an outlier if it satisfies two conditions (1) the
log change from last period
%TCIMACRO{\TEXTsymbol{>} }%
%BeginExpansion
$>$
%EndExpansion
5 or
%TCIMACRO{\TEXTsymbol{<} }%
%BeginExpansion
$<$
%EndExpansion
-5, or the log change into next period
%TCIMACRO{\TEXTsymbol{>}}%
%BeginExpansion
$>$%
%EndExpansion
5 or
%TCIMACRO{\TEXTsymbol{<} }%
%BeginExpansion
$<$
%EndExpansion
-5; (2) the deviation from the log mean
%TCIMACRO{\TEXTsymbol{>} }%
%BeginExpansion
$>$
%EndExpansion
5 or
%TCIMACRO{\TEXTsymbol{<} }%
%BeginExpansion
$<$
%EndExpansion
-5. Note that in this way, we first take care of the zeros since they won't
enter the candidates for outliers. Second, the observation adjacent to the
outlier is not likely to be misidentified as an outlier since it will be close
to the mean. There are a few scenarios in which I might fail to identify an
outlier: (1) if an outlier has no adjacent observations (2) if the value of
the outlier is large enough to make the average close to itself, so it does
not satisfy condition (2) and will not be identified as outlier. However, the
adjacent values might be identified as an outlier since they are likely to be
away from the mean. In the tables in \textquotedblleft%
\texttt{output/data\_management/tables/potential\_outliers.xlsx}%
\textquotedblright, I list all the country pairs which have at least one
observation satisfying condition (2).

Next, I supplement missing reporting countries in Eurostat with OECD. The idea
is to use Eurostat as the primary data source while OECD as secondary when the
country is not a reporting country in Eurostat.

Now I am ready to impute missing values. I proceed in several steps:

\begin{enumerate}
\item Impute additional zeros (or positive numbers) in employment, number of
enterprises and revenue using the same variable in the secondary data source (OECD).

\item Impute additional zeros in employment, number of enterprises and revenue
if FDI stock (same direction, inward or outward) is zero.

\item From now on, the focus is to impute missing revenue. The imputation of
employment and number of enterprises are done.

In this step, I impute revenue to be zero if at least one of employment or
number of enterprises is zero, and the other is missing or zero.

\item Impute additional zeros in revenue using variables of the opposite
direction. For example, I impute inward revenue from A to B as zero if outward
revenue, employment and number of enterprises from A to B (reported by A) have
at least one non-positive and the others are missing or non-positive. It also
requires inward stock to be missing.

\item After step 4, the imputation of inward and outward revenue is done
independently. From step 5, I will use outward revenue to impute missing
inward revenue. Extrapolate still missing sales data with inward FDI stocks
(and outward revenue) using both cross-sectional and over time variation. For
the period 1995 -- 2012, I run the following regression for host and home
countries with at least three observations%
\[
\log X_{ilt}=\beta\log Q_{ilt}+\delta_{i}+\delta_{l}+\delta_{i}\times
t+\delta_{l}\times t+\varepsilon_{ilt},
\]
where the dependent variable is the inward sales while the key independent
variable is either inward stock or outward sales. I estimate this regression
using different time periods and the coefficient $\beta$ seems very stable.
(see \texttt{output/data\_management/extrap\_activities.csv})

\item Next I consider using the time-series property of the data only and do
not bring in any new variable besides inward sales. I extrapolate over time
using a constant growth model within each pair. For country pairs with at
least 4 observations between 2001 and 2012, I estimate the following equation%
\[
logX_{ilt}=δ _{il}+δ _{t}+δ _{i}\times t+δ _{l}\times
t+ε _{ilt}%
\]


Note that I do not impose a pair-specific trend since the trend can be
imprecisely estimated with only a few observations within a pair. Instead, I
impose the growth rates to have host and home specific components, and also a
global trend, which is not restricted to be linear.

\item Impute additional zeros by identifying missing values between
consecutive runs of missing values or zeros (start and end with zeros). In
practice, I require the missing values can potentially be replaced as zeros,
i.e., it cannot have positive stock and other MNE activities (inward
employment, number of enterprises and outward employment, sales and number of
enterprises). After such runs are identified, I identify missing values that
are \textquotedblleft squeezed\textquotedblright\ between two zeros. These
values are replaced with zeros.
\end{enumerate}

\subsection{Extrapolation for total inward activities}

The extrapolation for total inward activities is a bit easier. The procedures
are similar to the extrapolation of bilateral activities. I describe the
procedures as follows.

\begin{enumerate}
\item Drop outliers in total inward/outward variables defined using
year-to-year growth rates

I compute both the deviation from the log mean and the log change in a certain
variable, and define an observation to be an outlier if it satisfies two
conditions (1) the log change from last period
%TCIMACRO{\TEXTsymbol{>} }%
%BeginExpansion
$>$
%EndExpansion
5 or
%TCIMACRO{\TEXTsymbol{<} }%
%BeginExpansion
$<$
%EndExpansion
-5, or the log change into next period
%TCIMACRO{\TEXTsymbol{>}}%
%BeginExpansion
$>$%
%EndExpansion
5 or
%TCIMACRO{\TEXTsymbol{<} }%
%BeginExpansion
$<$
%EndExpansion
-5; (2) the deviation from the log mean
%TCIMACRO{\TEXTsymbol{>} }%
%BeginExpansion
$>$
%EndExpansion
5 or
%TCIMACRO{\TEXTsymbol{<} }%
%BeginExpansion
$<$
%EndExpansion
-5.

\item Supplement missing data in Eurostat with OECD -- document them well

First, identify countries that report Eurostat or OECD. If a country reports
in Eurostat, use Eurostat as the primary source. If a country reports in OECD,
use OECD as the primary source.

\item Impute additional zeros using FDI stocks

If one of the Eurostat or OECD stock is non-positive, and the other is missing
or non-positive, impute the missing MNE activities (employment or revenue) to
be zero. If employment is zero, impute revenue to be zero too (very few observations).

\item Extrapolate still missing sales data with inward FDI stocks and
employment using both cross-sectional and over time variation. For the period
1995 -- 2012, I run the following regression for host and home countries with
at least three observations%
\[
\log X_{lt}=\beta\log Q_{lt}+\delta_{l}+\delta_{l}\times t+\delta
_{t}+\varepsilon_{lt},
\]
where the dependent variable is the inward sales while the key independent
variable is either inward stock or employment. I estimate this regression
using different time periods and the coefficient $\beta$ seems very stable.
(see \texttt{output/data\_management/extrap\_tot\_in\_activities.csv})

\item Next I consider using the time-series property of the data only and do
not bring in any new variable besides inward sales. I extrapolate over time
using a constant growth model within each pair. For country pairs with at
least 6 observations between 2001 and 2012, I estimate the following equation%
\[
\log X_{lt}=\delta_{l}+\delta_{t}+\delta_{l}\times t+\varepsilon_{lt}.
\]
Besides the linear growth model extrapolation for observations with positive
sales, I also impute additional zeros using the time series data. I first
identify consecutive runs of missing values or zeros. I require the missing
values can potentially be replaced as zeros, i.e., it cannot have positive
stock and other MNE activities (inward employment, number of enterprises and
outward employment, sales and number of enterprises). \qquad After such runs
are identified, I identify missing values that are \textquotedblleft
squeezed\textquotedblright\ between two zeros. These values are replaced with zeros.
\end{enumerate}

\subsection{Inward v.s. Outward}

Ramondo, Rodriguez-Clare and Tintelnot (2015) give two reasons for using
outward sales as the primary source. First, they argue statistics reported by
th host country is more likely on \textquotedblleft immediate
owners\textquotedblright\ rather than \textquotedblleft ultimate beneficiary
owners (UBO)\textquotedblright. Second, sales reported by the host country may
be only for local sales and miss sales from all other countries.

This may be true for their UNCTAD data (though I highly doubt since some of
the UNCTAD data should come from Eurostat and OECD). However, I cannot find
support for their arguments in FATS statistical manuals. For the first
argument, I found the following related points in the relevant manuals

\begin{itemize}
\item UNCTAD Manual

\begin{itemize}
\item Inward FATS: As far as possible, it is recommended that countries use
the UBO unit when compiling operational statistics for inward investment
(Volume 2, II.39)\footnote{The footnote in that paragraph reads: Out of 15
countries providing operational data in the OECD's Manual (OECD, 2001) eight
(Belgium, France, Germany, Japan, Luxembourg, Norway, Poland and Portugal) use
immediate foreign owner, and seven (Finland, Ireland, Italy, the Netherlands,
Sweden, the United Kingdom and the United States) use ultimate owner.}

\item Outward FATS: The second issue deals with the treatment of foreign
investments of those domestic enterprises, which are themselves foreign-owned.
This volume recommends that the compiling country should collect data for all
resident enterprises direct investor, regardless of where they are owned.
However, in its published statistics it should provide separate breakdowns for
the foreign affiliates of domestically and foreign-owned enterprises. (Volume
2, II.41)
\end{itemize}

\item FATS Manual (Eurostat)

\begin{itemize}
\item Inward FATS: Ultimate Control Institutional Unit (UCI) is recommended
(see I.1.1)

\item Outward FATS: Ultimate Control Institutional Unit (UCI) is recommended
(see I.1.1)
\end{itemize}
\end{itemize}

UNCTAD recommends that countries use the UBO unit when compiling operational
statistics for inward investment (activities), but IMF does require that BOP
statistics record transactions based on the immediate foreign owner (II.39,
II.40 in UNCTAD, 2008, Vol 2). On the contrary, UNCTAD recommends countries
report outward MNE activities based on immediate owners (II.41 (ii)). I simply
cannot find any information about their second argument.

\section{Descriptive statistics}

In this section I describe the data I have constructed.

\bibliographystyle{econometrica}
\bibliography{myLib160429}


\appendix{}

\section{Tables}%

%TCIMACRO{\TeXButton{reset_counter}{\setcounter{table}{0}
%\renewcommand{\thetable}{A\arabic{table}}}}%
%BeginExpansion
\setcounter{table}{0}
\renewcommand{\thetable}{A\arabic{table}}%
%EndExpansion
%

%TCIMACRO{\TeXButton{oecd_overlap_years_coverage_out}{\newpage\input
%{tables/oecd_overlap_years_coverage_out.tex}}}%
%BeginExpansion
\newpage\begin{table}[htbp]\centering
\caption{Number of observations (origin*destination) from each outward MP dataset}
\begin{tabular}{l*{4}{c}}
\toprule
            &           1&           2&           3&       Total\\
\midrule
2006        &           0&        1059&           0&        1059\\
2007        &          13&         844&         241&        1098\\
2008        &         241&         108&         391&         740\\
2009        &         560&          10&         199&         769\\
2010        &        4934&           0&           0&        4934\\
Total       &        5748&        2021&         831&        8600\\
\bottomrule
\end{tabular}
\end{table}
%
%EndExpansion
%

%TCIMACRO{\TeXButton{oecd_overlap_years_diff_out}{\input
%{tables/oecd_overlap_years_diff_out.tex}}}%
%BeginExpansion
\begin{table}[h]\scriptsize\caption{Diff total outward isic3 v.s. isic4}\centering
\begin{threeparttable}\begin{tabular}{l*{10}c}\toprule
            &       count&        mean&          sd&         min&         p10&         p25&         p50&         p75&         p90&         max\\
\midrule
diff\_log\_n\_ent&         463&      0.0106&      0.0617&      -0.693&           0&           0&           0&           0&      0.0274&       0.693\\
diff\_log\_n\_emp&         294&     0.00738&      0.0289&     -0.0242&           0&           0&           0&           0&      0.0198&       0.249\\
diff\_log\_rev&         357&    -0.00227&      0.0756&      -0.620&    -0.00889&           0&           0&     0.00165&      0.0187&       0.487\\
\bottomrule\end{tabular}\begin{tablenotes}
\item[a] Diff in log points (isic3 - isic4).
\end{tablenotes}\end{threeparttable}\end{table}
%
%EndExpansion
%

%TCIMACRO{\TeXButton{oecd_overlap_years_coverage_in}{\newpage\input
%{tables/oecd_overlap_years_coverage_in.tex}}}%
%BeginExpansion
\newpage\begin{table}[htbp]\centering
\caption{Number of observations (origin*destination) from each outward MP dataset}
\begin{tabular}{l*{4}{c}}
\toprule
            &           1&           2&           3&       Total\\
\midrule
2007        &           0&         667&           0&         667\\
2008        &        4000&          31&          34&        4065\\
2009        &        4389&           0&           0&        4389\\
Total       &        8389&         698&          34&        9121\\
\bottomrule
\end{tabular}
\end{table}
%
%EndExpansion
%

%TCIMACRO{\TeXButton{oecd_overlap_years_diff_in}{\input
%{tables/oecd_overlap_years_diff_in.tex}}}%
%BeginExpansion
\begin{table}[h]\scriptsize\caption{Diff nonfin inward isic3 v.s. isic4}\centering
\begin{threeparttable}\begin{tabular}{l*{10}c}\toprule
            &       count&        mean&          sd&         min&         p10&         p25&         p50&         p75&         p90&         max\\
\midrule
diff\_log\_n\_ent&          33&       0.115&       0.128&           0&           0&           0&      0.0966&       0.160&       0.262&       0.613\\
diff\_log\_n\_emp&          33&     0.00752&       0.127&      -0.135&     -0.0860&     -0.0648&     -0.0156&      0.0200&       0.170&       0.439\\
diff\_log\_rev&          33&      0.0956&       0.253&     -0.0789&     -0.0243&     0.00244&      0.0194&      0.0594&       0.237&       1.067\\
\bottomrule\end{tabular}\begin{tablenotes}
\item[a] Diff in log points isic3 - isic4.
\end{tablenotes}\end{threeparttable}\end{table}
%
%EndExpansion
%

%TCIMACRO{\TeXButton{emp_psn_emp_cover_oecd_in_tot}{\newpage\input
%{tables/emp_psn_emp_cover_oecd_in_tot.tex}}}%
%BeginExpansion
\newpage\begin{table}[htbp]\centering
\caption{Number of observations with nonmissing values OECD inward employment tot}
\begin{tabular}{l*{4}{c}}
\toprule
            &    EMP only&PSN EMP only&        Both&       Total\\
\midrule
1995        &         257&           0&           0&         257\\
1996        &         189&           0&           0&         189\\
1997        &         288&           0&           0&         288\\
1998        &         397&           0&           0&         397\\
1999        &         275&           0&           0&         275\\
2000        &         390&           0&           0&         390\\
2001        &         430&           0&           0&         430\\
2002        &         499&           0&           0&         499\\
2003        &         332&           0&           0&         332\\
2004        &         302&           0&           0&         302\\
2005        &         388&           0&           0&         388\\
2006        &         438&           0&           0&         438\\
2007        &         294&           0&           0&         294\\
2008        &         148&          13&          80&         241\\
2009        &         120&          13&          24&         157\\
2010        &         102&          66&          22&         190\\
2011        &         108&          66&           3&         177\\
2012        &         100&          12&          34&         146\\
2013        &           0&           1&           0&           1\\
Total       &        5057&         171&         163&        5391\\
\bottomrule
\end{tabular}
\end{table}
%
%EndExpansion
%

%TCIMACRO{\TeXButton{emp_psn_emp_cover_oecd_in_totXfin}{\input
%{tables/emp_psn_emp_cover_oecd_in_totXfin.tex}}}%
%BeginExpansion
\begin{table}[htbp]\centering
\caption{Number of observations with nonmissing values OECD inward employment totXfin}
\begin{tabular}{l*{4}{c}}
\toprule
            &    EMP only&PSN EMP only&        Both&       Total\\
\midrule
2003        &           1&           0&           0&           1\\
2004        &           1&           0&           0&           1\\
2005        &           2&           0&           0&           2\\
2006        &          31&           0&           0&          31\\
2007        &          69&           0&           0&          69\\
2008        &           5&        2956&          80&        3041\\
2009        &           8&        3432&           3&        3443\\
2010        &           3&        3294&          22&        3319\\
2011        &           5&        3622&           3&        3630\\
2012        &           6&        3751&          32&        3789\\
2013        &           0&           1&           0&           1\\
Total       &         131&       17056&         140&       17327\\
\bottomrule
\end{tabular}
\end{table}
%
%EndExpansion
%

%TCIMACRO{\TeXButton{emp_psn_emp_diff_oecd_in}{\input
%{tables/emp_psn_emp_diff_oecd_in.tex}}}%
%BeginExpansion
\begin{table}[h]\scriptsize\caption{Diff between psn emp and emp - oecd inward}\centering
\begin{threeparttable}\begin{tabular}{l*{10}c}\toprule
            &       count&        mean&          sd&         min&         p10&         p25&         p50&         p75&         p90&         max\\
\midrule
diff\_log\_oecd\_in\_tot&         111&     -0.0389&       0.243&      -0.375&      -0.223&      -0.147&    -0.00131&           0&           0&       2.274\\
diff\_log\_oecd\_in\_totXfin&          97&    -0.00424&      0.0399&      -0.276&    -0.00446&    -0.00119&           0&           0&           0&       0.170\\
\bottomrule\end{tabular}\begin{tablenotes}
\item[a] Diff in log points (emp - psn emp).
\end{tablenotes}\end{threeparttable}\end{table}
%
%EndExpansion
%

%TCIMACRO{\TeXButton{emp_psn_emp_cover_oecd_out_tot}{\newpage\input
%{tables/emp_psn_emp_cover_oecd_out_tot.tex}}}%
%BeginExpansion
\newpage\begin{table}[htbp]\centering
\caption{Number of observations with nonmissing values OECD outward employment tot}
\begin{tabular}{l*{4}{c}}
\toprule
            &    EMP only&PSN EMP only&        Both&       Total\\
\midrule
1995        &         134&           0&           0&         134\\
1996        &         149&           0&           0&         149\\
1997        &         298&           0&           0&         298\\
1998        &         314&           0&           0&         314\\
1999        &         332&           0&           0&         332\\
2000        &         460&           0&           0&         460\\
2001        &         445&           0&           0&         445\\
2002        &         561&           0&           0&         561\\
2003        &         549&           0&           0&         549\\
2004        &         636&           0&           0&         636\\
2005        &         710&           0&           0&         710\\
2006        &         665&           0&           0&         665\\
2007        &         817&           0&          41&         858\\
2008        &         211&         189&         195&         595\\
2009        &         240&         209&         195&         644\\
2010        &         183&        3425&           0&        3608\\
2011        &         185&        3460&           0&        3645\\
2012        &         181&        3437&           0&        3618\\
Total       &        7070&       10720&         431&       18221\\
\bottomrule
\end{tabular}
\end{table}
%
%EndExpansion
%

%TCIMACRO{\TeXButton{emp_psn_emp_cover_oecd_out_fin}{\input
%{tables/emp_psn_emp_cover_oecd_out_fin.tex}}}%
%BeginExpansion
\begin{table}[htbp]\centering
\caption{Number of observations with nonmissing values OECD outward employment fin}
\begin{tabular}{l*{4}{c}}
\toprule
            &    EMP only&PSN EMP only&        Both&       Total\\
\midrule
1995        &          37&           0&           0&          37\\
1996        &          12&           0&           0&          12\\
1997        &         138&           0&           0&         138\\
1998        &         106&           0&           0&         106\\
1999        &         154&           0&           0&         154\\
2000        &         204&           0&           0&         204\\
2001        &         181&           0&           0&         181\\
2002        &         218&           0&           0&         218\\
2003        &         160&           0&           0&         160\\
2004        &         212&           0&           0&         212\\
2005        &         353&           0&           0&         353\\
2006        &         344&           0&           0&         344\\
2007        &         383&           0&           1&         384\\
2008        &         112&           4&           1&         117\\
2009        &          60&           6&           1&          67\\
2010        &           2&          20&           0&          22\\
2011        &           2&          19&           0&          21\\
2012        &           2&          19&           0&          21\\
Total       &        2680&          68&           3&        2751\\
\bottomrule
\end{tabular}
\end{table}
%
%EndExpansion
%

%TCIMACRO{\TeXButton{emp_psn_emp_diff_oecd_out}{\input
%{tables/emp_psn_emp_diff_oecd_out.tex}}}%
%BeginExpansion
\begin{table}[h]\scriptsize\caption{Diff between psn emp and emp - oecd outward}\centering
\begin{threeparttable}\begin{tabular}{l*{10}c}\toprule
            &       count&        mean&          sd&         min&         p10&         p25&         p50&         p75&         p90&         max\\
\midrule
diff\_log\_oecd\_out\_tot&         142&           0&           0&           0&           0&           0&           0&           0&           0&           0\\
diff\_log\_oecd\_out\_fin&           3&     -0.0682&      0.0438&     -0.0952&     -0.0952&     -0.0952&     -0.0916&     -0.0177&     -0.0177&     -0.0177\\
\bottomrule\end{tabular}\begin{tablenotes}
\item[a] Diff in log points (emp - psn emp).
\end{tablenotes}\end{threeparttable}\end{table}
%
%EndExpansion
%

%TCIMACRO{\TeXButton{emp_psn_emp_cover_es_in_totXfin}{\newpage\input
%{tables/emp_psn_emp_cover_es_in_totXfin.tex}}}%
%BeginExpansion
\newpage\begin{table}[htbp]\centering
\caption{Number of observations with nonmissing values Eurostat inward employment totXfin}
\begin{tabular}{l*{4}{c}}
\toprule
            &    EMP only&PSN EMP only&        Both&       Total\\
\midrule
1998        &           1&           0&           0&           1\\
1999        &           8&           0&           9&          17\\
2000        &          11&           0&         201&         212\\
2001        &           2&           0&         356&         358\\
2002        &           0&           0&         658&         658\\
2003        &         215&         264&        1660&        2139\\
2004        &         212&         436&        2009&        2657\\
2005        &         207&         609&        1904&        2720\\
2006        &          64&          40&        2807&        2911\\
2007        &           2&         575&        2879&        3456\\
2008        &           1&        1517&        3381&        4899\\
2009        &           0&        1852&        3296&        5148\\
2010        &           1&        2228&        3142&        5371\\
2011        &           0&        2639&        2917&        5556\\
2012        &           1&        2652&        2910&        5563\\
Total       &         725&       12812&       28129&       41666\\
\bottomrule
\end{tabular}
\end{table}
%
%EndExpansion
%

%TCIMACRO{\TeXButton{emp_psn_emp_diff_es_in}{\input
%{tables/emp_psn_emp_diff_es_in.tex}}}%
%BeginExpansion
\begin{table}[h]\scriptsize\caption{Diff between psn emp and emp - eurostat inward}\centering
\begin{threeparttable}\begin{tabular}{l*{10}c}\toprule
            &       count&        mean&          sd&         min&         p10&         p25&         p50&         p75&         p90&         max\\
\midrule
diff\_log\_es\_in\_totXfin&        6862&     -0.0533&       0.172&      -4.394&      -0.129&     -0.0460&    -0.00664&           0&           0&       0.160\\
\bottomrule\end{tabular}\begin{tablenotes}
\item[a] Diff in log points (emp - psn emp).
\end{tablenotes}\end{threeparttable}\end{table}
%
%EndExpansion
%

%TCIMACRO{\TeXButton{exclude_fin_oecd_in}{\newpage\input
%{tables/exclude_fin_oecd_in.tex}}}%
%BeginExpansion
\newpage\begin{table}[htbp]\centering
\caption{Cases excluding financial sector in different steps (OECD inward)}
\begin{tabular}{l*{4}{c}}
\toprule
            &           0&           1&           2&       Total\\
\midrule
1995        &         110&         150&           0&         260\\
1996        &         108&          85&           0&         193\\
1997        &         143&          81&          75&         299\\
1998        &         191&         126&          92&         409\\
1999        &         143&          95&          45&         283\\
2000        &         152&         194&          52&         398\\
2001        &         204&         232&           0&         436\\
2002        &         257&         211&          53&         521\\
2003        &          59&         202&         106&         367\\
2004        &          92&         180&         120&         392\\
2005        &         101&         222&         103&         426\\
2006        &         158&         267&          57&         482\\
2007        &          99&         218&           0&         317\\
2008        &        3060&         176&          20&        3256\\
2009        &        3486&         139&          20&        3645\\
2010        &        3339&         113&          51&        3503\\
2011        &        3630&         112&          51&        3793\\
2012        &        3840&         129&          53&        4022\\
2013        &           1&           0&           0&           1\\
Total       &       19173&        2932&         898&       23003\\
\bottomrule
\multicolumn{5}{l}{\footnotesize 0 - no adjustment needed}\\
\multicolumn{5}{l}{\footnotesize 1 - adj using total inward nonfinancial share}\\
\multicolumn{5}{l}{\footnotesize 2 - adj using host country nonfinancial output share}\\
\multicolumn{5}{l}{\footnotesize 3 - adj using home country nonfinancial output share}\\
\end{tabular}
\end{table}
%
%EndExpansion
%

%TCIMACRO{\TeXButton{exclude_fin_oecd_out}{\input
%{tables/exclude_fin_oecd_out.tex}}}%
%BeginExpansion
\begin{table}[htbp]\centering
\caption{Cases excluding financial sector in different steps (OECD outward)}
\begin{tabular}{l*{5}{c}}
\toprule
            &           0&           1&           2&           3&       Total\\
\midrule
1995        &          63&          86&          52&          17&         218\\
1996        &          36&         169&           0&           0&         205\\
1997        &         163&         193&          28&          10&         394\\
1998        &         132&         204&          16&          16&         368\\
1999        &         176&         252&           0&           0&         428\\
2000        &         219&         339&           0&           0&         558\\
2001        &         231&         315&           0&           0&         546\\
2002        &         260&         400&           0&           0&         660\\
2003        &         169&         427&           0&           0&         596\\
2004        &         215&         436&           0&           0&         651\\
2005        &         407&         369&           2&           1&         779\\
2006        &         343&         336&           0&           0&         679\\
2007        &         325&         569&           0&           0&         894\\
2008        &         121&         482&          17&          10&         630\\
2009        &          66&         540&          42&          12&         660\\
2010        &          24&        3956&          55&         162&        4197\\
2011        &          23&        3974&          56&         163&        4216\\
2012        &          23&        3846&          44&         302&        4215\\
Total       &        2996&       16893&         312&         693&       20894\\
\bottomrule
\multicolumn{6}{l}{\footnotesize 0 - no adjustment needed}\\
\multicolumn{6}{l}{\footnotesize 1 - adj using total inward nonfinancial share}\\
\multicolumn{6}{l}{\footnotesize 2 - adj using host country nonfinancial output share}\\
\multicolumn{6}{l}{\footnotesize 3 - adj using home country nonfinancial output share}\\
\end{tabular}
\end{table}
%
%EndExpansion
%

%TCIMACRO{\TeXButton{exclude_fin_es_out}{\begin{table}[htbp]\centering
\caption{Cases excluding financial sector in different steps (Eurostat outward)}
\begin{tabular}{l*{5}{c}}
\toprule
            &           0&           1&           2&           3&       Total\\
\midrule
2004        &         287&         198&           0&           0&         485\\
2005        &         399&         224&           0&           0&         623\\
2006        &         441&         398&           0&           1&         840\\
2007        &         549&        1581&          32&         339&        2501\\
2008        &         572&        1766&           3&         325&        2666\\
2009        &         744&        2467&           6&         326&        3543\\
2010        &        1354&        4293&          13&         493&        6153\\
2011        &        1333&        4377&          13&         504&        6227\\
2012        &        1203&        3702&          15&         681&        5601\\
Total       &        6882&       19006&          82&        2669&       28639\\
\bottomrule
\multicolumn{6}{l}{\footnotesize 0 - no adjustment needed}\\
\multicolumn{6}{l}{\footnotesize 1 - adj using total inward nonfinancial share}\\
\multicolumn{6}{l}{\footnotesize 2 - adj using host country nonfinancial output share}\\
\multicolumn{6}{l}{\footnotesize 3 - adj using home country nonfinancial output share}\\
\end{tabular}
\end{table}
%
%}}%
%BeginExpansion
\begin{table}[htbp]\centering
\caption{Cases excluding financial sector in different steps (Eurostat outward)}
\begin{tabular}{l*{5}{c}}
\toprule
            &           0&           1&           2&           3&       Total\\
\midrule
2004        &         287&         198&           0&           0&         485\\
2005        &         399&         224&           0&           0&         623\\
2006        &         441&         398&           0&           1&         840\\
2007        &         549&        1581&          32&         339&        2501\\
2008        &         572&        1766&           3&         325&        2666\\
2009        &         744&        2467&           6&         326&        3543\\
2010        &        1354&        4293&          13&         493&        6153\\
2011        &        1333&        4377&          13&         504&        6227\\
2012        &        1203&        3702&          15&         681&        5601\\
Total       &        6882&       19006&          82&        2669&       28639\\
\bottomrule
\multicolumn{6}{l}{\footnotesize 0 - no adjustment needed}\\
\multicolumn{6}{l}{\footnotesize 1 - adj using total inward nonfinancial share}\\
\multicolumn{6}{l}{\footnotesize 2 - adj using host country nonfinancial output share}\\
\multicolumn{6}{l}{\footnotesize 3 - adj using home country nonfinancial output share}\\
\end{tabular}
\end{table}
%
%EndExpansion
%

%TCIMACRO{\TeXButton{newpage}{\newpage}}%
%BeginExpansion
\newpage
%EndExpansion


\section{Figures}%

%TCIMACRO{\TeXButton{reset_counter}{\setcounter{figure}{0}
%\renewcommand{\thefigure}{A\arabic{figure}}}}%
%BeginExpansion
\setcounter{figure}{0}
\renewcommand{\thefigure}{A\arabic{figure}}%
%EndExpansion
%

%TCIMACRO{\TeXButton{in_output_share_HUN}{\newpage\begin{figure}[ptbh]\caption
%{Inward MP output share in Hungary}\label{fig:in_output_share_HUN}
%\centering\includegraphics[scale=0.9]{{in_output_share_HUN.pdf}}\end{figure}%
%}}%
%BeginExpansion
\newpage\begin{figure}[ptbh]\caption{Inward MP output share in Hungary}%
\label{fig:in_output_share_HUN}
\centering\includegraphics[scale=0.9]{{in_output_share_HUN.pdf}}\end{figure}%
%EndExpansion



\end{document}